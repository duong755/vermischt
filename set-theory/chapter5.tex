\chapter{Số thực}\label{chapter:real-numbers}

\section{Số thực là gì?}

\subsection{Hệ tiên đề về số thực}

Có ít nhất hai cách hiểu cho câu hỏi ``Số thực là gì?\@'' Đầu tiên, chúng ta có thể hiểu rằng người hỏi đang muốn được biết một \textit{định nghĩa toán học} cho số thực. Đó là một định nghĩa hình thức tương tự như định nghĩa số tự nhiên, số nguyên, số hữu tỉ, tính chia hết, \ldots đã được nêu trong tài liệu này. Thứ hai, chúng ta có thể hiểu rằng người hỏi đang muốn liên hệ cái-được-gọi-là-số-thực với thực tế. Nói cách khác, để trả lời cho cách hiểu thứ hai, người trả lời cần định nghĩa số thực như một đối tượng nào đó tương đương trong thực tế hoặc gần thực tế.

Chúng tôi đưa ra một câu trả lời trực giác cho cách hiểu thứ hai như sau: \textit{Hình dung một đường thẳng kéo dài vô tận về hai phía. Mỗi điểm được đánh dấu trên đường thẳng đó tương ứng với một số thực. Nói rõ hơn, trên đường thẳng đó, chúng ta đánh dấu hai điểm khác nhau, lần lượt gọi là $0$ và $1$. Các điểm $x$ trên đường thẳng này tương ứng với một số (thực) âm nếu điểm $0$ nằm giữa $x$ và $1$. Các điểm $x$ trên đường thẳng này tương ứng với một số (thực) dương nếu hoặc $x$ nằm giữa $0$ và $1$, hoặc $x$ trùng $1$, hoặc $1$ nằm giữa $0$ và $x$.}

Câu trả lời trực giác hình học cho cách hiểu thứ hai có thể làm hài lòng nhiều người nhưng là không đủ tốt đối với một định nghĩa toán học. Với định nghĩa trực giác như vậy, chúng ta khó lòng nói về các phép toán với các số thực như cộng, trừ, nhân, chia. Đối với những người học và làm toán, việc biết các số thực và các phép toán với số thực, quan hệ giữa các số thực có những tính chất gì quan trọng hơn việc biết số thực là gì trong thực tế. Trong chương này, chúng ta định nghĩa số thực bằng một hệ tiên đề (hay tính chất). Chúng ta coi những đối tượng thỏa mãn hệ tiên đề (hay tính chất) này là các số thực.

\begin{axiom}
    Tập hợp số thực được kí hiệu là $\mathbb{R}$. Các phần tử của $\mathbb{R}$ thỏa mãn ba nhóm tiên đề sau.

    \textbf{Các tiên đề về trường.} $\mathbb{R}$ có hai phép toán hai ngôi là phép cộng (được kí hiệu là $+$) và phép nhân (được kí hiệu là $\cdot$) và các phép toán này thỏa mãn các tính chất sau:
    \begin{enumerate}[label={(\roman*)}]
        \item Phép cộng có tính chất kết hợp. Nói cách khác, với mọi số thực $x, y, z$, chúng ta có
              \[
                  (x + y) + z = x + (y + z).
              \]
        \item Phép cộng có phần tử đồng nhất. Nói cách khác, tồn tại số thực $0$ sao cho với mọi số thực $x$, chúng ta có
              \[
                  x + 0 = 0 + x = x.
              \]
        \item Mỗi số thực có phần tử đối. Nói cách khác, với mỗi số thực $x$, tồn tại số thực $(-x)$ thỏa mãn
              \[
                  x + (-x) = (-x) + x = 0.
              \]
        \item Phép cộng có tính chất giao hoán. Nói cách khác, với mọi số thực $x, y$, chúng ta có
              \[
                  x + y = y + x.
              \]
        \item Phép nhân có tính chất kết hợp. Nói cách khác, với mọi số thực $x, y, z$, chúng ta có
              \[
                  (x \cdot y) \cdot z = x \cdot (y \cdot z).
              \]
        \item Phép nhân có tính chất phân phối với phép cộng. Nói cách khác, với mọi số thực $x, y, z$, chúng ta có
              \[
                  \begin{split}
                      (x + y)\cdot z = x\cdot z + y\cdot z, \\
                      z\cdot (x + y) = z\cdot x + z\cdot y.
                  \end{split}
              \]

        \item Phép nhân có phần tử đồng nhất, phần tử này khác $0$. Nói cách khác, tồn tại số thực $1\ne 0$ sao cho với mọi số thực $x$, chúng ta có
              \[
                  x + 1 = 1 + x = x.
              \]
        \item Phép nhân có tính chất giao hoán. Nói cách khác, với mọi số thực $x, y$, chúng ta có
              \[
                  x\cdot y = y\cdot x.
              \]
        \item Mỗi số thực khác $0$ có phần tử nghịch đảo. Nói cách khác, với mỗi số thực $x\ne 0$, tồn tại số thực $x^{-1}$ sao cho
              \[
                  x\cdot x^{-1} = x^{-1}\cdot x = 1.
              \]
    \end{enumerate}

    \textbf{Các tiên đề về thứ tự.} $\mathbb{R}$ có quan hệ $\leq$ thỏa mãn các tính chất sau:
    \begin{enumerate}[label={(\roman*)}]
        \item $\leq$ là một quan hệ thứ tự toàn phần.
        \item Với mọi số thực $x, y$, nếu $x\leq y$ thì với mọi số thực $z$, chúng ta có $x + z\leq y + z$.
        \item Với mọi số thực $x, y$, nếu $0\leq x$ và $0\leq y$ thì $0\leq x\cdot y$.
    \end{enumerate}

    \textbf{Tiên đề về cận trên (hay tiên đề về tính đầy đủ).} Nếu một tập hợp con khác rỗng của $\mathbb{R}$ có cận trên thì cũng có cận trên nhỏ nhất.
\end{axiom}

Chúng ta bình luận và làm rõ thêm hệ tiên đề vừa nêu. Các tiên đề về trường và các tiên đề về thứ tự có lẽ không có gì xa lạ với bạn đọc. Chúng tôi chỉ lưu ý thêm ba điều về hai nhóm tiên đề này: (1) Các phần tử $0$, $1$, $(-x)$, và $x^{-1}$ được hiểu như các kí hiệu đơn thuần, ở thời điểm này chúng ta \textit{chưa} coi đó như những số hay phép toán quen thuộc; (2) Tiên đề về sự tồn tại của hai phần tử $0$ và $1$ không khẳng định tính duy nhất của những phần tử như vậy, chúng ta sẽ chứng minh tính duy nhất của hai phần tử đó ở một mục khác trong chương này.

Tiên đề về cận trên là phần ít quen thuộc nhất trong hệ tiên đề trên, và chúng ta cần làm rõ khái niệm được nhắc tới trong tiên đề này: \textit{cận trên} và \textit{cận trên nhỏ nhất}.

\begin{definition}[Cận trên và Cận dưới]
    Cho một tập hợp $S$ được định nghĩa một quan hệ thứ tự bộ phận $\leq$ và $A$ là một tập hợp con của $S$.
    \begin{enumerate}[label={(\roman*)}]
        \item Một phần tử $u$ của $S$ được gọi là một \textbf{cận trên\index{Cận trên}} của $A$ nếu như với mỗi phần tử $a$ của $A$, chúng ta có $a\leq u$. Chúng ta còn nói $A$ bị chặn trên bởi $u$.
        \item Một phần tử $\ell$ của $S$ được gọi là một \textbf{cận dưới\index{Cận dưới}} của $A$ nếu như với mỗi phần tử $a$ của $A$, chúng ta có $\ell\leq a$. Chúng ta còn nói $A$ bị chặn dưới bởi $\ell$.
        \item Tập hợp $A$ được gọi là bị chặn nếu $A$ có cả cận trên và cận dưới.
        \item Một phần tử $x$ của $S$ được gọi là một \textbf{cận trên nhỏ nhất\index{Cận trên nhỏ nhất}} hay \textbf{cận trên đúng\index{Cận trên đúng}} của $A$ nếu như với mỗi cận trên $u$ của $A$, chúng ta có $x\leq u$. Cận trên nhỏ nhất của $A$ được kí hiệu là $\sup A$.
        \item Một phần tử $y$ của $S$ được gọi là một \textbf{cận dưới lớn nhất\index{Cận dưới lớn nhất}} hay \textbf{cận dưới đúng\index{Cận dưới đúng}} của $A$ nếu như với mỗi cận dưới $\ell$ của $A$, chúng ta có $\ell\leq y$. Cận dưới lớn nhất của $A$ được kí hiệu là $\inf A$.
    \end{enumerate}
\end{definition}

Chúng ta theo dõi một số ví dụ.
\begin{example}
    Tập hợp
    \[
        S = \left\{ 1, \frac{1}{2}, \frac{1}{3}, \ldots \right\} = \left\{ \frac{1}{n} \mid \text{$n$ là một số nguyên dương} \right\}
    \]

    là một tập hợp con của tập hợp số hữu tỉ $\mathbb{Q}$. Tập hợp $\mathbb{Q}$ được sắp thứ tự toàn phần.
    \begin{itemize}
        \item $S$ bị chặn trên bởi $1, 2, \frac{5}{2}, 3, \ldots$ và bị chặn dưới bởi $0, \frac{-1}{2}, -1, \ldots$
        \item $1$ là cận trên nhỏ nhất của $S$.
        \item $S$ không có phần tử nhỏ nhất. Bởi vì mỗi phần tử $\frac{1}{m}$ của $S$, luôn có phần tử nhỏ hơn, chẳng hạn $\frac{1}{m+1}, \frac{1}{2m}, \ldots$
        \item $0$ là cận dưới lớn nhất của $S$. Giả sử phản chứng rằng $S$ có một cận dưới lớn hơn $0$. Cận dưới đó (là một số hữu tỉ vì chúng ta đang xét $S$ là tập hợp con của $\mathbb{Q}$). Chúng ta kí hiệu phân số tối giản của cận dưới đó là $\frac{p}{q}$. Nhưng vì $\frac{p}{q}\geq \frac{1}{q} > \frac{1}{2q}$ nên $\frac{p}{q}$ không phải cận dưới của $S$, dẫn đến giả sử phản chứng là sai. Do đó chúng ta khẳng định $0$ là cận dưới lớn nhất của $S$.
        \item $1$ vừa là cận trên nhỏ nhất, vừa là phần tử lớn nhất của $S$. Còn $0$ là cận dưới lớn nhất của $S$ nhưng không thuộc $S$, và do đó không phải phần tử nhỏ nhất của $S$.
    \end{itemize}
\end{example}

Quay lại với hệ tiên đề về số thực. Trong chương trước, chúng ta đã chỉ ra được tập hợp số hữu tỉ $\mathbb{Q}$ cùng với hai phép toán cộng, nhân, và quan hệ $\leq$ thỏa mãn các tiên đề về trường và các tiên đề về thứ tự. Mệnh đề dưới đây cho thấy tập hợp số hữu tỉ không thỏa mãn tiên đề về cận trên, hay chúng ta còn nói rằng tập hợp số hữu tỉ không đầy đủ theo quan hệ thứ tự $\leq$.

\begin{proposition}\label{proposition:irrational-cut}
    Trong tập hợp số hữu tỉ
    \begin{enumerate}[label={(\roman*)}]
        \item Chứng minh rằng không tồn tại số hữu tỉ $x$ nào thỏa mãn $x^{2} = 2$.
        \item Chứng minh rằng tập hợp
              \[
                  S = \{ x \mid x\in\mathbb{Q}, 0 < x \text{ và } x^{2} < 2 \}
              \]

              không có phần tử lớn nhất.
        \item Chứng minh rằng tập hợp $S$ ở phần (ii) không có cận trên nhỏ nhất.
    \end{enumerate}
\end{proposition}

\begin{proof}
    \begin{enumerate}[label={(\roman*)}]
        \item Giả sử phản chứng rằng tồn tại số hữu tỉ $x$ sao cho $x^{2} = 2$. Chúng ta kí hiệu phân số tối giản của $x$ là $\frac{p}{q}$. Vì ${\left(\frac{p}{q}\right)}^{2} = 2$ nên $p^{2} = 2q^{2}$. Vì $2$ là ước của $2q^{2}$ nên $2$ cũng là ước của $p^{2}$. Theo bổ đề Euclid (Định lý~\ref{theorem:euclid-lemma}), chúng ta suy ra $2$ là ước của $p$, do đó tồn tại số tự nhiên $a$ sao cho $p = 2a$. Cùng với việc $p^{2} = 2q^{2}$, chúng ta suy ra $4a^{2} = 2q^{2}$, kéo theo $2a^{2} = q^{2}$. Một lần nữa, theo bổ đề Euclid, chúng ta suy ra $2$ là ước của $q$. Như vậy $2$ là ước chung của $p$ và $q$, điều này mâu thuẫn với việc $\frac{p}{q}$ là một phân số tối giản.

              Vậy không tồn tại số hữu tỉ $x$ nào thỏa mãn $x^{2} = 2$.
        \item Chúng ta chọn một số hữu tỉ $\frac{a}{b}$ thuộc $S$ (trong đó $a, b$ là các số nguyên dương). Theo định nghĩa của $S$, chúng ta suy ra $a^{2} < 2b^{2}$. Xét số hữu tỉ $\frac{2a + 2b}{a + 2b}$.
              \begin{align*}
                  {(2a + 2b)}^{2} & = 4a^{2} + 8ab + 4b^{2}   \\
                                  & < 2a^{2} + 8ab + 8b^{2}   \\
                                  & = 2(a^{2} + 4ab + 4b^{2}) \\
                                  & = 2{(a + 2b)}^{2}
              \end{align*}

              Do đó $\frac{2a + 2b}{a + b}$ là một phần tử của $S$. Bên cạnh đó, $\frac{a}{b} < \frac{2a + 2b}{a + 2b}$, vì
              \begin{align*}
                  a(a + 2b) & = a^{2} + 2ab < 2ab + 2b^{2} = b(2a + 2b)
              \end{align*}

              Như vậy, với mỗi phần tử $x$ thuộc $S$, chúng ta luôn tìm được được một phần tử khác của $S$ nhưng lớn hơn $x$. Do đó tập hợp $S$ không có phần tử lớn nhất.
        \item Tập hợp $S$ có cận trên. Chẳng hạn, $2$ là một cận trên của $S$, bởi vì với mọi $x$ thuộc $S$, $x^{2} < 2 < 4$, kéo theo $(x - 2)(x + 2) < 0$, và $x < 2$. Mặt khác, nếu $y$ là một cận trên của $S$ thì $y > 0$.

              Tiếp theo, chúng ta chứng minh rằng nếu số hữu tỉ $y$ là một cận trên của $S$ thì $y^{2} > 2$. Giả sử phản chứng rằng $y^{2}\leq 2$. Theo phần (i), chúng ta suy ra $y^{2} < 2$, kéo theo $y$ là một phần tử của $S$. $y$ là một cận trên của $S$ và là một phần tử của $S$ thì $y$ cũng là phần tử lớn nhất của $S$. Điều này mâu thuẫn với kết quả đã chứng minh ở phần (ii). Do đó giả sử phản chứng là sai, và chúng ta suy ra $y^{2} > 2$.

              Chọn $y$ là một cận trên của $S$. Chúng ta kí hiệu $\frac{a}{b}$ là phân số của $y$ ($a, b$ là các số nguyên dương). Vì $y^{2} > 2$ nên $a^{2} > 2b^{2}$. Chúng ta tiếp tục xét số hữu tỉ $\frac{2a + 2b}{a + 2b}$.
              \begin{align*}
                  {(2a + 2b)}^{2} & = 4a^{2} + 8ab + 4b^{2}   \\
                                  & > 2a^{2} + 8ab + 8b^{2}   \\
                                  & = 2(a^{2} + 4ab + 4b^{2}) \\
                                  & = 2{(a + 2b)}^{2}
              \end{align*}

              Do đó $\frac{2a + 2b}{a + b}$ là một cận trên của $S$. Ngoài ra, $\frac{2a + 2b}{a + 2b} < \frac{a}{b}$, vì
              \begin{align*}
                  b(2a + 2b) = 2ab + 2b^{2} < 2ab + a^{2} = a(a + 2b)
              \end{align*}

              Như vậy với mỗi cận trên $y$ của $S$, chúng ta luôn tìm được một cận trên khác của $S$ và nhỏ hơn $y$. Do đó tập hợp $S$ không có cận trên nhỏ nhất.
    \end{enumerate}
\end{proof}

\subsection{Dẫn nhập về việc xây dựng tập hợp số thực}

Vì sao cần xây dựng tập hợp số thực? Ở đây chúng tôi dẫn ra khía cạnh giảng dạy và khía cạnh cơ sở toán học. Trong khía cạnh giảng dạy, việc chỉ dẫn cách xây dựng tập hợp số thực sẽ cho thấy mối liên hệ giữa đối tượng mới (tập hợp số thực) với các đối tượng quen thuộc hơn (tập hợp số hữu tỉ, tập hợp số nguyên, tập hợp số tự nhiên). Điều đó có ích hơn so với việc thừa nhận hệ tiên đề và coi tập hợp số thực như một công cụ tiện lợi một cách khó hiểu. Ở khía cạnh cơ sở toán học, những người làm toán cố gắng không thừa nhận quá nhiều thứ. Đúng là trong một thời gian dài, các nhà Toán học vẫn sử dụng phương pháp tiên đề và xuất phát từ các tiên đề cùng các luật logic để chứng minh các định lý. Nhưng tư tưởng của phương pháp tiên đề không chỉ đơn thuần là thừa nhận một số mệnh đề rồi áp dụng luật logic, mà còn là việc xuất phát từ một ít tiên đề rồi từ đó xây dựng nên tất cả những thứ khác. Nếu thừa nhận quá nhiều thì chúng ta chẳng biết được bao nhiêu.

Xây dựng tập hợp số thực là gì? Xây dựng tập hợp số thực là việc tạo ra một đối tượng toán học thỏa mãn hệ tiên đề về số thực. Chúng ta có thể so sánh hệ tiên đề về số thực với một bản thiết kế, khi đó việc xây dựng tập hợp số thực chính là tạo ra một công trình, tác phẩm giống như bản thiết kế đó. Xây dựng được tập hợp số thực đồng nghĩa với việc \textit{chứng minh} bản thiết kế là khả thi. Trong toán học, chúng ta có thuật ngữ riêng để gọi một công trình tương ứng với một bản thiết kế, đó là \textit{mô hình} và \textit{hệ tiên đề}. Mô hình là một đối tượng, hay cấu trúc toán học thỏa mãn một hệ tiên đề nào đó. Để minh họa, chúng tôi đưa ra một mô hình bằng lý thuyết tập hợp (được đề xuất bởi nhà toán học John von Neumann) cho hệ tiên đề Peano về số tự nhiên như sau:
\begin{itemize}
    \item $0$ là tập hợp rỗng $\varnothing$.
    \item $S$ là một phép toán trên tập hợp: $S(A) = A \cup \{ A \}$.
    \item Các số tự nhiên tương ứng với các tập hợp sau (dưới đây chỉ liệt kê bốn số tự nhiên)
          \begin{align*}
              0 & = \varnothing,                                                                                                \\
              1 & = 0 \cup \{ 0 \} = \{ \varnothing \},                                                                         \\
              2 & = 1 \cup \{ 1 \} = \{ 0, 1 \}  = \{ \varnothing, \{ \varnothing \} \}                                         \\
              3 & = 2 \cup \{ 2 \} = \{ 0, 1, 2 \} = \{ \varnothing, \{ \varnothing \}, \{ \varnothing, \{ \varnothing \} \} \}
          \end{align*}
    \item Quan hệ bằng nhau giữa các số tự nhiên được nhìn nhận là quan hệ bằng nhau giữa các tập hợp.
\end{itemize}

Việc kiểm tra một mô hình có thỏa mãn một hệ tiên đề hay không chính là công việc chứng minh.

\subsection{So sánh hai mô hình: lát cắt Dedekind và dãy Cauchy hữu tỉ}

Chỉ còn lại câu hỏi là làm sao để xây dựng tập hợp số thực. Hiện nay, hai cách xây dựng tập hợp số thực thường được sử dụng nhất là \textit{lát cắt Dedekind} và \textit{dãy Cauchy hữu tỉ}, với cơ sở là tập hợp số hữu tỉ. Trong chương này, chúng tôi giải thích ý tưởng và nêu chi tiết về cả hai cách xây dựng. Cách xây dựng được trình bày trong chương này mang nhiều chi tiết kĩ thuật, đặc biệt là với dãy Cauchy hữu tỉ. Chúng tôi khuyên bạn đọc theo dõi những kết quả chính trước (được đánh dấu là Định lý), chú ý tới thứ tự, bình luận, và không cần nắm bắt chứng minh của tất cả trong lần đọc đầu tiên.

\section{Lát cắt Dedekind}

Trong mục này, không gian mà chúng ta làm việc (định nghĩa và chứng minh các kết quả liên quan đến lát cắt Dedekind) là tập hợp số hữu tỉ.

\subsection{Định nghĩa lát cắt Dedekind}

Trong phần mở đầu của chương này, chúng tôi có đưa ra một định nghĩa trực giác về tập hợp số thực. Theo định nghĩa trực giác đó, tập hợp số thực là một đường thẳng kéo dài vô tận về hai phía, còn mỗi số thực tương ứng với một điểm được đánh dấu trên đường thẳng đó. Định nghĩa trực giác này có thể được xem như khởi nguồn cho định nghĩa lát cắt Dedekind sau đây.

\begin{definition}[Lát cắt Dedekind]
    Một lát cắt Dedekind\index{Lát cắt Dedekind} trong một tập hợp $S$ được sắp thứ tự toàn phần là một phân hoạch gồm hai tập hợp $A, B$ sao cho
    \begin{enumerate}[label={(DC\arabic*)}]
        \item $A$ khác rỗng và $A$ không phải toàn bộ tập hợp $S$.
        \item Mọi phần tử của $A$ nhỏ hơn mọi phần tử của $B$.
        \item $A$ không có phần tử lớn nhất.
        \item Nếu $x$ thuộc $A$ thì bất cứ phần tử nào nhỏ hơn $x$ và thuộc $S$ cũng thuộc $A$ (đặc điểm này còn được phát biểu là $A$ đóng dưới).
    \end{enumerate}

    Một lát cắt như vậy được kí hiệu là $(A, B)$, hoặc chỉ là $A$, bởi vì $B = S\setminus A$ ($B$ hoàn toàn được xác định khi biết $A$).

    Chúng ta kí hiệu tập hợp các lát cắt Dedekind trong tập hợp $S$ là $\mathscr{D}_{S}$.
\end{definition}

Để xây dựng tập hợp số thực bằng lát cắt Dedekind, chúng ta sử dụng các lát cắt trong tập hợp số hữu tỉ. Tập hợp các lát cắt Dedekind trong tập hợp số hữu tỉ được kí hiệu là $\mathscr{D}_{\mathbb{Q}}$. Chúng ta cũng dùng cách gọi vắn tắt là lát cắt để chỉ lát cắt Dedekind trên tập hợp số hữu tỉ, trừ khi ngữ cảnh phát biểu khác đi.

Để hiểu rõ hơn định nghĩa lát cắt, chúng ta theo dõi các ví dụ và phản ví dụ sau.
\begin{example}
    Tập hợp
    \[
        A = \{ x \mid x\in\mathbb{Q} \wedge x < q \}
    \]

    trong đó $q$ là một số hữu tỉ, là một lát cắt. Chúng ta kiểm tra điều này qua từng điều trong định nghĩa lát cắt.
    \begin{enumerate}[label={(DC\arabic*)}]
        \item $A$ khác rỗng vì $q - 1$ là một phần tử của $A$. Bên cạnh đó, $A$ cũng không phải toàn bộ tập hợp số hữu tỉ vì $q$ không phải một phần tử của $A$.
        \item $B = \mathbb{Q} - A = \{ x \mid x\in\mathbb{Q} \wedge q\leq x \}$. Mọi phần tử của $A$ nhỏ hơn mọi phần tử của $B$ theo tính chất bắc cầu của quan hệ $\leq$ trên tập hợp số hữu tỉ.
        \item $A$ không có phần tử lớn nhất vì với mỗi phần tử $x$ của $A$, chúng ta luôn tìm được một phần tử khác lớn hơn, chẳng hạn như $\frac{q + x}{2}$.
        \item Nếu một số hữu tỉ $x$ thuộc $A$ thì mọi số hữu tỉ nhỏ hơn $x$ cũng thuộc $A$. Điều này được suy ra từ tính chất bắc cầu của quan hệ $\leq$ trên tập hợp số hữu tỉ.
    \end{enumerate}
\end{example}

\begin{example}
    Tập hợp
    \[
        A = \{ x \mid \text{$x$ là số hữu tỉ thỏa mãn $x < 0$ hoặc $x^{2} < 2$} \}
    \]

    là một lát cắt. Điều này được suy ra từ tính chất bắc cầu của quan hệ $\leq$ trên tập hợp số hữu tỉ và Mệnh đề~\ref{proposition:irrational-cut}.
\end{example}

\begin{counterexample}
    Tập hợp
    \[
        A = \{ x \mid \text{$x$ là số hữu tỉ thỏa mãn $x > 0$ và $x^{2} < 2$} \}
    \]

    \textbf{không phải} một lát cắt. Bởi vì $1$ là phần tử của $A$ nhưng $0 < 1$ lại không phải một phần tử của $A$.
\end{counterexample}

\begin{counterexample}
    Tập hợp
    \[
        A = \left\{ \frac{-1}{n} \mid \text{$n$ là một số nguyên dương} \right\}
    \]

    \textbf{không phải} một lát cắt. Bởi vì $-1$ là phần tử của $A$ nhưng $-2 < -1$ lại không phải một phần tử của $A$.
\end{counterexample}

\noindent Trước khi tiếp tục xây dựng tập hợp số thực bằng lát cắt, chúng ta lưu ý kết quả sau về lát cắt nói chung.
\begin{proposition}\label{proposition:upper-bound-of-dedekind-cut}
    Tập hợp $A$ là tập hợp con của một tập hợp $S$ được sắp thứ tự toàn phần sao cho $A$ khác rỗng và $A$ không phải toàn bộ tập hợp $S$. Chứng minh rằng $A$ là một lát cắt khi và chỉ khi $S\setminus A$ chỉ chứa tất cả các cận trên của $A$.
\end{proposition}

\begin{proof}
    ($\Rightarrow$) $A$ là một lát cắt.

    Theo định nghĩa lát cắt, mọi phần tử của $A$ nhỏ hơn mọi phần tử của $S\setminus A$. Do đó mọi phần tử của $S\setminus A$ là các cận trên của $A$.

    Giả sử phản chứng rằng nếu phần tử $x$ của $S$ là một cận trên của $A$ thì $x$ thuộc $A$. Theo giả sử phản chứng, $x$ là phần tử lớn nhất của $A$, và điều này mâu thuẫn với định nghĩa lát cắt rằng $A$ không có phần tử lớn nhất. Do đó giả sử phản chứng là sai, kéo theo mọi cận trên của $A$ là phần tử của $S - A$.

    Do đó $S\setminus A$ chỉ chứa tất cả các cận trên của $A$.

    \bigskip

    ($\Leftarrow$) $S\setminus A$ chỉ chứa tất cả các cận trên của $A$.

    Giả sử phản chứng rằng $A$ có phần tử lớn nhất. Chúng ta kí hiệu phần tử lớn nhất của $A$ là $x$. Vì $x$ là một cận trên của $A$ nên $x$ cũng là một phần tử của $S\setminus A$. Điều này mâu thuẫn với định nghĩa hiệu của hai tập hợp. Do đó giả sử phản chứng là sai, kéo theo $A$ không có phẩn tử lớn nhất.

    Giả sử $a\in A$ và $x < a$. Vì $x < a$ nên $x$ không phải cận trên của $A$, kéo theo $x$ không phải phần tử của $S\setminus A$. Do đó $x$ là một phần tử của $A$.

    Theo định nghĩa lát cắt, chúng ta kết luận $A$ là một lát cắt.
\end{proof}

Trong các mục tiếp theo, chúng ta lần lượt định nghĩa quan hệ thứ tự giữa các lát cắt, phép toán cộng, nhân hai lát cắt và kiểm tra xem những cấu trúc đó có thỏa mãn hệ tiên đề về số thực hay không.

\subsection{Quan hệ thứ tự giữa các lát cắt}

Việc hình dung tập hợp số thực như một đường thẳng cho chúng ta một cách nhìn khá trực quan về quan hệ thứ tự (tương ứng với khái niệm bên trái, bên phải trong thực tế).

\begin{definition}\label{definition:order-relation-between-dedekind-cuts}
    $A$ và $B$ là hai lát cắt. Chúng ta nói lát cắt $A$ có quan hệ $\leq$ với lát cắt $B$ và kí hiệu là $A\leq B$ nếu và chỉ nếu $A\subseteq B$.
\end{definition}

\begin{theorem}
    Quan hệ $\leq$ trên tập hợp các lát cắt $\mathscr{D}_{\mathbb{Q}}$ ở Định nghĩa~\ref{definition:order-relation-between-dedekind-cuts} là một quan hệ thứ tự toàn phần.
\end{theorem}

\begin{proof}
    Với mỗi lát cắt $A$, chúng ta luôn có $A\subseteq A$, kéo theo $A\leq A$. Do đó quan hệ $\leq$ trên tập hợp $\mathscr{D}_{\mathbb{Q}}$ có tính chất phản xạ.

    Với mỗi lát cắt $A, B, C$, nếu $A\leq B$ và $B\leq C$ thì $A\subseteq B$ và $B\subseteq C$. Vì quan hệ bao hàm giữa các tập hợp có tính chất bắc cầu nên $A\subseteq C$, kéo theo $A\leq C$. Do đó quan hệ $\leq$ trên tập hợp $\mathscr{D}_{\mathbb{Q}}$ có tính chất bắc cầu.

    Với mỗi lát cắt $A, B$, nếu $A\leq B$ và $B\leq A$ thì $A\subseteq B$ và $B\subseteq A$, kéo theo $A = B$. Do đó quan hệ $\leq$ trên tập hợp $\mathscr{D}_{\mathbb{Q}}$ có tính chất phản đối xứng.

    Như vậy quan hệ $\leq$ trong tập hợp các lát cắt $\mathscr{D}_{\mathbb{Q}}$ là một quan hệ thứ tự.

    \bigskip

    Chúng ta chọn hai lát cắt $A, B$ bất kì. Nếu $A = B$ thì $A\leq B$ và $B\leq A$. Nếu $A\ne B$, chúng ta xét hai trường hợp sau.
    \begin{enumerate}[label={\textbf{Trường hợp \arabic*.}},itemindent=2cm]
        \item Mọi phần tử của $A$ đều thuộc $B$.

              Điều này đồng nghĩa với $A\subset B$, kéo theo $A\leq B$.
        \item Tồn tại một phần tử của $A$ nhưng không thuộc $B$.

              Giả sử phần tử $a$ của $A$ không thuộc $B$. Theo Mệnh đề~\ref{proposition:upper-bound-of-dedekind-cut}, $a$ là một cận trên của $B$. Mà $B$ không có phần tử lớn nhất, nên chúng ta suy ra mọi phần tử của $B$ đều nhỏ hơn $a$. Vì $A$ là một lát cắt nên mọi số hữu tỉ nhỏ hơn $a$ đều thuộc $A$. Kết hợp hai điều vừa thu được, chúng ta suy ra mọi phần tử của $B$ đều là phần tử của $A$. Do đó $B\subset A$, kéo theo $B\leq A$.
    \end{enumerate}

    Do đó với hai lát cắt $A, B$ bất kì, $A\leq B$ hoặc $B\leq A$. Vậy quan hệ $\leq$ trên tập hợp các lát cắt $\mathscr{D}_{\mathbb{Q}}$ là một quan hệ thứ tự toàn phần.
\end{proof}

Chúng ta đặc biệt lưu ý lát cắt sau.
\[
    \begin{split}
        O = \{ x \mid x\in\mathbb{Q} \wedge x < 0 \}.
    \end{split}
\]

Với cơ sở là quan hệ thứ tự toàn phần trên tập hợp các lát cắt $\mathscr{D}_{\mathbb{Q}}$, chúng ta đưa ra định nghĩa sau.
\begin{definition}
    \begin{enumerate}[label={(\roman*)}]
        \item Một lát cắt $A$ được gọi là lát cắt dương nếu và chỉ nếu $O < A$.
        \item Một lát cắt $A$ được gọi là lát cắt âm nếu và chỉ nếu $A < O$.
    \end{enumerate}
\end{definition}

Như vậy, một lát cắt $A$ là không âm nếu và chỉ nếu $O\leq A$, là không dương nếu $A\leq O$.

\begin{theorem}
    Nếu lát cắt $A$ là một lát cắt dương thì tồn tại một phần tử $a$ của $A$ sao cho $a > 0$.
\end{theorem}

\begin{proof}
    Giả sử phản chứng rằng lát cắt dương $A$ không có số hữu tỉ dương nào. Như vậy mọi phần tử $x$ của $A$ đều thỏa mãn $x\leq 0$. Do đó, $A\subseteq O$. Theo định nghĩa quan hệ $\leq$ trên tập hợp $\mathscr{D}_{\mathbb{Q}}$, chúng ta suy ra $A\leq O$. Điều này mâu thuẫn với giả thiết $A$ là một lát cắt dương ($A > O$).

    Vậy nếu lát cắt $A$ là một lát cắt dương thì tồn tại một phần tử $a$ của $A$ sao cho $a > 0$.
\end{proof}

\subsection{Phép cộng lát cắt}

\begin{theorem}[Phép toán cộng lát cắt]
    Cho hai lát cắt $A$ và $B$. Khi đó tập hợp sau
    \[
        A + B = \{ a + b \mid a\in A\wedge b\in B \}
    \]

    là một lát cắt.
\end{theorem}

\begin{proof}
    Chúng ta kiểm tra từng điều kiện của một lát cắt.
    \begin{enumerate}[label={(DC\arabic*)}]
        \item Vì $A$ và $B$ khác rỗng nên tồn tại hai số hữu tỉ $a$ và $b$ lần lượt thuộc $A$ và $B$. Theo định nghĩa của tập hợp $A + B$ thì $a + b$ là một phần tử của $A + B$. Do đó tập hợp $A + B$ khác rỗng.
        \item Chúng ta chọn $c$ là một cận trên của $A$, và $d$ là một cận trên của $B$. Với mọi phần tử $a$ thuộc $A$ và $b$ thuộc $B$, chúng ta có $a + b\leq c + b \leq c + d$. Do đó $A + B$ không phải toàn bộ tập hợp số hữu tỉ.
        \item Chúng ta chọn $a + b$ là một phần tử bất kì của tập hợp $A + B$, trong đó $a$ thuộc $A$ và $b$ thuộc $B$. Vì $A$ là một lát cắt nên $A$ không có phần tử lớn nhất, do đó tồn tại phần tử $a'$ của $A$ sao cho $a < a'$. Từ việc $a < a'$, chúng ta suy ra $a + b < a' + b$. Do đó trong tập hợp $A + B$, với mỗi phần tử, luôn tồn tại phần tử lớn hơn, kéo theo $A + B$ không có phần tử lớn nhất.
        \item Chúng ta chọn $a + b$ là một phần tử bất kì của tập hợp $A + B$, và $x$ là một số hữu tỉ nhỏ hơn $a + b$. Vì $x < a + b$ nên $x + (-b) < a$. Theo định nghĩa lát cắt, vì $x + (-b) < a$ nên $x + (-b)$ thuộc $A$. Theo định nghĩa của tập hợp $A + B$, $x = (x + (-b)) + b$ là một phần tử của $A + B$. Do đó $A + B$ đóng dưới.
    \end{enumerate}

    Vậy $A + B$ là một lát cắt.
\end{proof}

Sau đây, chúng ta kiểm tra bốn tiên đề đầu tiên của các tiên đề về trường. Chúng tôi gặp khó khăn với việc kiểm tra tiên đề thứ ba trong các tiên đề về trường. Để giải quyết khó khăn đó, chúng tôi sử dụng một tính chất sâu hơn của số hữu tỉ. Định nghĩa trong mệnh đề sau còn được sử dụng ở những phần sau của tài liệu.

\begin{proposition}[Phần nguyên của số hữu tỉ]\label{proposition:integral-part-of-rational-numbers}
    Với mỗi số hữu tỉ $x$, tồn tại duy nhất số nguyên $k$ sao cho $k\leq x < k+1$. (Số nguyên $k$ như vậy được gọi là \textbf{phần nguyên\index{Phần nguyên}} của số hữu tỉ $x$, và được kí hiệu là $\floor{x}$.)
\end{proposition}

\begin{proof}
    Chúng ta kí hiệu $\frac{a}{b}$ là phân số tối giản của $x$ (lưu ý rằng $b$ là một số nguyên dương). Theo thuật toán chia Euclid, tồn tại duy nhất số nguyên $k$ và số nguyên $r$ sao cho $a = kb + r$ và $0\leq r < b$. Do đó
    \[
        k = \frac{kb}{b} \leq \frac{kb + r}{b} = \frac{a}{b} < \frac{(k+1)b}{b} = k+1
    \]

    Giả sử số nguyên $\ell$ thỏa mãn $\ell\leq x < \ell + 1$.

    Giả sử phản chứng rằng $\ell < k$. Khi đó $\ell\leq k - 1$. Cùng với $x < \ell + 1$, chúng ta suy ra $x < \ell + 1\leq (k-1) + 1 = k$. Điều này mâu thuẫn với $k\leq x$.

    Giả sử phản chứng rằng $\ell > k$. Khi đó $k\leq \ell - 1$. Cùng với $x < k + 1$, chúng ta suy ra $x < k + 1\leq (\ell - 1) + 1 = \ell$. Điều này mâu thuẫn với $\ell\leq x$.

    Do đó, $k = \ell$. Như vậy, với mỗi số hữu tỉ $x$, tồn tại duy nhất số tự nhiên $k$ sao cho $k\leq x < k+1$.
\end{proof}

Chứng minh các đẳng thức về lát cắt đồng nghĩa với việc chứng minh hai tập hợp bằng nhau. Nhắc lại, để chứng minh hai tập hợp bằng nhau, chúng ta cần chỉ ra tập hợp này là bộ phận của tập hợp kia và ngược lại, hoặc chỉ ra định nghĩa của hai tập hợp đó là tương đương.

\begin{theorem}[Các tính chất của phép cộng lát cắt]\label{theorem:properties-of-dedekind-cuts-addition}
    Trong tập hợp các lát cắt $\mathscr{D}_{\mathbb{Q}}$
    \begin{enumerate}[label={(\roman*)}]
        \item Phép toán cộng lát cắt có tính chất kết hợp. Nói cách khác, với mọi lát cắt $A, B, C$, chúng ta có
              \[
                  (A + B) + C = A + (B + C).
              \]
        \item Phép toán cộng lát cắt có phần tử đồng nhất. Nói cách khác, tồn tại lát cắt $O$ sao cho với mọi lát cắt $A$, chúng ta có
              \[
                  A + O = O + A = A.
              \]
        \item Mỗi lát cắt có một lát cắt đối. Nói cách khác, với mỗi lát cắt $A$, tồn tại lát cắt $A'$ sao cho
              \[
                  A + A' = A' + A = O.
              \]
        \item Phép cộng lát cắt có tính chất giao hoán. Nói cách khác, với mọi lát cắt $A, B$, chúng ta có
              \[
                  A + B = B + A.
              \]
    \end{enumerate}
\end{theorem}

\begin{proof}
    \begin{enumerate}[label={(\roman*)}]
        \item Theo định nghĩa phép cộng lát cắt và tính chất kết hợp của phép cộng số hữu tỉ
              \begin{align*}
                  (A + B) + C & = \{ x + c \mid x\in A + B \wedge c\in C \}                  \\
                              & = \{ (a + b) + c \mid (a\in A\wedge b\in B)\wedge c\in C \}  \\
                              & = \{ a + (b + c) \mid a\in A \wedge (b\in B\wedge c\in C) \} \\
                              & = \{ a + y \mid a\in A \wedge y\in B + C \}                  \\
                              & = A + (B + C).
              \end{align*}

              Do đó phép cộng lát cắt có tính chất kết hợp.
        \item Chúng ta định nghĩa lát cắt $O$ là tập hợp các số hữu tỉ nhỏ hơn $0$.

              Với mỗi phần tử $a + x$ của lát cắt $A + O$ ($a$ thuộc $A$ và $x$ thuộc $O$), chúng ta có $a + x < a$. Do đó $A + O \subseteq A$. Mặt khác, trong lát cắt $A$, với mỗi phần tử $a$, tồn tại một phần tử $a'$ sao cho $a < a'$. Chúng ta có $a = a' + ((-a') + a)$. $a'$ là một phần tử của $A$ và $(-a') + a$ là một phần tử của $A + O$. Do đó $A \subseteq A + O$. Như vậy, $A + O = A$.

              Hoàn toàn tương tự, với mỗi phần tử $x + a$ của lát cắt $O + A$ ($x$ thuộc $O$ và $a$ thuộc $A$), chúng ta có $x + a < a$. Do đó $O + A \subseteq A$. Mặt khác, trong lát cắt $A$, với mỗi phần tử $a$, tồn tại một phần tử $a'$ sao cho $a < a'$. Chúng ta có $a = (a + (-a')) + a'$. $a'$ là một phần tử của $A$ và $a + (-a')$ là một phần tử của $O + A$. Do đó $A \subseteq O + A$. Như vậy, $O + A = A$.
        \item Chúng ta định nghĩa tập hợp $A'$ như sau
              \[
                  A' = \{ x - a' \mid x < 0 \wedge a'\in \mathbb{Q} - A \}
              \]

              Trước tiên, chúng ta chứng minh rằng $A'$ là một lát cắt.
              \begin{enumerate}[label={(DC\arabic*)}]
                  \item Vì $A$ là một lát cắt nên $\mathbb{Q} - A$ khác rỗng. Chọn $a'$ thuộc $\mathbb{Q} - A$ và chọn $x$ là một số hữu tỉ nhỏ hơn $0$. Theo định nghĩa của tập hợp $A'$, $x - a'$ thuộc $A'$. Do đó $A'$ khác rỗng.
                  \item Giả sử $x - a'$ là một phần tử của $A'$ ($x < 0$ và $a'$ thuộc $\mathbb{Q} - A$). Chọn $a$ là một phần tử của $A$. Chúng ta có $x < 0$ và $a < a'$. Điều này kéo theo $-a' < -a$ và $x - a' < 0 - a = -a$. Như vậy $-a$ là một cận trên của $A'$. Do đó $A'$ không phải toàn bộ tập hợp số hữu tỉ.
                  \item Giả sử $x - a'$ là một phần tử của $A'$ ($x < 0$ và $a'$ thuộc $\mathbb{Q} - A$). Vì $O$ là một lát cắt nên tồn tại một số hữu tỉ $y$ nhỏ hơn $0$ và lớn hơn $x$. Vì $x - a'$ thuộc $A'$ và $x < y$ nên $x - a' < y - a'$. Phần tử $y - a'$ của $A'$ lớn hơn $x - a'$. Điều này có nghĩa là trong tập hợp $A'$, với mỗi phần tử, luôn tồn tại phần tử lớn hơn. Do đó $A'$ không có phần tử lớn nhất.
                  \item Giả sử $x - a'$ là một phần tử của $A'$ ($x < 0$ và $a'$ thuộc $\mathbb{Q} - A$) và số hữu tỉ $y$ thỏa mãn $y < x - a'$. Khi đó $(a' - x) + y < 0$ và
                        \[
                            y = ((x - a') + (a' - x)) + y = (x + ((a' - x) + y)) - a'
                        \]

                        Vì $x + ((a' - x) + y) < x + 0 < 0$ và $a'$ thuộc $\mathbb{Q} - A$ nên theo định nghĩa của tập hợp $A'$, chúng ta suy ra $y$ thuộc $A'$. Do đó $A'$ đóng dưới.
              \end{enumerate}

              Như vậy $A'$ là một lát cắt. Tiếp theo, chúng ta chứng minh $A + A' = A' + A = O$.

              Giả sử $a + (x - a')$ là một phần tử của $A + A'$ (trong đó $a$ thuộc $A$, $x < 0$, và $a'$ thuộc $\mathbb{Q} - A$). Vì $a - a' < 0$ và $x < 0$ nên $a + (x - a') = x + (a - a') < x < 0$. Do đó $A + A' \subseteq O$.

              Giả sử $(x - a') + a$ là một phần tử của $A + A'$ (trong đó $a$ thuộc $A$, $x < 0$, và $a'$ thuộc $\mathbb{Q} - A$). Vì $a - a' < 0$ và $x < 0$ nên $(x - a') + a = x + (a - a') < x < 0$. Do đó $A' + A \subseteq O$.

              Để chứng minh $O\subseteq A + A'$ và $O\subseteq A' + A$, chúng ta sẽ chỉ ra sự tồn tại của hai phần tử lần lượt thuộc $A$ và $A'$ sao cho tổng của chúng nhỏ hơn $0$.

              Giả sử $x$ là một số hữu tỉ nhỏ hơn $0$ và $a$ là một phần tử của $A$. Chúng ta xét tập hợp $S$ gồm tất cả các số nguyên $n$ sao cho $nx$ thuộc $A$.
              \[
                  S = \{ n \mid n\in\mathbb{Z} \wedge nx\in A \}
              \]

              Theo Mệnh đề~\ref{proposition:integral-part-of-rational-numbers}, với mỗi số hữu tỉ $q$, tồn tại số nguyên $k$ sao cho $k\leq \frac{q}{x} < k + 1$. Vì $x < 0$ nên
              \[
                  \begin{split}
                      (k + 1)x < \frac{q}{x}\cdot x = q, \\
                      kx \geq \frac{q}{x}\cdot x = q.
                  \end{split}
              \]

              Chọn $q$ là một phần tử của $A$ thì $(k + 1)x < q$ cho thấy tập hợp $S$ khác rỗng. Chọn $q$ là một cận trên của $A$ thì việc $kx\geq q$ cho thấy tập hợp $S$ không phải toàn bộ tập hợp số nguyên, tức là tồn tại số nguyên $n_{0}$ nào đó sao cho $n_{0}x$ không thuộc $A$.

              Bên cạnh đó, bằng định nghĩa của lát cắt, cùng tính chất phân phối của phép nhân với phép cộng số hữu tỉ và tính tương thích của phép cộng với quan hệ $\leq$ trên tập hợp số hữu tỉ, chúng ta rút ra hai nhận xét: (1) Nếu $n$ thuộc $S$ thì $(n + 1)$ cũng thuộc $S$; (2) Nếu $n$ không thuộc $S$ thì $(n - 1)$ cũng không thuộc $S$.

              Với những điều trên, chúng ta suy ra rằng với mọi $n$ thuộc $A$, chúng ta có $n\geq n_{0}$. Như vậy tập hợp $S$ thỏa mãn giả thiết của nguyên lý thứ tự tốt. Theo nguyên lý thứ tự tốt, $S$ có phần tử nhỏ nhất. Chúng ta kí hiệu phần tử nhỏ nhất của $S$ là $m$. Vì số nguyên $m$ là phần tử nhỏ nhất của $S$ nên $m - 1$ không thuộc $S$. Nói cách khác, $mx$ thuộc $A$ và $(m - 1)x$ không thuộc $A$. Do $A$ là một lát cắt, nên tồn tại phần tử $a$ của $A$ sao cho $mx < a$. Khi đó chúng ta có $mx - a < 0$. Ngoài ra
              \begin{align*}
                  x & = mx - (m-1)x = (a + (mx - a)) - (m-1)x \\
                    & = a + ((mx-a) - (m-1)x)                 \\
                    & = ((mx - a) - (m-1)x) + a
              \end{align*}

              Vì $a$ thuộc $A$ và $(mx - a) - (m-1)x$ thuộc $A'$ nên theo định nghĩa phép cộng lát cắt, chúng ta có $x$ thuộc $A + A'$ và $A' + A$. Vì chúng ta đang xét số hữu tỉ $x$ bất kì nhỏ hơn $0$, nên chúng ta suy ra $O \subseteq A + A'$ và $O \subseteq A' + A$.

              Như vậy, $O = A + A'$ và $O = A' + A$.
        \item Theo định nghĩa phép cộng lát cắt và tính chất giao hoán của phép cộng số hữu tỉ
              \begin{align*}
                  A + B & = \{ a + b \mid a\in A \wedge b\in B \} \\
                        & = \{ b + a \mid b\in B \wedge a\in A \} \\
                        & = B + A.
              \end{align*}

              Do đó phép cộng lát cắt có tính chất giao hoán.
    \end{enumerate}
\end{proof}

Với định lý sau đây, chúng ta khẳng định được tính duy nhất của phần tử đồng nhất của phép cộng lát cắt và lát cắt đối.
\begin{theorem}
    \begin{enumerate}[label={(\roman*)}]
        \item Phép cộng lát cắt có đúng một phần tử đồng nhất.
        \item Với mỗi lát cắt $A$, tồn tại duy nhất một lát cắt là lát cắt đối của $A$.
    \end{enumerate}
\end{theorem}

\begin{proof}
    \begin{enumerate}[label={(\roman*)}]
        \item Định lý~\ref{theorem:properties-of-dedekind-cuts-addition} đã chỉ ra sự tồn tại của phần tử đồng nhất của phép cộng lát cắt, phần tử đó là lát cắt $O$.

              Giả sử lát cắt $O'$ thỏa mãn $A + O' = O' + A = A$ với mọi lát cắt $A$. Khi đó, theo phần (i), (ii), (iii) của Định lý~\ref{theorem:properties-of-dedekind-cuts-addition} và định nghĩa của $O'$, chúng ta có $O' = O' + O = O$.

              Vậy phép cộng lát cắt có đúng một phần tử đồng nhất.
        \item Định lý~\ref{theorem:properties-of-dedekind-cuts-addition} đã chỉ ra rằng với mỗi lát cắt $A$, tồn tại một lát cắt $A'$ sao cho $A + A' = A' + A = O$.

              Giả sử lát cắt $A''$ thỏa mãn $A + A'' = A'' + A = O$. Khi đó, theo phần (i), (ii), (iii) của Định lý~\ref{theorem:properties-of-dedekind-cuts-addition} và định nghĩa của $A''$, chúng ta có
              \begin{align*}
                  A'' & = A'' + O        \\
                      & = A'' + (A + A') \\
                      & = (A'' + A) + A' \\
                      & = O + A'         \\
                      & = A'
              \end{align*}

              Vậy, với mỗi lát cắt $A$, tồn tại duy nhất một lát cắt là lát cắt đối của $A$.
    \end{enumerate}
\end{proof}

Với mỗi lát cắt $A$, chúng ta kí hiệu bởi $-A$ lát cắt sau
\[
    -A = A' = \{ x - a' \mid x < 0 \wedge a'\in \mathbb{Q} - A \}.
\]

\begin{theorem}\label{theorem:additive-inversion-is-involutive}
    Với mọi lát cắt $A$, chúng ta có $A = -(-A)$.
\end{theorem}

\begin{proof}
    Theo định nghĩa của $-A$ và phần (i), (ii), (iii) của Định lý~\ref{theorem:properties-of-dedekind-cuts-addition}
    \begin{align*}
        (-(-A)) & = (-(-A)) + O          \\
                & = (-(-A)) + ((-A) + A) \\
                & = ((-(-A)) + (-A)) + A \\
                & = O + A                \\
                & = A.
    \end{align*}

    Vậy, với mọi lát cắt $A$, chúng ta có $A = -(-A)$.
\end{proof}

Tiếp theo, chúng ta kiểm tra tính tương thích của phép cộng lát cắt với quan hệ thứ tự $\leq$ trên tập hợp $\mathscr{D}_{\mathbb{Q}}$.

\begin{theorem}
    Với mọi lát cắt $A, B$, nếu $A\leq B$ thì với mọi lát cắt $C$, chúng ta có $A + C\leq B + C$.
\end{theorem}

\begin{proof}
    Giả sử hai lát cắt $A, B$ thỏa mãn $A\leq B$. Theo định nghĩa quan hệ thứ tự $\leq$ trên tập hợp $\mathscr{D}_{\mathbb{Q}}$, chúng ta có $A\subseteq B$.

    Giả sử $a + c$ là một phần tử bất kì của $A + C$ (trong đó, $a$ thuộc $A$ và $c$ thuộc $C$). Vì $A\subseteq B$ nên $a$ thuộc $B$. Theo định nghĩa phép cộng lát cắt, chúng ta suy ra $a + c$ thuộc $B + C$. Do đó $A + C \subseteq B + C$.

    Vậy với mọi lát cắt $A, B$, nếu $A\leq B$ thì với mọi lát cắt $C$, chúng ta có $A + C\leq B + C$.
\end{proof}

Ngược lại, nếu các lát cắt $A, B, C$ thỏa mãn $A + C\leq B + C$ thì chúng ta cũng có $A\leq B$. Tuy nhiên đây là một hệ quả trực tiếp của định lý trên, bởi vì $A + C\leq B + C$ kéo theo $(A + C) + (-C) \leq (B + C) + (-C)$.

Kết hợp định lý trên với phương pháp chứng minh bằng phản chứng, chúng ta thu được các hệ quả sau.

\begin{corollary}
    \begin{enumerate}[label={(\roman*)}]
        \item Với mọi lát cắt $A, B$, nếu $A < B$ thì với mọi lát cắt $C$, chúng ta có $A + C < B + C$.
        \item Nếu $A$ và $B$ là các lát cắt không âm thì $A + B$ là một lát cắt không âm.
        \item Nếu $A$ và $B$ là các lát cắt dương thì $A + B$ là một lát cắt dương.
        \item Nếu $A$ và $B$ là các lát cắt âm thì $A + B$ là một lát cắt âm.
        \item Lát cắt $A$ là lát cắt dương khi và chỉ khi lát cắt $-A$ là lát cắt âm.
    \end{enumerate}
\end{corollary}

\begin{proposition}\label{proposition:nonnegative-elements-of-dedekind-cuts-addition}
    Với mọi lát cắt $A, B$, nếu $A > O$ và $B > O$ thì với mỗi số hữu tỉ $x$ không âm (nếu có) trong $A + B$, tồn tại số hữu tỉ $a$ không âm trong $A$ và số hữu tỉ $b$ không âm trong $B$ sao cho $x = a + b$.
\end{proposition}

\begin{proof}
    Vì phép cộng lát cắt có tính chất giao hoán nên không mất tính tổng quát, chúng ta giả sử $A\leq B$.

    Vì $A > O$ và $B > O$ nên $0$ là một phần tử của $A, B$ và trong $A, B$ tồn tại số hữu tỉ dương.

    Nếu số hữu tỉ $x$ thuộc $A + B$ và $x = 0$ thì $x = 0 + 0$.

    Nếu số hữu tỉ $x$ thuộc $A + B$ và $x > 0$, chúng ta xét các trường hợp sau và chỉ ra rằng $x$ có thể được viết dưới dạng tổng của hai số hữu tỉ không âm lần lượt thuộc $A$ và $B$.
    \begin{enumerate}[label={\textbf{Trường hợp \arabic*.}},itemindent=1.5cm]
        \item $A = B$.

              Giả sử phản chứng rằng $\frac{x}{2}$ không thuộc $A$ (và $B$). Khi đó $\frac{x}{2}$ là một cận trên của $A$ và $B$, kéo theo $x = \frac{x}{2} + \frac{x}{2}$ là một cận trên của $A + B$, điều này mâu thuẫn với giả thiết $x$ thuộc $A + B$. Do đó $\frac{x}{2}$ thuộc $A$ và $B$.
        \item $A < B$.

              \textbf{Khả năng 1.} $x$ thuộc $A$.

              $x$ thuộc $A$ thì $x$ cũng thuộc $B$ vì $A\subset B$, $\frac{x}{2}$ thuộc $A$ và $B$. Chúng ta viết $x = \frac{x}{2} + \frac{x}{2}$.

              \textbf{Khả năng 2.} $x$ không thuộc $A$ và $x$ thuộc $B$.

              Chúng ta viết $x = 0 + x$.

              \textbf{Khả năng 3.} $x$ không thuộc $A$ và không thuộc $B$.

              Theo định nghĩa tổng hai lát cắt, tồn tại phần tử $a$ của $A$ và phần tử $b$ của $B$ sao cho $x = a + b$. Nếu $a$ là số hữu tỉ âm thì $x < b$, kéo theo $x$ thuộc $B$ (vì $B$ đóng dưới). Nếu $b$ là số hữu tỉ âm thì $x < a$, kéo theo $x$ thuộc $A$ (vì $A$ đóng dưới). Việc $x$ thuộc $A$ hay $x$ thuộc $B$ đều mâu thuẫn với giả thiết. Do đó $a$ và $b$ là hai số hữu tỉ không âm. Do đó $x = a + b$ viết dưới dạng tổng của hai số hữu tỉ không âm lần lượt thuộc $A$ và $B$.
    \end{enumerate}

    Vậy với mọi lát cắt $A, B$, nếu $A > O$ và $B > O$ thì với mỗi số hữu tỉ $x$ không âm trong $A + B$, tồn tại số hữu tỉ $a$ không âm trong $A$ và số hữu tỉ $b$ không âm trong $B$ sao cho $x = a + b$.
\end{proof}

\subsection{Phép nhân lát cắt}

Khi định nghĩa phép nhân lát cắt, chúng ta cũng gặp vấn đề như khi định nghĩa phép nhân hai số nguyên. Trong trường này này, chúng ta cũng giải quyết tương tự như định nghĩa phép nhân hai số nguyên.

\begin{theorem}[Phép nhân hai lát cắt]
    Cho hai lát cắt $A$ và $B$.
    \begin{enumerate}[label={(\roman*)}]
        \item Nếu $A\geq O$ và $B\geq O$ thì tập hợp sau
              \[
                  A\cdot B = \{ ab \mid a\in A\wedge b\in B\wedge a\geq 0\wedge b\geq 0 \} \cup O
              \]

              là một lát cắt.
        \item Nếu $A\geq 0$ và $B < O$ thì tập hợp $A\cdot B = -A\cdot (-B)$ là một lát cắt.
        \item Nếu $A < 0$ và $B\geq O$ thì tập hợp $A\cdot B = -(-A)\cdot B$ là một lát cắt.
        \item Nếu $A < 0$ và $B < O$ thì tập hợp $A\cdot B = (-A)\cdot (-B)$ là một lát cắt.
    \end{enumerate}
\end{theorem}

\begin{proof}
    \begin{enumerate}[label={(\roman*)}]
        \item Chúng ta kiểm tra các điều kiện trong định nghĩa lát cắt.

              Nếu ít nhất một trong hai lát cắt $A$ và $B$ bằng $O$ thì tập hợp $\{ ab \mid a\in A\wedge b\in B\wedge a\geq 0\wedge b\geq 0 \}$ là tập hợp rỗng, kéo theo $A\cdot B = O$, và là một lát cắt.

              Dưới đây, chúng ta xét trường hợp $A > O$ và $B > O$.
              \begin{enumerate}[label={(DC\arabic*)}]
                  \item Theo định nghĩa của $A\cdot B$, $A\cdot B$ không phải tập hợp rỗng, vì $A\cdot B \supseteq O$ và $O$ khác rỗng.
                  \item $A > O$ và $B > O$ thì tập hợp $\{ ab \mid a\in A\wedge b\in B\wedge a\geq 0\wedge b\geq 0 \}$ khác rỗng. Vì $A$ và $B$ là các lát cắt nên $A$ và $B$ có cận trên. Chọn $a_{0}, b_{0}$ lần lượt là cận trên của $A$ và $B$. Vì $A > O, B > 0$ và $A, B$ không có phần tử lớn nhất nên $a_{0} > 0$ và $b_{0} > 0$. Như vậy, nếu phần tử $x$ của $A\cdot B$ là một số hữu tỉ không vượt quá $0$ thì $x < a_{0}b_{0}$, còn nếu $x$ là một số hữu tỉ lớn hơn $0$ thì $x$ thuộc tập hợp $\{ ab \mid a\in A\wedge b\in B\wedge a\geq 0\wedge b\geq 0 \}$, kéo theo tồn tại $a$ thuộc $A$ và $b$ thuộc $B$ sao cho $x = ab$. Khi đó chúng ta có $ab\leq ab_{0} \leq a_{0}b_{0}$, do đó $A\cdot B$ có cận trên, tức là $A\cdot B$ không phải toàn bộ tập hợp số hữu tỉ.
                  \item Giả sử $x$ là một phần tử của $A\cdot B$. Vì $A > O$ và $B > O$ nên trong $A$ tồn tại phần tử $a$ sao cho $a > 0$ và trong $B$ tồn tại phần tử $b$ sao cho $b > 0$.

                        Nếu $x\leq 0$ thì $x\leq 0 < ab$.

                        Nếu $x > 0$ thì tồn tại các số hữu tỉ dương $a'$ thuộc $A$ và $b'$ thuộc $B$ sao cho $x = a'b'$. Vì $A$ và $B$ không có phần tử lớn nhất nên trong $A$ tồn tại phần tử $a''$ sao cho $a' < a''$ và trong $B$ tồn tại phần tử $b''$ sao cho $b' < b''$. Khi đó chúng ta có $x = a'b' < a'b'' < a''b''$.

                        Như vậy, trong tập hợp $A\cdot B$, với mỗi phần tử $x$, tồn tại phần tử lớn hơn. Do đó $A\cdot B$ không có phần tử lớn nhất.
                  \item Giả sử $x$ là một phần tử của $A\cdot B$, và $y$ là một số hữu tỉ sao cho $x > y$.

                        Nếu $y\leq 0$ thì $y$ thuộc $A\cdot B$, theo định nghĩa của $A\cdot B$, bởi $A\cdot B$ chứa mọi số hữu tỉ âm, và $\{ ab \mid a\in A\wedge b\in B\wedge a\geq 0\wedge b\geq 0 \}$ chứa số $0$ (vì $A > O$ và $B > O$).

                        Nếu $y > 0$ thì $x > y > 0$. Khi đó tồn tại số hữu tỉ dương $a$ thuộc $A$ và tồn tại số hữu tỉ dương $b$ thuộc $B$ sao cho $x = ab$. Khi đó
                        \[
                            y = x - (x - y) = ab - a\cdot\frac{x-y}{a} = a\cdot\left( b - \frac{x - y}{a} \right).
                        \]

                        Trong đó, $a$ thuộc $A$, $b - \frac{x - y}{a} > 0$ và thuộc $B$. Theo định nghĩa của tập hợp $A\cdot B$, chúng ta suy ra $y$ thuộc $A\cdot B$. Như vậy, tập hợp $A\cdot B$ đóng dưới.
              \end{enumerate}

              Vậy $A\cdot B$ là một lát cắt.
        \item Nếu $A\geq 0$ và $B < O$ thì $A\geq 0$ và $-B > O$. Theo phần (i), tập hợp $A\cdot (-B)$ là một lát cắt. Do đó $A\cdot B = -A\cdot (-B)$ là một lát cắt.
        \item Nếu $A < 0$ và $B\geq O$ thì $-A > O$ và $B\geq O$. Theo phần (i), tập hợp $(-A)\cdot B$ là một lát cắt. Do đó $A\cdot B = -(-A)\cdot B$ là một lát cắt.
        \item Nếu $A < 0$ và $B < O$ thì $-A > O$ và $-B > O$. Theo phần (i), tập hợp $(-A)\cdot (-B)$ là một lát cắt. Do đó $A\cdot B = (-A)\cdot (-B)$ là một lát cắt.
    \end{enumerate}
\end{proof}

Tính tương thích của phép nhân lát cắt với quan hệ thứ tự $\leq$ trên tập hợp $\mathscr{D}_{\mathbb{Q}}$ được rút ra trực tiếp từ định nghĩa.
\begin{theorem}
    Với mọi lát cắt $A, B$, nếu $A\geq O$ và $B\geq O$ thì $A\cdot B\geq O$.
\end{theorem}

\begin{proof}
    Theo định nghĩa của lát cắt $A\cdot B$, nếu $A\geq O$ và $B\geq O$ thì $O\subseteq A\cdot B$. Do đó, nếu $A\geq O$ và $B\geq O$ thì $A\cdot B\geq O$.
\end{proof}

Khi chứng minh các tính chất của phép nhân số nguyên, chúng ta xem xét từng trường hợp số nguyên âm, số nguyên không âm. Với các lát cắt, chúng ta cũng làm tương tự. Để làm gọn chứng minh cho các tính chất của phép nhân lát cắt, chúng ta sử dụng mệnh đề sau đây.
\begin{proposition}\label{proposition:dedekind-cuts-and-sign}
    Với mọi lát cắt $A, B$, chúng ta có
    \begin{enumerate}[label={(\roman*)}]
        \item $-A\cdot B = (-A)\cdot B = A\cdot (-B)$.
        \item $A\cdot B = (-A)\cdot (-B)$.
    \end{enumerate}
\end{proposition}

\begin{proof}
    \begin{enumerate}[label={(\roman*)}]
        \item Chúng ta xét đủ các trường hợp sau nhằm áp dụng định nghĩa tích hai lát cắt (có bốn trường hợp trong định nghĩa) và Định lý~\ref{theorem:additive-inversion-is-involutive}.
              \begin{enumerate}[label={\textbf{Trường hợp \arabic*.}},itemindent=1cm]
                  \item $A = B = O$.

                        Theo định nghĩa tích hai lát cắt, chúng ta có $A\cdot B = O\cdot O = O$.
                  \item $A = O$ và $B\ne O$.

                        Nếu $B > O$
                        \begin{align*}
                            -A\cdot B   & = -O = O                      & \text{(áp dụng định nghĩa tích lát cắt cho $A$ và $B$)}  \\
                            (-A)\cdot B & = O\cdot B = O                & \text{(áp dụng định nghĩa tích lát cắt cho $O$ và $B$)}  \\
                            A\cdot (-B) & = -A\cdot (-(-B)) = -A\cdot B & \text{(áp dụng định nghĩa tích lát cắt cho $A$ và $-B$)}
                        \end{align*}

                        Nếu $B < O$
                        \begin{align*}
                            -A\cdot B   & = A\cdot (-B) = O\cdot (-B) = O & \text{(áp dụng định nghĩa tích lát cắt cho $A$ và $B$, $O$ và $-B$)} \\
                            (-A)\cdot B & = O\cdot B = -O\cdot (-B) = O   & \text{(áp dụng định nghĩa tích lát cắt cho $O$ và $B$)}              \\
                        \end{align*}
                  \item $B = O$ và $A\ne O$.

                        Nếu $A > O$
                        \begin{align*}
                            -A\cdot B   & = -O = O                          & \text{(áp dụng định nghĩa tích lát cắt cho $A$ và $B$)}  \\
                            (-A)\cdot B & = -(-(-A))\cdot B = -A\cdot B = O & \text{(áp dụng định nghĩa tích lát cắt cho $-A$ và $B$)} \\
                            A\cdot (-B) & = A\cdot O = O                    & \text{(áp dụng định nghĩa tích lát cắt cho $A$ và $O$)}
                        \end{align*}

                        Nếu $A < O$
                        \begin{align*}
                            -A\cdot B   & = (-A)\cdot B = (-A)\cdot O = O & \text{(áp dụng định nghĩa tích lát cắt cho $A$ và $B$, $-A$ và $O$)} \\
                            A\cdot (-B) & = O\cdot (-B) = O               & \text{(áp dụng định nghĩa tích lát cắt cho $O$ và $-B$)}
                        \end{align*}

                  \item $A > O$ và $B > O$.
                        \begin{align*}
                            (-A)\cdot B & = -(-(-A))\cdot B = -A\cdot B & \text{(áp dụng định nghĩa tích lát cắt cho $-A$ và $B$)} \\
                            A\cdot (-B) & = -A\cdot (-(-B)) = -A\cdot B & \text{(áp dụng định nghĩa tích lát cắt cho $A$ và $-B$)}
                        \end{align*}
                  \item $A > O$ và $B < O$.
                        \begin{align*}
                            (-A)\cdot B & = (-(-A))\cdot (-B) = A\cdot (-B) & \text{(áp dụng định nghĩa tích lát cắt cho $-A$ và $B$)} \\
                            -A\cdot B   & = A\cdot (-B)                     & \text{(áp dụng định nghĩa tích lát cắt cho $A$ và $B$)}
                        \end{align*}
                  \item $A < O$ và $B > O$.
                        \begin{align*}
                            A\cdot (-B) & = -A\cdot (-(-B)) = -A\cdot B & \text{(áp dụng định nghĩa tích lát cắt cho $A$ và $-B$)} \\
                            -A\cdot B   & = (-A)\cdot B                 & \text{(áp dụng định nghĩa tích lát cắt cho $A$ và $B$)}
                        \end{align*}
                  \item $A < O$ và $B < O$.
                        \begin{align*}
                            A\cdot (-B) & = -(-A)\cdot (-B) & \text{(áp dụng định nghĩa tích lát cắt cho $A$ và $-B$)} \\
                            (-A)\cdot B & = -(-A)\cdot (-B) & \text{(áp dụng định nghĩa tích lát cắt cho $-A$ và $B$)} \\
                            -A\cdot B   & = -(-A)\cdot (-B) & \text{(áp dụng định nghĩa tích lát cắt cho $A$ và $B$)}
                        \end{align*}
              \end{enumerate}

              Như vậy, trong mọi trường hợp, chúng ta đều có $(-A)\cdot B = A\cdot (-B) = -A\cdot B$.
        \item Theo phần (i), chúng ta suy ra
              \[
                  (-A)\cdot (-B) = -(-(-A))\cdot (-B) = -A\cdot (-B) = A\cdot (-(-B)) = A\cdot B.
              \]
    \end{enumerate}
\end{proof}

Nếu $A$ và $B$ là các lát cắt không âm thì tập hợp con $\{ ab \mid a\in A\wedge b\in B\wedge a\geq 0\wedge b\geq 0 \}$ gồm tất cả các phần tử không âm của $A\cdot B$. Mệnh đề dưới đây khẳng định điều ngược lại: Với mỗi phần tử $x$ không âm (nếu có) của $A\cdot B$ (trong đó $A$ và $B$ là các lát cắt không âm), chúng ta có thể viết $x = ab$, trong đó $a, b$ lần lượt là các phần tử không âm nào đó của $A$ và $B$.

\begin{proposition}\label{proposition:nonnegative-elements-of-dedekind-cuts-multiplication}
    Với mọi lát cắt $A, B$, nếu $A\geq O$ và $B\geq O$ thì với mỗi số hữu tỉ $x$ không âm (nếu có) trong $A\cdot B$, tồn tại số hữu tỉ $a$ không âm trong $A$ và số hữu tỉ $b$ không âm trong $B$ sao cho $x = ab$.
\end{proposition}

\begin{proof}
    Nếu $A = O$ hoặc $B = O$ thì $A\cdot B = O$, kéo theo $A\cdot B$ không chứa số hữu tỉ không âm nào.

    Nếu $A > O$ và $B > O$ thì trong $A\cdot B$ tồn tại số hữu tỉ không âm. Khi đó tập hợp sau
    \[
        (A\cdot B) \setminus O = \{ ab \mid a\in A \wedge b\in B \wedge a\geq 0 \wedge b\geq 0 \}
    \]

    khác rỗng. Chúng ta xét hai trường hợp. Nếu phần tử $x$ của $A\cdot B$ là $0$ thì $x = 0\cdot b$, trong đó $0$ thuộc $A$ và $b$ là một số hữu tỉ không âm thuộc $B$. Nếu phần tử $x$ của $A\cdot B$ là một số hữu tỉ dương thì $x$ thuộc $(A\cdot B)\setminus O$, kéo theo tồn tại hai số hữu tỉ không âm $a$ và $b$ lần lượt thuộc $A$ và $B$ sao cho $x = ab$.

    Vậy với mọi lát cắt $A, B$, nếu $A\geq O$ và $B\geq O$ thì với mỗi số hữu tỉ $x$ không âm trong $A\cdot B$, tồn tại số hữu tỉ $a$ không âm trong $A$ và số hữu tỉ $b$ không âm trong $B$ sao cho $x = ab$.
\end{proof}

\begin{theorem}
    \begin{enumerate}[label={(\roman*)}]
        \item Phép nhân lát cắt có tính chất kết hợp. Nói cách khác, với mọi lát cắt $A, B, C$, chúng ta có
              \[
                  (A\cdot B)\cdot C = A\cdot (B\cdot C).
              \]
        \item Phép nhân lát cắt có tính chất giao hoán. Nói cách khác, với mọi lát cắt $A, B$, chúng ta có
              \[
                  A\cdot B = B\cdot A.
              \]
    \end{enumerate}
\end{theorem}

\begin{proof}
    \begin{enumerate}[label={(\roman*)}]
        \item Theo chứng minh của Mệnh đề~\ref{proposition:dedekind-cuts-and-sign}, nếu ít nhất một trong các lát cắt $A, B, C$ bằng $O$ thì $(A\cdot B)\cdot C = A\cdot (B\cdot C)$. Dưới đây, chúng ta xét tất cả các trường hợp mà $A\ne O, B\ne O, C\ne O$.
              \begin{enumerate}[label={\textbf{Trường hợp \arabic*.}},itemindent=1cm]
                  \item $A > O, B > O, C > O$.

                        Theo định nghĩa phép nhân lát cắt, Mệnh đề~\ref{proposition:nonnegative-elements-of-dedekind-cuts-multiplication}, và tính chất kết hợp của phép nhân số hữu tỉ, chúng ta có
                        \begin{align*}
                            (A\cdot B)\cdot C & = \{ xc \mid x\in A\cdot B\wedge c\in C\wedge x\geq 0\wedge c\geq 0 \} \cup O                        \\
                                              & = \{ (ab)c \mid a\in A\wedge b\in B\wedge c\in C\wedge a\geq 0\wedge b\geq 0\wedge c\geq 0 \} \cup O \\
                                              & = \{ a(bc) \mid a\in A\wedge b\in B\wedge c\in C\wedge a\geq 0\wedge b\geq 0\wedge c\geq 0 \} \cup O \\
                                              & = \{ ay \mid a\in A\wedge y\in B\cdot C\wedge a\geq 0\wedge y\geq 0 \} \cup O                        \\
                                              & = A\cdot (B\cdot C).
                        \end{align*}
                  \item $A > O, B > O, C < O$
                        \begin{align*}
                            (A\cdot B)\cdot C & = -(A\cdot B)\cdot (-C) & \text{(theo Mệnh đề~\ref{proposition:dedekind-cuts-and-sign})} \\
                                              & = -A\cdot (B\cdot (-C)) & \text{(theo \textbf{Trường hợp 1})}                            \\
                                              & = A\cdot (-B\cdot (-C)) & \text{(theo Mệnh đề~\ref{proposition:dedekind-cuts-and-sign})} \\
                                              & = A\cdot (B\cdot C)     & \text{(theo Mệnh đề~\ref{proposition:dedekind-cuts-and-sign})}
                        \end{align*}
                  \item $A > O, B < O, C > O$
                        \begin{align*}
                            (A\cdot B)\cdot C & = -(A\cdot (-B))\cdot C & \text{(theo Mệnh đề~\ref{proposition:dedekind-cuts-and-sign})} \\
                                              & = -A\cdot ((-B)\cdot C) & \text{(theo \textbf{Trường hợp 1})}                            \\
                                              & = A\cdot (-(-B)\cdot C) & \text{(theo Mệnh đề~\ref{proposition:dedekind-cuts-and-sign})} \\
                                              & = A\cdot (B\cdot C)     & \text{(theo Mệnh đề~\ref{proposition:dedekind-cuts-and-sign})}
                        \end{align*}
                  \item $A > O, B < O, C < O$
                        \begin{align*}
                            (A\cdot B)\cdot C & = (-(A\cdot B))\cdot (-C) & \text{(theo Mệnh đề~\ref{proposition:dedekind-cuts-and-sign})} \\
                                              & = (A\cdot (-B))\cdot (-C) & \text{(theo Mệnh đề~\ref{proposition:dedekind-cuts-and-sign})} \\
                                              & = A\cdot ((-B)\cdot (-C)) & \text{(theo \textbf{Trường hợp 1})}                            \\
                                              & = A\cdot (B\cdot C)       & \text{(theo Mệnh đề~\ref{proposition:dedekind-cuts-and-sign})}
                        \end{align*}
                  \item $A < O, B > O, C > O$
                        \begin{align*}
                            (A\cdot B)\cdot C & = (-(-A)\cdot B)\cdot C & \text{(theo Mệnh đề~\ref{proposition:dedekind-cuts-and-sign})} \\
                                              & = -((-A)\cdot B)\cdot C & \text{(theo Mệnh đề~\ref{proposition:dedekind-cuts-and-sign})} \\
                                              & = -(-A)\cdot (B\cdot C) & \text{(theo \textbf{Trường hợp 1})}                            \\
                                              & = A\cdot (B\cdot C)     & \text{(theo Mệnh đề~\ref{proposition:dedekind-cuts-and-sign})}
                        \end{align*}
                  \item $A < O, B > O, C < O$
                        \begin{align*}
                            (A\cdot B)\cdot C & = (-A\cdot B)\cdot (-C)   & \text{(theo Mệnh đề~\ref{proposition:dedekind-cuts-and-sign})} \\
                                              & = ((-A)\cdot B)\cdot (-C) & \text{(theo Mệnh đề~\ref{proposition:dedekind-cuts-and-sign})} \\
                                              & = (-A)\cdot (B\cdot (-C)) & \text{(theo \textbf{Trường hợp 1})}                            \\
                                              & = (-A)\cdot (-B\cdot C)   & \text{(theo Mệnh đề~\ref{proposition:dedekind-cuts-and-sign})} \\
                                              & = A\cdot (B\cdot C)       & \text{(theo Mệnh đề~\ref{proposition:dedekind-cuts-and-sign})}
                        \end{align*}
                  \item $A < O, B < O, C > O$
                        \begin{align*}
                            (A\cdot B)\cdot C & = ((-A)\cdot (-B))\cdot C & \text{(theo Mệnh đề~\ref{proposition:dedekind-cuts-and-sign})} \\
                                              & = (-A)\cdot ((-B)\cdot C) & \text{(theo \textbf{Trường hợp 1})}                            \\
                                              & = (-A)\cdot (-B\cdot C)   & \text{(theo Mệnh đề~\ref{proposition:dedekind-cuts-and-sign})} \\
                                              & = A\cdot (B\cdot C)       & \text{(theo Mệnh đề~\ref{proposition:dedekind-cuts-and-sign})}
                        \end{align*}
                  \item $A < O, B < O, C < O$
                        \begin{align*}
                            (A\cdot B)\cdot C & = ((-A)\cdot (-B))\cdot C     & \text{(theo Mệnh đề~\ref{proposition:dedekind-cuts-and-sign})}       \\
                                              & = -((-A)\cdot (-B))\cdot (-C) & \text{(theo Mệnh đề~\ref{proposition:dedekind-cuts-and-sign})}       \\
                                              & = -(-A)\cdot ((-B)\cdot (-C)) & \text{(theo \textbf{Trường hợp 1})}                                  \\
                                              & = -(-A)\cdot (B\cdot C)       & \text{(theo Mệnh đề~\ref{proposition:dedekind-cuts-and-sign})}       \\
                                              & = (-(-A))\cdot (B\cdot C)     & \text{(theo Mệnh đề~\ref{proposition:dedekind-cuts-and-sign})}       \\
                                              & = A\cdot (B\cdot C)           & \text{(theo Định lý~\ref{theorem:additive-inversion-is-involutive})}
                        \end{align*}
              \end{enumerate}

              Như vậy, với mọi lát cắt $A, B, C$, chúng ta có $(A\cdot B)\cdot C = A\cdot (B\cdot C)$.
        \item Theo chứng minh của Mệnh đề~\ref{proposition:dedekind-cuts-and-sign}, nếu ít nhất một trong các lát cắt $A, B$ bằng $O$ thì $A\cdot B = B\cdot A = O$. Dưới đây, chúng ta xét tất cả các trường hợp mà $A\ne O, B\ne O$.
              \begin{enumerate}[label={\textbf{Trường hợp \arabic*.}},itemindent=1cm]
                  \item $A > O, B > O$.

                        Theo định nghĩa phép nhân lát cắt, Mệnh đề~\ref{proposition:nonnegative-elements-of-dedekind-cuts-multiplication}, và tính chất giao hoán của phép nhân số hữu tỉ, chúng ta có
                        \begin{align*}
                            A\cdot B & = \{ ab \mid a\in A\wedge b\in B\wedge a\geq 0\wedge b\geq 0 \} \cup O \\
                                     & = \{ ba \mid a\in A\wedge b\in B\wedge a\geq 0\wedge b\geq 0 \} \cup O \\
                                     & = B\cdot A.
                        \end{align*}
                  \item $A > O, B < O$
                        \begin{align*}
                            A\cdot B & = -A\cdot (-B)   & \text{(theo Mệnh đề~\ref{proposition:dedekind-cuts-and-sign})}       \\
                                     & = -(-B)\cdot A   & \text{(theo \textbf{Trường hợp 1})}                                  \\
                                     & = (-(-B))\cdot A & \text{(theo Mệnh đề~\ref{proposition:dedekind-cuts-and-sign})}       \\
                                     & = B\cdot A       & \text{(theo Định lý~\ref{theorem:additive-inversion-is-involutive})}
                        \end{align*}
                  \item $A < O, B > O$
                        \begin{align*}
                            A\cdot B & = -(-A)\cdot B   & \text{(theo Mệnh đề~\ref{proposition:dedekind-cuts-and-sign})}       \\
                                     & = -B\cdot (-A)   & \text{(theo \textbf{Trường hợp 1})}                                  \\
                                     & = B\cdot (-(-A)) & \text{(theo Mệnh đề~\ref{proposition:dedekind-cuts-and-sign})}       \\
                                     & = B\cdot A       & \text{(theo Định lý~\ref{theorem:additive-inversion-is-involutive})}
                        \end{align*}
                  \item $A < O, B < O$
                        \begin{align*}
                            A\cdot B & = (-A)\cdot (-B) & \text{(theo Mệnh đề~\ref{proposition:dedekind-cuts-and-sign})} \\
                                     & = (-B)\cdot (-A) & \text{(theo \textbf{Trường hợp 1})}                            \\
                                     & = B\cdot A       & \text{(theo Mệnh đề~\ref{proposition:dedekind-cuts-and-sign})}
                        \end{align*}
              \end{enumerate}

              Như vậy, với mọi lát cắt $A, B$, chúng ta có $A\cdot B = B\cdot A$.
    \end{enumerate}
\end{proof}

\begin{theorem}
    Phép nhân lát cắt có tính chất phân phối với phép cộng lát cắt. Nói cách khác, với mọi lát cắt $A, B, C$, chúng ta có
    \[
        \begin{split}
            (A + B)\cdot C = A\cdot C + B\cdot C, \\
            C\cdot (A + B) = C\cdot A + C\cdot B.
        \end{split}
    \]
\end{theorem}

\begin{proof}
    Nếu $A = O$ thì
    \[
        (A + B)\cdot C = B\cdot C = O + B\cdot C = A\cdot C + B\cdot C.
    \]

    Nếu $B = O$ thì
    \[
        (A + B)\cdot C = A\cdot C = A\cdot C + O = A\cdot C + B\cdot C.
    \]

    Nếu $C = O$ thì
    \[
        (A + B)\cdot C = O = O + O = A\cdot C + B\cdot C.
    \]

    Chúng ta xem xét các trường hợp mà $A\ne O, B\ne O, C\ne O$.
    \begin{enumerate}[label={\textbf{Trường hợp \arabic*.}},itemindent=2cm]
        \item $A > O, B > O, C > O$.

              Vì $A > O, B > O, C > O$ nên $(A + B)\cdot C$ và $A\cdot C + B\cdot C$ là các lát cắt dương.

              Để được ngắn gọn, trong chứng minh này, chúng ta kí hiệu $A_{0}$ là tập hợp con của $A$ gồm tất cả các phần tử hữu tỉ không âm của $A$.

              Theo định nghĩa phép nhân lát cắt và tính chất phân phối của phép nhân với phép cộng số hữu tỉ, chúng ta có
              \begin{align*}
                  (A + B)\cdot C      & = \{ xc \mid x\in {(A+B)}_{0} \wedge c\in C_{0} \} \cup O                                                                                                                                                 \\
                                      & = \{ (a+b)c \mid a\in A_{0}\wedge b\in B_{0}\wedge c\in C_{0} \}\cup O                                           & \text{(theo Mệnh đề~\ref{proposition:nonnegative-elements-of-dedekind-cuts-addition})} \\
                                      & = \{ ac + bc \mid a\in A_{0}\wedge b\in B_{0}\wedge c\in C_{0} \} \cup O;                                                                                                                                 \\
                  A\cdot C + B\cdot C & = \{ a\cdot c \mid a\in A_{0}\wedge c\in C_{0} \}\cup O + \{ b\cdot c \mid b\in B_{0}\wedge c\in C_{0} \}\cup O.
              \end{align*}

              Chúng ta sẽ chứng minh rằng $(A + B)\cdot C \subseteq A\cdot C + B\cdot C$ và $A\cdot C + B\cdot C \subseteq (A + B)\cdot C$.

              Giả sử $x$ là một phần tử của $(A + B)\cdot C$. Nếu $x\leq 0$ thì $x$ cũng thuộc $A\cdot C + B\cdot C$ vì $A\cdot C + B\cdot C$ là một lát cắt dương.

              Còn nếu $x > 0$ thì theo Mệnh đề~\ref{proposition:nonnegative-elements-of-dedekind-cuts-multiplication}, tồn tại số hữu tỉ không âm $d$ thuộc $A + B$ và số hữu tỉ không âm $c$ thuộc $C$ sao cho $x = dc$. Theo Mệnh đề~\ref{proposition:nonnegative-elements-of-dedekind-cuts-addition}, tồn tại số hữu tỉ không âm $a$ thuộc $A$ và số hữu tỉ không âm $b$ thuộc $B$ sao cho $a + b = d$. Khi đó, $x = (a + b)c = ac + bc$. Theo định nghĩa phép nhân lát cắt và phép cộng lát cắt, chúng ta suy ra $ac$ thuộc $A\cdot C$, $bc$ thuộc $B\cdot C$, và $ac + bc$ thuộc $A\cdot C + B\cdot C$.

              Do đó $(A + B)\cdot C \subseteq A\cdot C + B\cdot C$.

              Giả sử $x$ là một phần tử của $A\cdot C + B\cdot C$. Nếu $x\leq 0$ thì $x$ cũng thuộc $(A + B)\cdot C$ vì $(A + B)\cdot C$ là một lát cắt dương.

              Còn nếu $x > 0$ thì theo Mệnh đề~\ref{proposition:nonnegative-elements-of-dedekind-cuts-addition}, tồn tại phần tử $y$ không âm của $A\cdot C$ và phần tử $z$ không âm của $B\cdot C$ sao cho $x = y + z$. Theo Mệnh đề~\ref{proposition:nonnegative-elements-of-dedekind-cuts-multiplication}, chúng ta có hai điều sau
              \begin{itemize}
                  \item Tồn tại số hữu tỉ không âm $a$ thuộc $A$ và số hữu tỉ không âm $c_{1}$ thuộc $C$ sao cho $y = ac_{1}$.
                  \item Tồn tại số hữu tỉ không âm $b$ thuộc $B$ và số hữu tỉ không âm $c_{2}$ thuộc $C$ sao cho $z = bc_{2}$.
              \end{itemize}

              Vì $x > 0$ nên $a$ và $b$ không thể đồng thời bằng $0$ (ngược lại sẽ dẫn đến $x = y + z = ac_{1} + bc_{2} = 0 + 0 = 0$). Chúng ta chọn $c = \dfrac{ac_{1} + bc_{2}}{a + b}$. Khi đó $c$ không vượt quá số lớn hơn trong hai số $c_{1}$ và $c_{2}$. Vì $C$ có tính đóng dưới nên $c$ thuộc $C$.
              \[
                  x = y + z = ac_{1} + bc_{2} = (a + b)\cdot\frac{ac_{1} + bc_{2}}{a + b} = ac + bc = (a + b)c.
              \]

              Đẳng thức trên cho thấy $x$ cũng thuộc lát cắt $(A + B)\cdot C$.

              Do đó $A\cdot C + B\cdot C \subseteq (A + B)\cdot C$.

              Như vậy $(A + B)\cdot C = A\cdot C + B\cdot C$.

              Chúng ta sẽ đưa các trường hợp dưới đây về \textbf{Trường hợp 1}.
        \item $A > O, B < O, C > O$.
              \begin{enumerate}[label={\textbf{Khả năng \arabic*.}},itemindent=1.5cm]
                  \item $A + B = O$.
                        \begin{align*}
                            (A + B)\cdot C & = O\cdot C = O = A\cdot C + (-A\cdot C)                                                                  \\
                                           & = A\cdot C + (-A)\cdot C                & \text{(theo Mệnh đề~\ref{proposition:dedekind-cuts-and-sign})} \\
                                           & = A\cdot C + B\cdot C
                        \end{align*}
                  \item $A + B > O$.
                        \begin{align*}
                            A\cdot C & = ((A + B) + (-B))\cdot C                                                                       \\
                                     & = (A + B)\cdot C + (-B)\cdot C & \text{(theo \textbf{Trường hợp 1})}                            \\
                                     & = (A + B)\cdot C + (-B\cdot C) & \text{(theo Mệnh đề~\ref{proposition:dedekind-cuts-and-sign})}
                        \end{align*}

                        Do đó $(A + B)\cdot C = A\cdot C + B\cdot C$.
                  \item $A + B < O$.
                        \begin{align*}
                            -B\cdot C & = (-B)\cdot C                   & \text{(theo Mệnh đề~\ref{proposition:dedekind-cuts-and-sign})} \\
                                      & = (A + (-(A + B)))\cdot C                                                                        \\
                                      & = A\cdot C + (-(A + B))\cdot C  & \text{(theo \textbf{Trường hợp 1})}                            \\
                                      & = A\cdot C + (-(A + B)\cdot C). & \text{(theo Mệnh đề~\ref{proposition:dedekind-cuts-and-sign})}
                        \end{align*}

                        Do đó $(A + B)\cdot C = A\cdot C + B\cdot C$.
              \end{enumerate}
        \item $A < O, B > O, C > O$.
              \begin{enumerate}[label={\textbf{Khả năng \arabic*.}},itemindent=1.5cm]
                  \item $A + B = O$.
                        \begin{align*}
                            (A + B)\cdot C & = O\cdot C = O = A\cdot C + (-A\cdot C) \\
                                           & = A\cdot C + (-A)\cdot C                \\
                                           & = A\cdot C + B\cdot C.
                        \end{align*}
                  \item $A + B > O$.
                        \begin{align*}
                            B\cdot C & = ((-A) + (A + B))\cdot C                                                                        \\
                                     & = (-A)\cdot C + (A + B)\cdot C  & \text{(theo \textbf{Trường hợp 1})}                            \\
                                     & = (-A\cdot C) + (A + B)\cdot C. & \text{(theo Mệnh đề~\ref{proposition:dedekind-cuts-and-sign})}
                        \end{align*}

                        Do đó $(A + B)\cdot C = A\cdot C + B\cdot C$.
                  \item $A + B < O$.
                        \begin{align*}
                            -A\cdot C & = (-A)\cdot C                   & \text{(theo Mệnh đề~\ref{proposition:dedekind-cuts-and-sign})} \\
                                      & = ((-(A + B)) + B)\cdot C                                                                        \\
                                      & = (-(A + B))\cdot C + B\cdot C  & \text{(theo \textbf{Trường hợp 1})}                            \\
                                      & = (-(A + B)\cdot C) + B\cdot C. & \text{(theo Mệnh đề~\ref{proposition:dedekind-cuts-and-sign})}
                        \end{align*}

                        Do đó $(A + B)\cdot C = A\cdot C + B\cdot C$.
              \end{enumerate}
        \item $A < O, B < O, C > O$.
              \begin{align*}
                  -(A + B)\cdot C & = (-(A + B))\cdot C         & \text{(theo Mệnh đề~\ref{proposition:dedekind-cuts-and-sign})} \\
                                  & = ((-A) + (-B))\cdot C                                                                       \\
                                  & = (-A)\cdot C + (-B)\cdot C & \text{(theo \textbf{Trường hợp 1})}                            \\
                                  & = (-A\cdot C) + (-B\cdot C) & \text{(theo Mệnh đề~\ref{proposition:dedekind-cuts-and-sign})} \\
                                  & = -(A\cdot C + B\cdot C).
              \end{align*}

              Do đó $(A + B)\cdot C = A\cdot C + B\cdot C$.
        \item $C < O$.
              \begin{align*}
                  (A + B)\cdot C & = -(A + B)\cdot (-C)              & \text{(theo Mệnh đề~\ref{proposition:dedekind-cuts-and-sign})} \\
                                 & = -(A\cdot (-C) + B\cdot (-C))    & \text{(theo \textbf{Trường hợp 1, 2, 3, 4})}                   \\
                                 & = (-A\cdot (-C)) + (-B\cdot (-C)) & \text{(theo Mệnh đề~\ref{proposition:dedekind-cuts-and-sign})} \\
                                 & = A\cdot C + B\cdot C.            & \text{(theo Mệnh đề~\ref{proposition:dedekind-cuts-and-sign})}
              \end{align*}
    \end{enumerate}

    Như vậy, trong mọi trường hợp, chúng ta có $(A + B)\cdot C = A\cdot C + B\cdot C$. Mặt khác, vì phép nhân lát cắt có tính chất giao hoán nên chúng ta cũng có $C\cdot (A + B) = C\cdot A + C\cdot B$.
\end{proof}

\begin{theorem}
    Với mọi lát cắt $A$, chúng ta có $A\cdot I = I\cdot A = A$, trong đó $I$  là lát cắt $\{ x \mid x\in\mathbb{Q} \wedge x < 1 \}$.
\end{theorem}

\begin{proof}
    $I\ne O$ vì $0$ thuộc $I$ nhưng $0$ không thuộc $O$.

    Chúng ta xét ba trường hợp sau.
    \begin{enumerate}[label={\textbf{Trường hợp \arabic*.}},itemindent=2cm]
        \item $A = O$.

              Theo chứng minh của Mệnh đề~\ref{proposition:dedekind-cuts-and-sign}, $A\cdot I = O\cdot I = O = I\cdot O = I\cdot A$.
        \item $A > O$.

              Theo định nghĩa phép nhân lát cắt, chúng ta có
              \[
                  A\cdot I = \{ ae \mid a\in A\wedge e\in I\wedge a\geq 0\wedge e\geq 0 \}\cup O
              \]

              Giả sử $x$ là một phần tử của $A$. Nếu $x < 0$ thì $x$ thuộc $O$, kéo theo $x$ thuộc $A\cdot I$. Còn nếu $x\geq 0$, chúng ta chọn $a$ là một phần tử của $A$ và $x < a$ (tồn tại $a$ như vậy vì $A$ không có phần tử lớn nhất). Khi đó $x = a\cdot\dfrac{x}{a}$, mà $a$ thuộc $A$ và $0\leq \dfrac{x}{a} < 1$ nên $x$ thuộc $A\cdot I$. Như vậy $A\subseteq A\cdot I$.

              Giả sử $x$ là một phần tử của $A\cdot I$. Nếu $x < 0$ thì $x$ thuộc $A$ (vì $A$ là một lát cắt dương). Còn nếu $x\geq 0$ thì theo Mệnh đề~\ref{proposition:nonnegative-elements-of-dedekind-cuts-multiplication}, tồn tại số hữu tỉ không âm $a$ thuộc $A$ và số hữu tỉ không âm $e$ thuộc $I$ sao cho $x = ae$. Vì $e < 1$ và $a\geq 0$ nên $ae\leq a\cdot 1 = a$, kéo theo $x = ae$ thuộc $A$. Như vậy $A\cdot I\subseteq A$.

              Do đó $A\cdot I = A$. Theo tính chất giao hoán của phép nhân lát cắt, chúng ta suy ra $I\cdot A = A\cdot I = A$.
        \item $A < O$.

              Theo \textbf{Trường hợp 2}, Mệnh đề~\ref{proposition:dedekind-cuts-and-sign}, và Định lý~\ref{theorem:additive-inversion-is-involutive}, chúng ta có
              \[
                  \begin{split}
                      A\cdot I = -(-A)\cdot I = -(-A) = A, \\
                      I\cdot A = -I\cdot (-A) = -(-A) = A.
                  \end{split}
              \]

              Do đó $A\cdot I = I\cdot A = A$.
    \end{enumerate}

    Vậy với mọi lát cắt $A$, chúng ta có $A\cdot I = I\cdot A = A$.
\end{proof}

Để kiểm chứng tiên đề cuối cùng trong các tiên đề về trường, chúng tôi sử dụng định nghĩa và kết quả sau (ba phần cuối cùng của mệnh đề sau được sử dụng để kiểm tra tiên đề về trường).
\begin{proposition}[Lũy thừa nguyên]
    $q$ là một số hữu tỉ và $n$ là một số nguyên. Nếu $q\ne 0$, bằng đệ quy, chúng ta định nghĩa
    \[
        q^{n} = \begin{cases}
            1                 & \text{khi $n = 0$}                    \\
            q\cdot q^{n-1}    & \text{khi $n$ là một số nguyên dương} \\
            \dfrac{1}{q^{-n}} & \text{khi $n$ là một số nguyên âm}
        \end{cases}
    \]

    Còn nếu $q = 0$, chúng ta định nghĩa $q^{n} = 0$ với mọi số nguyên dương $n$.

    \begin{enumerate}[label={(\roman*)}]
        \item Chứng minh rằng nếu $q\ne 0$ thì với mọi số nguyên $n$, chúng ta có $q^{-n} = \dfrac{1}{q^{n}}$.
        \item Chứng minh rằng nếu $q\ne 0$ thì với mọi số nguyên $n$, chúng ta có $q^{n+1} = qq^{n}$.
        \item Chứng minh rằng nếu $q\ne 0$ thì với mọi số nguyên $m, n$, chúng ta có $q^{m + n} = q^{m}q^{n}$.
        \item Chứng minh rằng nếu $0 < q < 1$ thì $0 < q^{n} < 1$ với mọi số nguyên dương $n$.
        \item Chứng minh rằng nếu $0 < q < 1$ thì $1 < q^{-n}$ với mọi số nguyên dương $n$.
        \item (Bất đẳng thức Bernoulli) Nếu số hữu tỉ $q$ thỏa mãn $q > -1$ thì với mọi số nguyên không âm $n$, có ${(1 + q)}^{n}\geq 1 + nq$.
    \end{enumerate}
\end{proposition}

\begin{proof}
    \begin{enumerate}[label={(\roman*)}]
        \item Nếu $n = 0$ thì $q^{-n} = q^{0} = 1 = \frac{1}{1} = \frac{1}{q^{n}}$.

              Nếu $n$ là một số nguyên dương thì $q^{-n} = \dfrac{1}{q^{n}}$ theo định nghĩa lũy thừa nguyên.

              Nếu $n$ là một số nguyên âm thì $q^{n} = \dfrac{1}{q^{-n}}$ theo định nghĩa lũy thừa nguyên. Vì $q^{-n}$ là nghịch đảo của $\dfrac{1}{q^{-n}}$ và $\dfrac{1}{q^{n}}$ là nghịch đảo của $q^{n}$ nên $q^{-n} = \dfrac{1}{q^{n}}$.

              Như vậy, nếu $q\ne 0$ thì với mọi số nguyên $n$, chúng ta có $q^{-n} = \dfrac{1}{q^{n}}$.
        \item Trong trường hợp $n$ là một số nguyên không âm thì $q^{n+1} = qq^{n}$ theo định nghĩa lũy thừa nguyên.

              Trong trường hợp $n = -1$ thì $q^{n+1} = q^{0} = 1 = \dfrac{q}{q} = qq^{-1}$.

              Trong trường hợp $n$ là số nguyên âm nhỏ hơn $-1$ thì theo định nghĩa lũy thừa nguyên, chúng ta có
              \[
                  q^{n+1} = \dfrac{1}{q^{-(n+1)}} = \dfrac{1}{q^{-n-1}} = \dfrac{q}{qq^{-n-1}}
              \]

              Theo hai trường hợp trên, chúng ta có $q^{-n} = qq^{-n-1}$. Cùng với định nghĩa lũy thừa nguyên và phần (i), chúng ta suy ra
              \[
                  \dfrac{q}{qq^{-n-1}} = \dfrac{q}{q^{-n}} = qq^{n}
              \]

              Như vậy, nếu $q\ne 0$ thì $q^{n+1} = qq^{n}$ với mọi số nguyên $n$.
        \item Trước hết, chúng ta chứng minh bằng nguyên lý quy nạp toán học cho trường hợp $m + n\geq 0$.

              Khi $n = -m$ thì theo định nghĩa lũy thừa nguyên và phần (i), chúng ta có $q^{m + n} = q^{0} = 1 = \frac{q^{m}}{q^{m}} = q^{m}q^{-m} = q^{m}q^{n}$.

              Giả sử với $n = k\geq -m$, có $q^{m+k} = q^{m}q^{k}$. Theo định nghĩa lũy thừa nguyên, $q^{m+(k+1)} = q^{(m+k)+1} = qq^{m+k}$. Theo giả thiết quy nạp, $qq^{m+k} = q(q^{m}q^{k}) = q^{m}(qq^{k})$. Theo phần (ii), chúng ta có $q^{m}(qq^{k}) = q^{m}q^{k+1}$.

              Theo nguyên lý quy nạp toán học, $q^{m + n} = q^{m}q^{n}$ với mọi số nguyên $m, n$ thỏa mãn $m+n\geq 0$.

              Trong trường hợp $m + n < 0$, theo trường hợp trước và phần (i), chúng ta có
              \[
                  q^{m+n} = \dfrac{1}{q^{(-m) + (-n)}} = \dfrac{1}{q^{-m}q^{-n}} = \dfrac{1}{q^{-m}}\dfrac{1}{q^{-n}} = q^{m}q^{n}.
              \]

              Như vậy, nếu $q\ne 0$ thì $q^{m + n} = q^{m}q^{n}$ với mọi số nguyên $m, n$.
        \item Khi $n = 1$, chúng ta có $0 < q^{1} < 1$ vì $0 < q < 1$ và $q^{1} = q$.

              Giả sử với $n = k\geq 1$, có $0 < q^{k} < 1$ và $q^{-k} > 1$. Vì $0 < q < 1$ nên $0 < q\cdot q^{k} = q^{k+1}$. Bên cạnh đó, theo giả thiết quy nạp và định nghĩa lũy thừa nguyên, chúng ta có $q^{k+1} = q\cdot q^{k} < q^{k} < 1$. Do đó $0 < q^{k+1} < 1$.

              Theo nguyên lý quy nạp toán học, nếu $0 < q < 1$ thì $0 < q^{n} < 1$ với mọi số nguyên dương $n$.
        \item Theo phần (iv), nếu $0 < q < 1$ thì $0 < q^{n} < 1$. Bên cạnh đó, vì $0 < q^{-n}$ nên $q^{-n}q^{n} < q^{-n}$, kéo theo $1 < q^{-n}$.
        \item Khi $n = 0$ thì chúng ta có ${(1 + q)}^{n} = {(1 + q)}^{0} = 1 \geq 1 + 0q$.

              Giả sử với số nguyên $n = k\geq 0$, có ${(1 + q)}^{k} \geq 1 + kq$. Vì $(1 + q) > 0$ nên theo định nghĩa lũy thừa nguyên và  giả thiết quy nạp, chúng ta suy ra
              \[
                  {(1 + q)}^{k+1} = (1 + q){(1 + q)}^{k} \geq (1 + q)(1 + kq) = 1 + (k+1)q + q^{2} \geq 1 + (k + 1)q
              \]

              Vậy theo nguyên lý quy nạp toán học, nếu số hữu tỉ $q$ thỏa mãn $q > -1$ thì với mọi số nguyên không âm $n$, có ${(1 + q)}^{n}\geq 1 + nq$.
    \end{enumerate}
\end{proof}

\begin{theorem}
    Với mỗi lát cắt $A\ne O$, tồn tại lát cắt $B$ sao cho $A\cdot B = B\cdot A = I$.
\end{theorem}

\begin{proof}
    Chúng ta xét hai trường hợp lát cắt dương và lát cắt âm.
    \begin{enumerate}[label={\textbf{Trường hợp \arabic*.}},itemindent=2cm]
        \item $A > O$.

              Vì $A > O$ nên tập hợp $\mathbb{Q}\setminus A$ gồm toàn các số hữu tỉ dương.

              Chúng ta định nghĩa tập hợp $B$ như sau và chứng minh $B$ là một lát cắt.
              \[
                  B = \left\{ \frac{e}{x} \mid e\in I \wedge x\in \mathbb{Q}\setminus A\wedge e\geq 0 \right\}\cup O
              \]

              \begin{enumerate}[label={(DC\arabic*)}]
                  \item Vì $O$ là tập hợp con của $B$ nên $B$ khác rỗng.
                  \item Chọn $a$ là một số hữu tỉ dương thuộc $A$. Khi đó, mọi số hữu tỉ âm thuộc $B$ đều nhỏ hơn $a$. Bên cạnh đó, các phần tử $\dfrac{e}{x}$ của $B$ (trong đó $e$ thuộc $I$ và $e\geq 0$, $x$ thuộc $\mathbb{Q}\setminus A$) thỏa mãn $\dfrac{e}{x}\leq \dfrac{e}{a} < \dfrac{1}{a}$. Do đó, mọi phần tử của $B$ đều khác $\dfrac{1}{a}$, kéo theo $B$ không phải toàn bộ tập hợp số hữu tỉ.
                  \item Giả sử $b$ là một phần tử của $B$. Nếu $b < 0$ thì $0$ chính là một phần tử của $B$ và lớn hơn $b$. Còn nếu $b\geq 0$ thì theo định nghĩa của tập hợp $B$, tồn tại số hữu tỉ không âm $e$ thuộc $I$ và tồn tại số hữu tỉ $x$ thuộc $\mathbb{Q}\setminus A$ sao cho $b = \dfrac{e}{a}$. Vì $I$ là một lát cắt nên trong $I$ tồn tại phần tử $e'$ lớn hơn $e$, kéo theo $b = \dfrac{e}{a} < \dfrac{e'}{a}$. Hơn nữa, theo định nghĩa của tập hợp $B$, $\dfrac{e'}{a}$ cũng là một phần tử của $B$. Do đó tập hợp $B$ không có phần tử lớn nhất, vì với mỗi phần tử của $B$, chúng ta luôn tìm được phần tử lớn hơn.
                  \item Giả sử $b$ là một phần tử của $B$ và $b'$ là một số hữu tỉ nhỏ hơn $b$. Nếu $b' < 0$ thì $b'$ thuộc $B$ vì $O$ là tập hợp con của $B$. Còn nếu $b'\geq 0$ thì $b > 0$. Khi đó $b' = \dfrac{b'}{b}\cdot b$. Theo định nghĩa tập hợp $B$, chúng ta suy ra $b'$ thuộc $B$. Do đó $B$ đóng dưới.
              \end{enumerate}

              Như vậy $B$ là một lát cắt. Tiếp theo, chúng ta chứng minh $A\cdot B \subseteq I$ và $I\subseteq A\cdot B$.

              \bigskip

              Giả sử $x$ là một phần tử của $A\cdot B$. Nếu $x < 0$ thì $x$ cũng thuộc $I$. Nếu $x\geq 0$ thì theo Mệnh đề~\ref{proposition:nonnegative-elements-of-dedekind-cuts-multiplication}, tồn tại số hữu tỉ không âm $a$ thuộc $A$ và số hữu tỉ không âm $b$ thuộc $B$ sao cho $x = ab$. Theo định nghĩa của tập hợp $B$, tồn tại số hữu tỉ không âm $e$ thuộc $I$ và số hữu tỉ dương $y$ thuộc $\mathbb{Q}\setminus A$ sao cho $b = \dfrac{e}{y}$. Từ những đẳng thức trên, chúng ta suy ra $x = a\cdot\dfrac{e}{y} = e\cdot\dfrac{a}{y}$. Vì $e\geq 0, \dfrac{a}{y}\geq 0$ và $e < 1, \dfrac{a}{y} < 1$, chúng ta suy ra $x < 1$, điều này kéo theo $x$ thuộc $I$. Do đó $A\cdot B \subseteq I$.
              \bigskip

              Giả sử $x$ là một phần tử của $I$. Chúng ta sẽ chứng minh $x$ cũng là một phần tử của $A\cdot B$.
              \begin{enumerate}[label={\textbf{Khả năng \arabic*.}},itemindent=1.5cm]
                  \item $x\leq 0$

                        $x\leq 0$ thì $x$ cũng thuộc $A\cdot B$.
                  \item $0 < x < 1$.

                        Vì $A\cdot B$ không có phần tử lớn nhất nên trong $A\cdot B$ tồn tại phần tử $y > x$, lưu ý rằng $0 < y < 1$. Chúng ta xét tập hợp sau
                        \[
                            S = \{ n \mid n\in\mathbb{Z} \wedge y^{n}\in A \}
                        \]

                        Theo bất đẳng thức Bernoulli, với mọi số nguyên dương $n$, chúng ta có
                        \[
                            y^{-n} = \dfrac{1}{y^{n}} = {\left(\frac{1}{y}\right)}^{n} = {\left(1 + \left(\frac{1}{y} - 1\right)\right)}^{n}\geq 1 + n\left(\frac{1}{y} - 1\right)
                        \]

                        Chọn $c$ là một số hữu tỉ dương bất kì. Theo Mệnh đề~\ref{proposition:integral-part-of-rational-numbers}, tồn tại số nguyên $m$ sao cho $m > \dfrac{c - 1}{\dfrac{1}{y} - 1}$, kéo theo $1 + m\left(\frac{1}{y} - 1\right) > c$. Chọn số nguyên dương $n > m$ thì chúng ta có $y^{-n} > c$. Điều này kéo theo $cy^{n} < 1$ và $y^{n} < \frac{1}{c}$.

                        Trên đây, chúng ta vừa chứng minh mệnh đề: Với mỗi số hữu tỉ dương $c$, tồn tại số nguyên $n$ sao cho $y^{n} < \frac{1}{c}$. Chúng ta sẽ sử dụng mệnh đề này để chỉ ra tập hợp $S$ thỏa mãn giả thiết của nguyên lý thứ tự tốt.

                        Chọn $a$ là một số hữu tỉ dương thuộc $A$, khi đó $\frac{1}{a}$ cũng là một số hữu tỉ dương. Theo mệnh đề vừa chứng minh, tồn tại số nguyên $n$ sao cho $y^{n} < a$. Theo tính chất đóng dưới của lát cắt, $y^{n} < a$ kéo theo $y^{n}$ thuộc $A$ và số nguyên $n$ đó thuộc tập hợp $S$ (theo định nghĩa tập hợp $S$). Do đó tập hợp $S$ khác rỗng.

                        Chọn $a_{0}$ là một cận trên của $A$. Vì $A$ là một lát cắt dương nên $a_{0} > 0$. Theo mệnh đề vừa chứng minh, tồn tại số nguyên $n$ sao cho $y^{n} < \frac{1}{a_{0}}$, kéo theo $y^{-n} > a_{0}$, điều này có nghĩa là $y^{-n}$ là một cận trên của $A$ và $-n$ không thuộc $S$ (theo định nghĩa tập hợp $S$). Ngoài ra, nếu số nguyên $m$ không thuộc $S$ thì số nguyên $m-1$ cũng không thuộc $S$, vì $y^{m} < y^{m-1}$. Do đó, nếu chọn $n_{0}$ là một số nguyên không thuộc $S$ thì mọi phần tử $n$ của $S$ phải lớn hơn hoặc bằng $n_{0}$.

                        Theo nguyên lý thứ tự tốt, tập hợp $S$ có phần tử nhỏ nhất. Chúng ta kí hiệu phần tử nhỏ nhất của $S$ là $p$. Khi đó $y^{p}$ thuộc $A$ và $y^{p-1}$ không thuộc $A$, và
                        \[
                            x = y\cdot\frac{x}{y} = \frac{y^{p}}{y^{p-1}}\cdot\frac{x}{y} = y^{p}\cdot \frac{x/y}{y^{p-1}}
                        \]

                        Vì $0 < y^{p}$, $y^{p}$ thuộc $A$, $0\leq x/y < 1$ và $y^{p-1}$ thuộc $\mathbb{Q}\setminus A$ nên $x$ thuộc lát cắt $A\cdot B$.
              \end{enumerate}

              Do đó $I\subseteq A\cdot B$.

              Vì $A\cdot B\subseteq I$ và $I\subseteq A\cdot B$ nên $A\cdot B = I$. Vì phép nhân lát cắt có tính chất giao hoán, chúng ta suy ra $B\cdot A = A\cdot B = I$.
        \item $A < O$.

              $A < O$ thì $-A > O$. Theo \textbf{Trường hợp 1}, tồn tại lát cắt $C$ sao cho $(-A)\cdot C = C\cdot (-A) = I$. Như vậy, với lát cắt $B = -C$, theo Mệnh đề~\ref{proposition:dedekind-cuts-and-sign}, chúng ta có
              \[
                  \begin{split}
                      A\cdot B = (-A)\cdot (-B) = (-A)\cdot C = I, \\
                      B\cdot A = (-B)\cdot (-A) = C\cdot (-A) = I.
                  \end{split}
              \]

              Do đó $A\cdot B = B\cdot A = I$.
    \end{enumerate}

    Vậy, nếu $A \ne O$ thì tồn tại lát cắt $B$ sao cho $A\cdot B = B\cdot A = I$.
\end{proof}

\subsection{Lát cắt và tiên đề về cận trên}

Chúng ta kiểm tra tiên đề cuối cùng: tiên đề về cận trên.

\begin{theorem}
    Cho tập hợp $\mathscr{D}$ là một tập hợp con khác rỗng của tập hợp các lát cắt Dedekind hữu tỉ $\mathscr{D}_{\mathbb{Q}}$ và $\mathscr{D}$ có cận trên. Khi đó $\mathscr{D}$ có cận trên nhỏ nhất.
\end{theorem}

\begin{proof}
    Chúng ta xét tập hợp $S = \bigcup\limits_{A\in\mathscr{D}} A$, trong đó $A$ là các phần tử (là các lát cắt) của $\mathscr{D}$. Tập hợp $S$ bao gồm các số hữu tỉ.

    Đầu tiên, chúng ta chứng minh $S$ là một lát cắt.
    \begin{enumerate}[label={(DC\arabic*)}, itemindent=0.2cm]
        \item $S$ là hợp thành của các lát cắt nên $S$ khác rỗng.
        \item Vì $\mathscr{D}$ có cận trên nên mọi phần tử $A$ của $\mathscr{D}$ đều nhỏ hơn hoặc bằng một lát cắt $B$ nào đó. Do đó $S\leq B$, kéo theo $S$ không phải toàn bộ tập hợp số hữu tỉ.
        \item Giả sử $x$ là một phần tử của $S$. Theo định nghĩa của $S$ và phép toán hợp của các tập hợp, tồn tại một lát cắt $A$ (là phần tử của $S$) sao cho $x$ thuộc $A$. Vì lát cắt $A$ không có phần tử lớn nhất nên tồn tại phần tử $y$ của $A$ sao cho $x < y$. Do $y$ thuộc $A$ nên $y$ cũng thuộc $S$. Do đó $S$ không có phần tử lớn nhất.
        \item Giả sử $x$ là một phần tử của $S$ và $y$ là một số hữu tỉ nhỏ hơn $x$.  Theo định nghĩa của $S$ và phép toán hợp của các tập hợp, tồn tại một lát cắt $A$ (là phần tử của $S$) sao cho $x$ thuộc $A$. Vì lát cắt $A$ đóng dưới nên $y$ thuộc $A$, kéo theo $y$ cũng thuộc $S$. Do đó $S$ đóng dưới.
    \end{enumerate}

    Như vậy $S$ là một lát cắt.

    Tiếp theo, chúng ta chứng minh $S$ là cận trên nhỏ nhất của $\mathscr{D}$.

    Giả sử $X$ là một cận trên của $\mathscr{D}$. Theo định nghĩa quan hệ $\leq$ trên tập hợp $\mathscr{D}_{\mathbb{Q}}$, mọi phần tử $A$ (cũng là các lát cắt) của $\mathscr{D}$ thỏa mãn $A\leq X$ ($A\subseteq X$). Do đó, hợp của tất cả các phần tử của $\mathscr{D}$, hay $S = \bigcup\limits_{A\in\mathscr{D}} A$ thỏa mãn $S\subseteq X$ ($S\leq X$). Do đó $S$ nhỏ hơn hoặc bằng của mọi cận trên của $\mathscr{D}$, điều này có nghĩa là $S$ là cận trên nhỏ nhất của $\mathscr{D}$.

    Vậy $\mathscr{D}$ có cận trên nhỏ nhất.
\end{proof}

\subsection{Liên hệ lát cắt Dedekind với số hữu tỉ}

\begin{theorem}
    Ánh xạ $\iota: \mathbb{Q}\to \mathscr{D}_{\mathbb{Q}}$ được định nghĩa bởi $\iota(q) = q^{*}$, trong đó $q^{*}$ là lát cắt $\{ x \mid x\in\mathbb{Q} \wedge x < q \}$ là một đơn ánh nhưng không phải một song ánh. Bên cạnh đó, với mọi số hữu tỉ $q_{1}, q_{2}$, chúng ta có
    \[
        \begin{split}
            \iota(q_{1} + q_{2}) = \iota(q_{1}) + \iota(q_{2}), \\
            \iota(q_{1}q_{2}) = \iota(q_{1})\iota(q_{2}), \\
            q_{1}\leq q_{2} \implies \iota(q_{1})\leq \iota(q_{2}).
        \end{split}
    \]
\end{theorem}

\begin{proof}
    \begin{enumerate}[label={(\roman*)}]
        \item Giả sử $\iota(q_{1}) = \iota(q_{2})$. Giả sử phản chứng rằng $q_{1} < q_{2}$, khi đó $q_{1}$ thuộc $\iota(q_{2})$ và $\iota(q_{1})\subset \iota(q_{2})$. Giả sử phản chứng rằng $q_{2} < q_{1}$, khi đó $q_{2}$ thuộc $\iota(q_{1})$ và $\iota(q_{2})\subset \iota(q_{1})$. thì $q_{1}$ thuộc $\iota(q_{2})$ và $q_{2}$ thuộc $\iota(q_{1})$. Do đó giả sử phản chứng là sai, kéo theo $q_{1} = q_{2}$. Như vậy $\iota$ là một đơn ánh.

              Theo Mệnh đề~\ref{proposition:irrational-cut}, $S = \{ x \mid x\in\mathbb{Q}, 0 < x \text{ và } x^{2} < 2 \}$ là một lát cắt. Giả sử phản chứng rằng tồn tại số hữu tỉ $q$ sao cho $\iota(q) = S$. Cũng theo Mệnh đề~\ref{proposition:irrational-cut}, $\mathbb{Q}\setminus S$ không có phần tử nhỏ nhất. Mà $\mathbb{Q}\setminus\iota(q)$ có phần tử nhỏ nhất là $q$. Do đó $\iota(q)\ne S$. Như vậy không tồn tại số hữu tỉ $q$ nào sao cho $\iota(q) = S$, điều này có nghĩa là $\iota$ không phải một toàn ánh, và cũng không phải một song ánh.

        \item Giả sử $x$ là một phần tử của $\iota(q_{1}) + \iota(q_{2})$. Theo định nghĩa phép cộng lát cắt, tồn tại hai phần tử $x_{1}$ trong $\iota(q_{1})$ và $x_{2}$ trong $\iota(q_{2})$ sao cho $x = x_{1} + x_{2}$. Vì $x_{1} < q_{1}$ và $x_{2} < q_{2}$ nên
              \[
                  x = x_{1} + x_{2} < q_{1} + x_{2} < q_{1} + q_{2}
              \]

              Theo định nghĩa của ánh xạ $\iota$, chúng ta suy ra $x$ thuộc $\iota(q_{1} + q_{2})$. Do đó $ \iota(q_{1}) + \iota(q_{2}) \subseteq \iota(q_{1} + q_{2})$.

              Giả sử $x$ là một phần tử của $\iota(q_{1} + q_{2})$. Chúng ta chọn $x_{1} = q_{1} + \frac{x - (q_{1} + q_{2})}{2}$ và $x_{2} = q_{2} + \frac{x - (q_{1} + q_{2})}{2}$. Vì $x < q_{1} + q_{2}$ nên $x_{1} < q_{1}$ và $x_{2} < q_{2}$, kéo theo $x_{1}$ thuộc $\iota(q_{1})$ và $x_{2}$ thuộc $\iota(q_{2})$. Bên cạnh đó
              \[
                  x_{1} + x_{2} = (q_{1} + q_{2}) + (x - (q_{1} + q_{2})) = x
              \]

              Theo định nghĩa phép cộng lát cắt thì $x$ thuộc $\iota(q_{1}) + \iota(q_{2})$. Do đó $\iota(q_{1} + q_{2}) \subseteq \iota(q_{1}) + \iota(q_{2})$.

              Như vậy, với mọi số hữu tỉ $q_{1}, q_{2}$, chúng ta có $ \iota(q_{1} + q_{2}) = \iota(q_{1}) + \iota(q_{2})$.
        \item Đầu tiên, chúng ta chứng minh $\iota(-q) = -\iota(q)$.

              Theo phần (i), $\iota(q) + \iota(-q) = \iota(-q) + \iota(q) = \iota(0) = O$. Do đó
              \begin{align*}
                  -\iota(q) & = (-\iota(q)) + O                      \\
                            & = (-\iota(q)) + (\iota(q) + \iota(-q)) \\
                            & = ((-\iota(q)) + \iota(q)) + \iota(-q) \\
                            & = O + \iota(-q)                        \\
                            & = \iota(-q).
              \end{align*}

              Để chứng minh $\iota(q_{1}q_{2}) = \iota(q_{1})\iota(q_{2})$ với mọi số hữu tỉ $q_{1}, q_{2}$, chúng ta xét các trường hợp sau.
              \begin{enumerate}[label={\textbf{Trường hợp \arabic*.}},itemindent=2cm]
                  \item $q_{1} = 0$ hoặc $q_{2} = 0$.

                        Nếu $q_{1} = 0$ thì $\iota(q_{1}q_{2}) = \iota(0) = O = O\cdot\iota(q_{2}) = \iota(q_{1})\iota(q_{2})$.

                        Nếu $q_{2} = 0$ thì $\iota(q_{1}q_{2}) = \iota(0) = O = \iota(q_{1})\cdot O = \iota(q_{1})\iota(q_{2})$.
                  \item $q_{1} > 0$ và $q_{2} > 0$.

                        $q_{1} > 0$ và $q_{2} > 0$ thì $\iota(q_{1}q_{2}), \iota(q_{1}), \iota(q_{2})$ là các lát cắt dương.

                        Giả sử $x$ là một phần tử của $\iota(q_{1}q_{2})$. Nếu $x < 0$ thì $x$ cũng thuộc $\iota(q_{1})\iota(q_{2})$. Nếu $x\geq 0$ thì theo Mệnh đề~\ref{proposition:nonnegative-elements-of-dedekind-cuts-multiplication}, tồn tại số hữu tỉ không âm $x_{1}$ thuộc $\iota(q_{1})$ và số hữu tỉ không âm $x_{2}$ thuộc $\iota(q_{2})$ sao cho $x = x_{1}x_{2}$. Theo định nghĩa phép nhân lát cắt thì $x_{1}x_{2}$ thuộc $\iota(q_{1})\iota(q_{2})$, kéo theo $x$ cũng thuộc $\iota(q_{1})\iota(q_{2})$. Do đó $\iota(q_{1}q_{2}) \subseteq \iota(q_{1})\iota(q_{2})$.

                        Giả sử $x$ là một phần tử của $\iota(q_{1})\iota(q_{2})$. Nếu $x < 0$ thì $x$ cũng thuộc $\iota(q_{1}q_{2})$. Nếu $x\geq 0$ thì theo định nghĩa phép nhân lát cắt, tồn tại số hữu tỉ không âm $x_{1}$ thuộc $\iota(q_{1})$ và số hữu tỉ không âm $x_{2}$ thuộc $\iota(q_{2})$ sao cho $x = x_{1}x_{2}$. Vì $0\leq x_{1} < q_{1}$ và $0\leq x_{2} < q_{2}$ nên $x_{1}x_{2}\leq q_{1}x_{2} < q_{1}q_{2}$, kéo theo $x = x_{1}x_{2}$ thuộc $\iota(q_{1}q_{2})$. Do đó $\iota(q_{1})\iota(q_{2}) \subseteq \iota(q_{1}q_{2})$.

                        Như vậy $\iota(q_{1}q_{2}) = \iota(q_{1})\iota(q_{2})$ với mọi số hữu tỉ dương $q_{1}, q_{2}$.
                  \item $q_{1} > 0$ và $q_{2} < 0$.
                        \begin{align*}
                            \iota(q_{1}q_{2}) & = \iota(-q_{1}(-q_{2})) = -\iota(q_{1}(-q_{2}))                                                                  \\
                                              & = -\iota(q_{1})\iota(-q_{2})                    & \text{(theo \textbf{Trường hợp 2})}                            \\
                                              & = \iota(q_{1})(-\iota(-q_{2}))                                                                                   \\
                                              & = \iota(q_{1})\iota(q_{2})                      & \text{(theo Mệnh đề~\ref{proposition:dedekind-cuts-and-sign})}
                        \end{align*}

                        Như vậy $\iota(q_{1}q_{2}) = \iota(q_{1})\iota(q_{2})$ với mọi số hữu tỉ dương $q_{1}$ và số hữu tỉ âm $q_{2}$.
                  \item  $q_{1} < 0$ và $q_{2} > 0$.
                        \begin{align*}
                            \iota(q_{1}q_{2}) & = \iota(-(-q_{1})q_{2}) = -\iota((-q_{1})q_{2})                                                                  \\
                                              & = -\iota(-q_{1})\iota(q_{2})                    & \text{(theo \textbf{Trường hợp 2})}                            \\
                                              & = (-\iota(-q_{1}))\iota(q_{2})                                                                                   \\
                                              & = \iota(q_{1})\iota(q_{2})                      & \text{(theo Mệnh đề~\ref{proposition:dedekind-cuts-and-sign})}
                        \end{align*}

                        Như vậy $\iota(q_{1}q_{2}) = \iota(q_{1})\iota(q_{2})$ với mọi số hữu tỉ âm $q_{1}$ và số hữu tỉ dương $q_{2}$.
                  \item  $q_{1} < 0$ và $q_{2} < 0$.
                        \begin{align*}
                            \iota(q_{1}q_{2}) & = \iota((-q_{1})(-q_{2}))                                                                          \\
                                              & = \iota(-q_{1})\iota(-q_{2})      & \text{(theo \textbf{Trường hợp 2})}                            \\
                                              & = (-\iota(q_{1}))(-\iota(-q_{2}))                                                                  \\
                                              & = \iota(q_{1})\iota(q_{2})        & \text{(theo Mệnh đề~\ref{proposition:dedekind-cuts-and-sign})}
                        \end{align*}

                        Như vậy $\iota(q_{1}q_{2}) = \iota(q_{1})\iota(q_{2})$ với mọi số hữu tỉ âm $q_{1}, q_{2}$.
              \end{enumerate}

              Vậy $\iota(q_{1}q_{2}) = \iota(q_{1})\iota(q_{2})$ với mọi số hữu tỉ $q_{1}, q_{2}$.
        \item Nếu $q_{1}\leq q_{2}$ thì mỗi phần tử $q$ của $\iota(q_{1})$ thỏa mãn $q < q_{1}\leq q_{2}$, tức là $q$ cũng thuộc $\iota(q_{2})$. Do đó, $q_{1}\leq q_{2}$ kéo theo $\iota(q_{1})\subseteq \iota(q_{2})$, $\iota(q_{1})\leq \iota(q_{2})$.
    \end{enumerate}
\end{proof}

Định lý trên được hiểu là đơn ánh $\iota$ bảo toàn phép cộng, phép nhân, và quan hệ thứ tự. Bên cạnh đó, cùng với việc $\iota$ ánh xạ một số hữu tỉ $q$ thành một lát cắt $q^{*} = \{ x \mid x\in\mathbb{Q}\wedge x < q \}$, các lát cắt $q^{*}$ có những tính chất mà số hữu tỉ có.

\section{Dãy Cauchy hữu tỉ}

Một cách khác để xây dựng tập hợp số thực là sử dụng dãy Cauchy hữu tỉ. Cách tiếp cận này được đề xuất lần đầu bởi hai nhà toán học Charles M\'{e}ray và Georg Cantor, một cách độc lập với nhau. Về sau, Richard Dedekind trích dẫn bài báo của Georg Cantor và giới thiệu khái niệm lát cắt. Ý tưởng xây dựng tập hợp số thực bằng dãy Cauchy xuất phát từ kết quả rằng ``Mọi dãy Cauchy (thực) đều hội tụ đến một số thực.'' Tuy nhiên, vì chúng ta muốn xây dựng tập hợp số thực từ các số hữu tỉ nên dãy Cauchy thực không thể là một xuất phát điểm. Thay vào đó, chúng ta sử dụng dãy Cauchy hữu tỉ và xem như chưa biết đến số thực.

\subsection{Tóm tắt}

\subsection{Dãy số hữu ti và dãy Cauchy hữu tỉ}

Bạn đọc có thể đã được nghe qua, hoặc học về khái niệm dãy số (dãy số vô hạn). Khái niệm dãy số có thể được mô tả một cách trực giác: Một dãy số là một danh sách số và \textit{mỗi số tự nhiên được gán với đúng một số trong danh sách}. Dãy số tự nhiên $0, 1, 2, \ldots$, dãy số lẻ $1, 3, 5, \ldots$ là những ví dụ về dãy số. Tuy nhiên, để tuân thủ tiêu chuẩn của toán học hiện đại, các khái niệm cần được định nghĩa hình thức. Dãy số được định nghĩa như một ánh xạ, như trong định nghĩa sau đây.

\begin{definition}[Dãy số hữu tỉ\index{Dãy số hữu tỉ}]
    Một \textbf{dãy số hữu tỉ} là một ánh xạ với tập nguồn là tập hợp số tự nhiên và tập đích là tập hợp số hữu tỉ.

    \noindent Một dãy số hữu tỉ $f: \mathbb{N}\to\mathbb{Q}$ được kí hiệu là ${(f_{n})}_{n\in\mathbb{N}}$, và $f_{n}$ là giá trị được gán với số tự nhiên $n$ bởi $f$. Ngoài cách kí hiệu trên, nhiều tác giả còn dùng các kí hiệu khác cho dãy số, chẳng hạn
    \[
        {(f_{n})}, \quad {(f_{n})}_{n=0}, \quad {(f_{n})}^{\infty}_{n=0}
    \]

    \noindent Trong dãy số hữu tỉ ${(f_{n})}_{n\in\mathbb{N}}$, một số tự nhiên $n$ cụ thể được gọi là một \textbf{chỉ số\index{Chỉ số}}.
\end{definition}

Giống như việc người ta vẫn hay tranh cãi $0$ có phải một số tự nhiên không, có những tài liệu định nghĩa dãy số bắt đầu bằng chỉ số $1$ thay vì $0$. Nhưng đây cũng chỉ là vấn đề quy ước, và giống như phương pháp quy nạp toán học, dãy số có thể bắt đầu bằng bất cứ chỉ số $n_{0}$ nào, với $n_{0}$ là một số nguyên. Tuy nhiên, để thuận theo tinh thần của tài liệu này, chúng ta sẽ luôn để dãy số bắt đầu với chỉ số $0$.

Chúng ta theo dõi một số ví dụ về dãy số hữu tỉ và định nghĩa một số kiểu dãy số đặc biệt.
\begin{example}
    Dãy số Fibonacci ${(F_{n})}_{n\in\mathbb{N}}$ được định nghĩa bằng quy nạp
    \[
        F_{n} = \begin{cases}
            0                 & \text{khi $n = 0$}, \\
            1                 & \text{khi $n = 1$}, \\
            F_{n-1} + F_{n-2} & \text{khi $n > 1$}.
        \end{cases}
    \]
\end{example}

\begin{example}[Dãy hằng số\index{Dãy hằng số}]
    ${(a_{n})}_{n\in\mathbb{N}}$ được định nghĩa $a_{n} = 0$ với mọi số tự nhiên $n$ là một dãy số. Đây được gọi là một dãy hằng số, vì giá trị của dãy số tại mọi số tự nhiên $n$ là bằng nhau.
\end{example}

\begin{example}[Dãy dừng\index{Dãy dừng}]
    ${(a_{n})}_{n\in\mathbb{N}}$ được định nghĩa bởi $a_{0} = 1$, $a_{1} = -1$, $a_{n} = 0$ với mọi số tự nhiên $n$ lớn hơn $1$ là một dãy số. Đây được gọi là một dãy dừng, vì giá trị của dãy số này tại mọi số tự nhiên $n$ là bằng nhau, bắt đầu từ một chỉ số nào đó (trong ví dụ này, chỉ số đó là $2$).
\end{example}

\begin{example}[Dãy đơn điệu\index{Dãy đơn điệu}]
    \begin{itemize}
        \item ${(a_{n})}_{n\in\mathbb{N}}$ được gọi là một dãy số tăng\index{Dãy số tăng thực sự} (đơn điệu tăng, tăng thực sự) khi và chỉ khi $a_{n+1} > a_{n}$ kể từ một chỉ số $n = n_{0}$ nào đó trở đi.
        \item ${(a_{n})}_{n\in\mathbb{N}}$ được gọi là một dãy số giảm\index{Dãy số giảm thực sự} (đơn điệu giảm, giảm thực sự) khi và chỉ khi $a_{n+1} < a_{n}$ kể từ một chỉ số $n = n_{0}$ nào đó trở đi.
        \item ${(a_{n})}_{n\in\mathbb{N}}$ được gọi là một dãy số không giảm\index{Dãy số không giảm} (đơn điệu không giảm) khi và chỉ khi $a_{n+1}\geq a_{n}$ kể từ một chỉ số $n = n_{0}$ nào đó trở đi.
        \item ${(a_{n})}_{n\in\mathbb{N}}$ được gọi là một dãy số không tăng\index{Dãy số không tăng} (đơn điệu không tăng) khi và chỉ khi $a_{n+1}\leq a_{n}$ kể từ một chỉ số $n = n_{0}$ nào đó trở đi.
    \end{itemize}
\end{example}

Khi có một dãy số, người ta thường quan tâm đến việc giá trị của dãy số sẽ như thế nào với chỉ số $n$ rất lớn, hay dãy số đó có hội tụ không. Để phát biểu một cách chặt chẽ về đặc điểm đó của dãy số, các nhà toán học đã đúc kết lại thành định nghĩa dãy số hội tụ. Trước khi đưa ra định nghĩa dãy số hữu tỉ hội tụ, chúng ta cần định nghĩa giá trị tuyệt đối của số hữu tỉ.
\begin{definition}[Giá trị tuyệt đối của số hữu tỉ\index{Giá trị tuyệt đối của số hữu tỉ}]
    Ánh xạ $\abs{\cdot}: \mathbb{Q}\to \mathbb{Q}_{\geq 0}$ được định nghĩa bởi
    \[
        \abs{x} = \begin{cases}
            x  & \text{nếu $x\geq 0$}, \\
            -x & \text{nếu $x < 0$}
        \end{cases}
    \]

    được gọi là \textbf{ánh xạ giá trị tuyệt đối}, hay \textbf{hàm giá trị tuyệt đối} của số hữu tỉ\index{Giá trị tuyệt đối của số hữu tỉ}. Số hữu tỉ không âm $\abs{x}$ được gọi là giá trị tuyệt đối của số hữu tỉ $x$.
\end{definition}

Giá trị tuyệt đối của số hữu tỉ cũng có các tính chất tương tự như giá trị tuyệt đối của số nguyên.
\begin{theorem}
    \begin{enumerate}[label={(\roman*)}]
        \item Với mọi số hữu tỉ $x$, có $0 \leq \abs{x}$. Bên cạnh đó, $\abs{x} = 0$ khi và chỉ khi $x = 0$.
        \item Với mọi số hữu tỉ $x$, có $-\abs{x}\leq x\leq \abs{x}$.
        \item Với mọi số hữu tỉ $x, y$, có $\abs{xy} = \abs{x}\abs{y}$.
        \item Với mọi số hữu tỉ $x, y$, có $\abs{x+y}\leq \abs{x} + \abs{y}$.
    \end{enumerate}
\end{theorem}

\begin{definition}
    Dãy số hữu tỉ ${(a_{n})}_{n\in\mathbb{N}}$ được gọi là
    \textbf{hội tụ đến số hữu tỉ $a$} nếu và chỉ nếu
    \[
        \forall \varepsilon > 0 \Biggl(\exists N(\varepsilon)\Bigl(\forall n\geq N(\varepsilon)(\abs{a_{n} - a} < \varepsilon)\bigr)\biggr).
    \]

    Dãy số hữu tỉ ${(a_{n})}_{n\in\mathbb{N}}$ được gọi là \textbf{hội tụ\index{Hội tụ}} nếu tồn tại số hữu tỉ $a$ sao cho dãy số hữu tỉ ${(a_{n})}_{n\in\mathbb{N}}$ hội tụ đến $a$. Số hữu tỉ $a$ khi đó được gọi là một \textbf{điểm giới hạn\index{Điểm giới hạn}}, hay \textbf{giới hạn\index{Giới hạn}} của dãy số hữu tỉ ${(a_{n})}_{n\in\mathbb{N}}$.

    Dãy số hữu tỉ ${(a_{n})}_{n\in\mathbb{N}}$ được gọi là \textbf{phân kì\index{Phân kì}} nếu dãy này không hội tụ.

    Bằng kí hiệu, chúng ta viết điều kiện cần và đủ để dãy số hữu tỉ ${(a_{n})}_{n\in\mathbb{N}}$ \textbf{không hội tụ đến số hữu tỉ $a$} như sau
    \[
        \exists\varepsilon_{0} > 0 \Biggl(\forall N\Bigl(\exists n\geq N (\abs{a_{n} - a}\geq\varepsilon_{0} )\Bigr)\Biggr).
    \]
\end{definition}

Trên đây là một định nghĩa hình thức cho khái niệm dãy số hữu tỉ hội tụ. Chúng tôi thừa nhận rằng cách định nghĩa này khó hiểu với những người mới học. Nếu bạn đọc cảm thấy đây là một cách định nghĩa phức tạp, thì chúng tôi cho rằng việc nên làm đầu tiên là đọc về vị từ và lượng từ trong Chương~\ref{chapter:logic-and-set-theory}. Thay cho (nhưng không hoàn toàn thay thế) cách định nghĩa trên, định nghĩa cho dãy số hữu tỉ hội tụ có thể được phát biểu bớt hình thức hơn như sau:
\begin{itemize}
    \item (Thông dịch trực tiếp logic hình thức thành câu văn) Dãy số hữu tỉ ${(a_{n})}_{n\in\mathbb{N}}$ được gọi là hội tụ đến số hữu tỉ $a$ nếu và chỉ nếu: với mỗi (số hữu tỉ) $\varepsilon > 0$, tồn tại số tự nhiên $N(\varepsilon)$ chỉ phụ thuộc vào $\varepsilon$ sao cho với mọi chỉ số $n\geq N(\varepsilon)$, chúng ta có $\abs{a_{n} - a} < \varepsilon$.
    \item (Phát biểu không hình thức) Dãy số hữu tỉ ${(a_{n})}_{n\in\mathbb{N}}$ được gọi là hội tụ đến số hữu tỉ $a$ nếu và chỉ nếu: với mỗi (số hữu tỉ) $\varepsilon > 0$ nhỏ tùy ý, luôn tồn tại một chỉ số mà từ chỉ số đó trở đi, khoảng cách từ $a_{n}$ đến $a$ nhỏ hơn $\varepsilon$.
\end{itemize}

Sau đây chúng tôi đưa ra một số ví dụ và phản ví dụ nhằm minh họa khái niệm dãy số hữu tỉ hội tụ và cách chứng minh hay bác bỏ sự hội tụ của một dãy số hữu tỉ.
\begin{example}
    Mọi dãy số hữu tỉ dừng đều hội tụ.

    Một cách tổng quát, chúng ta xét dãy hữu tỉ dừng ${(a_{n})}_{n\in\mathbb{N}}$ thoả mãn $a_{n} = a$ ($a$ là một số hữu tỉ) với mọi số tự nhiên $n\geq n_{0}$. Với định nghĩa này, chúng ta suy ra $\abs{a_{n} - a} = 0$ với mọi số tự nhiên $n\geq n_{0}$. Bên cạnh đó
    \[
        \forall\varepsilon > 0\Bigl(\exists N=n_{0}\bigl(\forall n\geq N( \abs{a_{n} - a} < \varepsilon )\bigr)\Bigr).
    \]

    Do đó dãy số hữu tỉ  ${(a_{n})}_{n\in\mathbb{N}}$ hội tụ đến số hữu tỉ $a$.

    Dãy số hữu tỉ hằng số là trường hợp riêng của dãy số hữu tỉ dừng nên mọi dãy hữu tỉ hằng số đều hội tụ.
\end{example}

\begin{counterexample}
    Dãy số tự nhiên không hội tụ đến bất kì số hữu tỉ nào.

    Giả sử phản chứng rằng dãy số tự nhiên hội tụ đến một số hữu tỉ $q$ nào đó.

    Chúng ta chọn $\varepsilon_{0} = \frac{1}{2}$. Theo định nghĩa dãy số hữu tỉ hội tụ, tồn tại số tự nhiên $N$ sao cho với mọi $n\geq N$, chúng ta có $\abs{n - q} < \frac{1}{2}$. Tiếp tục sử dụng số tự nhiên $N$ và số tự nhiên $n\geq N$, chúng ta có
    \[
        1 = \abs{(n + 1) - n} = \abs{(n+1) - q + (q - n)} \leq \abs{n+1 - q} + \abs{n-q} < \frac{1}{2} + \frac{1}{2} = 1
    \]

    là một mâu thuẫn. Do đó giả sử phản chứng là sai, kéo theo dãy số tự nhiên không hội tụ đến bất kì số hữu tỉ nào.
\end{counterexample}

\begin{example}
    Dãy số ${(a_{n})}_{n\in\mathbb{N}}$ được định nghĩa bởi $a_{n} = \frac{n}{n+1}$ hội tụ đến $1$.

    Chọn một số hữu tỉ $\varepsilon > 0$, chúng ta sẽ tìm số tự nhiên $N(\varepsilon)$ để sử dụng định nghĩa dãy số hữu tỉ hội tụ.

    $\abs{a_{n} - 1} < \varepsilon$ khi và chỉ khi $\frac{1}{n+1} < \varepsilon$. Bên cạnh đó, nếu $n \geq \floor{\frac{1}{\varepsilon}}$ (đây là kí hiệu phần nguyên, được định nghĩa ở Mệnh đề~\ref{proposition:integral-part-of-rational-numbers}) thì
    \[
        \frac{1}{n+1}\leq \frac{1}{\floor{\dfrac{1}{\varepsilon}} + 1} < \frac{1}{\dfrac{1}{\varepsilon}} = \varepsilon.
    \]

    Như vậy
    \[
        \forall\varepsilon > 0 \Biggl( \exists N=\floor{\frac{1}{\varepsilon}}\bigl(\forall n\geq N ( \abs{a_{n} - 1} < \varepsilon )\bigr) \Biggr).
    \]

    Theo định nghĩa dãy số hữu tỉ hội tụ, chúng ta kết luận dãy số ${(a_{n})}_{n\in\mathbb{N}}$ hội tụ đến $1$.
\end{example}

\begin{counterexample}
    Dãy số ${(b_{n})}_{n\in\mathbb{N}}$ được định nghĩa bởi $b_{n} = {(-1)}^{n}$ (với mọi số tự nhiên $n$) không hội tụ đến số hữu tỉ nào.

    Giả sử phản chứng rằng dãy số ${(b_{n})}_{n\in\mathbb{N}}$ hội tụ đến một số hữu tỉ $q$. Chúng ta chọn $\varepsilon = 1$. Theo định nghĩa dãy số hữu tỉ hội tụ, tồn tại số tự nhiên $N$ sao cho với mọi số tự nhiên $n\geq N$, chúng ta có $\abs{b_{n} - q} < 1$. Vẫn là với số $\varepsilon = 1$, số tự nhiên $N$ và $n\ge N$, chúng ta có
    \[
        2 = \abs{b_{n} - b_{n}} = \abs{(b_{n+1} - q) + (q - b_{n})} \leq \abs{b_{n+1} - q} + \abs{b_{n} - q} < 1 + 1 = 2
    \]

    là một mâu thuẫn. Do đó giả sử phản chứng là sai, kéo theo dãy số ${(b_{n})}_{n\in\mathbb{N}}$ không hội tụ đến bất kì số hữu tỉ nào.
\end{counterexample}

\begin{theorem}\label{theorem:uniqueness-of-limit-points-of-convergence-rational-sequences}
    Nếu một dãy số hữu tỉ hội tụ thì dãy số hữu tỉ đó chỉ có một điểm giới hạn.
\end{theorem}

\begin{proof}
    Giả sử phản chứng rằng dãy số hữu tỉ ${(a_{n})}_{n\in\mathbb{N}}$ hội tụ đến hai số hữu tỉ khác nhau là $a$ và $b$. Chọn $\varepsilon = \abs{a - b}$.

    Theo định nghĩa dãy số hữu tỉ hội tụ
    \begin{itemize}[topsep=0pt]
        \item Tồn tại số tự nhiên $N_{a}$ sao cho với mọi số tự nhiên $n\geq N_{a}$, chúng ta có $\abs{a_{n} - a} < \dfrac{\varepsilon}{2}$.
        \item Tồn tại số tự nhiên $N_{b}$ sao cho với mọi số tự nhiên $n\geq N_{b}$, chúng ta có $\abs{a_{n} - b} < \dfrac{\varepsilon}{2}$.
    \end{itemize}

    Chúng ta chọn $N$ là số tự nhiên lớn nhất trong hai số $N_{a}$ và $N_{b}$ (nói cách khác, $N = \max\{ N_{a}, N_{b} \}$). Nếu số tự nhiên $n\geq N$ thì $\abs{a_{n} - a} < \dfrac{\varepsilon}{2}$ và $\abs{a_{n} - b} < \dfrac{\varepsilon}{2}$, từ đó chúng ta có
    \[
        \abs{a - b} = \abs{(a - a_{n}) + (a_{n} - b)} \leq \abs{a_{n} - a} + \abs{a_{n} - b} < \frac{\varepsilon}{2} + \frac{\varepsilon}{2} = \varepsilon = \abs{a - b}
    \]

    là một điều vô lí. Do đó giả sử phản chứng là sai. Vậy nếu dãy số hữu tỉ ${(a_{n})}_{n\in\mathbb{N}}$ hội tụ đến một số hữu tỉ thì đó là số hữu tỉ duy nhất mà ${(a_{n})}_{n\in\mathbb{N}}$ hội tụ đến.
\end{proof}

Nếu dãy số hữu tỉ ${(a_{n})}_{n\in\mathbb{N}}$ hội tụ đến số hữu tỉ $a$ thì chúng ta kí hiệu $\lim a_{n} = a$, hoặc $\lim\limits_{n} a_{n} = a$ khi cần nhấn mạnh chỉ số.

Đối với dãy số, chúng ta cũng có khái niệm dãy số bị chặn.
\begin{definition}[Dãy số hữu tỉ bị chặn]
    Dãy số hữu tỉ ${(a_{n})}_{n\in\mathbb{N}}$ được gọi là \textbf{bị chặn\index{Dãy số hữu tỉ bị chặn}} nếu và chỉ nếu tồn tại số hữu tỉ dương $A$ sao cho $\abs{a_{n}}\leq A$ (hoặc $\abs{a_{n}} < A$) với mọi số tự nhiên $n$.
\end{definition}

Một cách trực giác, việc một dãy số hữu tỉ hội tụ đến một số hữu tỉ $a$ có thể được mô tả như sau: Từ một chỉ số $n$ nào đó trở đi, các giá trị của dãy số sẽ chụm quanh điểm giới hạn. Mô tả này cho thấy sẽ thật không hợp lý nếu như ``khoảng cách'' từ một giá trị nào đó của dãy số đến $a$ có thể lớn tùy ý. Chúng ta có kết quả sau đây.
\begin{theorem}\label{theorem:convergence-sequences-are-bounded}
    Nếu dãy số hữu tỉ ${(a_{n})}_{n\in\mathbb{N}}$ hội tụ thì ${(a_{n})}_{n\in\mathbb{N}}$ là một dãy số hữu tỉ bị chặn.
\end{theorem}

\begin{proof}
    Giả sử dãy số hữu tỉ ${(a_{n})}_{n\in\mathbb{N}}$ hội tụ đến một số hữu tỉ $a$ nào đó.

    Chúng ta chọn $\varepsilon = 1$. Theo định nghĩa dãy số hữu tỉ hội tụ, tồn tại số tự nhiên $N$ nào đó sao cho với mọi số tự nhiên $n\geq N$, chúng ta có $\abs{a_{n} - a} < 1$. Với mọi số tự nhiên $n\geq N$, chúng ta có
    \[
        \abs{a_{n}} = \abs{(a_{n} - a) + a} \leq \abs{a_{n} - a} + \abs{a} < 1 + \abs{a}.
    \]

    Theo nguyên lý cực hạn, tập hợp $\{ \abs{a_{0}}, \abs{a_{1}}, \ldots, \abs{a_{N-1}}, 1 + \abs{a} \}$ có phần tử lớn nhất. Chúng ta kí hiệu phần tử đó là $A$. $A$ là một số hữu tỉ dương vì các phần tử trong tập hợp trên đều là số hữu tỉ, và $1 + \abs{a} > 0$. Do đó, $\abs{a_{n}}\leq A$ với mọi số tự nhiên $n$, đồng nghĩa với việc dãy số hữu tỉ ${(a_{n})}_{n\in\mathbb{N}}$ bị chặn.
\end{proof}

Việc tính toán giới hạn của một số dãy số có thể được đơn giản hóa nhờ kết quả sau và những giới hạn đã biết.
\begin{proposition}\label{proposition:limits-of-sum-and-product}
    ${(a_{n})}_{n\in\mathbb{N}}$ và ${(a_{n})}_{n\in\mathbb{N}}$ là các dãy số hữu tỉ.
    \begin{enumerate}[label={(\roman*)}]
        \item Nếu $\lim a_{n} = a$ và $\lim b_{n} = b$ thì $\lim s_{n} = a + b$, trong đó dãy số hữu tỉ ${(s_{n})}_{n\in\mathbb{N}}$ được định nghĩa bởi $s_{n} = a_{n} + b_{n}$ với mọi số tự nhiên $n$ (Chúng ta còn kí hiệu ${(s_{n})}_{n\in\mathbb{N}}$ bởi ${(a_{n} + b_{n})}_{n\in\mathbb{N}}$).
        \item Nếu $\lim a_{n} = a$ thì $\lim c_{n} = ca$, trong đó dãy số hữu tỉ ${(c_{n})}_{n\in\mathbb{N}}$ được định nghĩa bởi $c_{n} = c\cdot a_{n}$ với mọi số tự nhiên $n$, trong đó $c$ là một hằng số hữu tỉ (Chúng ta còn kí hiệu ${(c_{n})}_{n\in\mathbb{N}}$ bởi ${(ca_{n})}_{n\in\mathbb{N}}$).
        \item Nếu $\lim a_{n} = a$ và $\lim b_{n} = b$ thì $\lim p_{n} = ab$, trong đó dãy số hữu tỉ ${(p_{n})}_{n\in\mathbb{N}}$ được định nghĩa bởi $p_{n} = a_{n}b_{n}$ với mọi số tự nhiên $n$ (Chúng ta còn kí hiệu ${(p_{n})}_{n\in\mathbb{N}}$ bởi ${(a_{n}b_{n})}_{n\in\mathbb{N}}$).
    \end{enumerate}
\end{proposition}

\begin{proof}
    \begin{enumerate}[label={(\roman*)}]
        \item Chúng ta lấy $\varepsilon$ là một số hữu tỉ dương bất kì.

              Theo định nghĩa dãy số hữu tỉ hội tụ, với số hữu tỉ dương $\dfrac{\varepsilon}{2}$
              \begin{itemize}
                  \item tồn tại số tự nhiên $N_{a}$ sao cho với mọi số tự nhiên $n\geq N_{a}$, có $\abs{a_{n} - a} < \dfrac{\varepsilon}{2}$
                  \item tồn tại số tự nhiên $N_{b}$ sao cho với mọi số tự nhiên $n\geq N_{b}$, có $\abs{b_{n} - b} < \dfrac{\varepsilon}{2}$.
              \end{itemize}

              Chúng ta định nghĩa $N = \max\{ N_{a}, N_{b} \}$. Nếu số tự nhiên $n\geq N$ thì $\abs{a_{n} - a} < \dfrac{\varepsilon}{2}$ và $\abs{b_{n} - b} < \dfrac{\varepsilon}{2}$. Khi đó chúng ta có
              \[
                  \abs{(a_{n} + b_{n}) - (a + b)}  = \abs{(a_{n} - a) + (b_{n} - b)} \leq \abs{a_{n} - a} + \abs{b_{n} - b} < \frac{\varepsilon}{2} + \frac{\varepsilon}{2} = \varepsilon.
              \]

              Những điều trên có nghĩa là: với mọi số hữu tỉ dương $\varepsilon$, tồn tại số tự nhiên $N$ sao cho với mọi số tự nhiên $n\geq N$, chúng ta có $\abs{(a_{n} + b_{n}) - (a + b)} < \varepsilon$. Theo định nghĩa dãy số hữu tỉ hội tụ, chúng ta suy ra $\lim s_{n} = a + b$.
        \item Nếu $c = 0$ thì ${(c_{n})}_{n\in\mathbb{N}}$ là một dãy hằng số, vì $c_{n} = c\cdot a_{n} = 0$ với mọi số tự nhiên $n$. Khi đó $\lim c_{n} = 0 = ca$.

              Nếu $c\ne 0$, chúng ta lấy $\varepsilon$ là một số hữu tỉ dương bất kì. Theo định nghĩa dãy số hữu tỉ hội tụ, với số hữu tỉ dương $\dfrac{\varepsilon}{\abs{c}}$, tồn tại số tự nhiên $N$ sao cho với mọi số tự nhiên $n\geq N$, có $\abs{a_{n} - a} < \dfrac{\varepsilon}{\abs{c}}$. Cũng là số tự nhiên $N$, nếu số tự nhiên $n\geq N$ thì chúng ta có
              \[
                  \abs{ca_{n} - ca} = \abs{c(a_{n} - a)} = \abs{c}\abs{a_{n} - a} \leq \abs{c}\cdot\frac{\varepsilon}{\abs{c}} = \varepsilon.
              \]

              Theo định nghĩa dãy số hữu tỉ hội tụ, chúng ta kết luận $\lim c_{n} = ca$.
        \item Chúng ta có
              \begin{align*}
                  \abs{a_{n}b_{n} - ab} & = \abs{(a_{n} - a)b_{n} + a(b_{n} - b)}                  \\
                                        & \leq \abs{a_{n} - a}\abs{b_{n}} + \abs{a}\abs{b_{n} - b}
              \end{align*}

              Dựa trên bất đẳng thức vừa thu được, chúng ta tìm số tự nhiên ``$N$'' để có thể áp dụng được định nghĩa dãy số hữu tỉ hội tụ. Vì ${(a_{n})}_{n\in\mathbb{N}}$ hội tụ đến $a$ và ${(b_{n})}_{n\in\mathbb{N}}$ hội tụ đến $b$ nên hai dãy số này bị chặn, kéo theo tồn tại hai số hữu tỉ dương $A, B$ sao cho $\abs{a_{n}}\leq A$ và $\abs{b_{n}}\leq B$ với mọi số tự nhiên $n$.

              Chúng ta chọn $\varepsilon$ là một số hữu tỉ dương bất kì. Theo định nghĩa dãy số hữu tỉ hội tụ
              \begin{itemize}
                  \item với số hữu tỉ dương $\dfrac{\varepsilon}{2B}$, tồn tại số tự nhiên $N_{a}$ sao cho với mọi số tự nhiên $n\geq N_{a}$, chúng ta có $\abs{a_{n} - a} < \dfrac{\varepsilon}{2B}$
                  \item với số hữu tỉ dương $\dfrac{\varepsilon}{2A}$, tồn tại số tự nhiên $N_{b}$ sao cho với mọi số tự nhiên $n\geq N_{b}$, chúng ta có $\abs{b_{n} - b} < \dfrac{\varepsilon}{2A}$.
              \end{itemize}

              Chúng ta định nghĩa $N = \max\{ N_{a}, N_{b}\}$. Nếu số tự nhiên $n\geq N$ thì $\abs{a_{n} - a} < \dfrac{\varepsilon}{2B}$ và $\abs{b_{n} - b} < \dfrac{\varepsilon}{2A}$. Khi đó chúng ta có
              \begin{align*}
                  \abs{a_{n}b_{n} - ab} & = \abs{(a_{n} - a)b_{n} + a(b_{n} - b)}                         \\
                                        & \leq \abs{a_{n} - a}\abs{b_{n}} + \abs{a}\abs{b_{n} - b}        \\
                                        & < \frac{\varepsilon}{2B}\cdot B + \frac{\varepsilon}{2A}\cdot A \\
                                        & = \frac{\varepsilon}{2} + \frac{\varepsilon}{2} = \varepsilon.
              \end{align*}

              Như vậy, với mỗi số hữu tỉ dương $\varepsilon$, tồn tại số tự nhiên $N$ sao cho với mọi số tự nhiên $n\geq N$, chúng ta có $\abs{a_{n}b_{n} - ab} < \varepsilon$. Theo định nghĩa giới hạn dãy số hữu tỉ, chúng ta kết luận $\lim p_{n} = ab$.
    \end{enumerate}
\end{proof}

Đến lúc này, chúng ta đã chuẩn bị đủ để định nghĩa và chứng minh các kết quả về dãy Cauchy hữu tỉ.

\begin{definition}[Dãy Cauchy hữu tỉ]
    Dãy số hữu tỉ ${(a_{n})}_{n\in\mathbb{N}}$ được gọi là một \textbf{dãy Cauchy hữu tỉ\index{Dãy Cauchy hữu tỉ}} nếu và chỉ nếu
    \[
        \forall\varepsilon > 0 \Biggl( \exists N(\varepsilon) \Bigl( \forall n\geq N(\varepsilon) \bigl(\forall m\geq N(\varepsilon) (\abs{a_{m} - a_{n}} < \varepsilon)\bigr) \Bigr) \Biggr).
    \]

    \noindent Điều kiện trên có thể phát biểu dưới dạng tương đương là
    \[
        \forall\varepsilon > 0 \Biggl( \exists N(\varepsilon) \Bigl( \forall n\geq N(\varepsilon) \bigl(\forall p > 0 (\abs{a_{n+p} - a_{n}} < \varepsilon)\bigr) \Bigr) \Biggr).
    \]

    \noindent Chúng ta kí hiệu tập hợp các dãy Cauchy hữu tỉ là $\mathscr{C}_{\mathbb{Q}}$.
\end{definition}

Chúng ta cũng có thể phát biểu định nghĩa dãy Cauchy hữu tỉ theo cách bớt hình thức hơn.
\begin{itemize}
    \item (Thông dịch trực tiếp từ định nghĩa hình thức) Dãy số hữu tỉ ${(a_{n})}_{n\in\mathbb{N}}$ được gọi là một dãy Cauchy hữu tỉ nếu và chỉ nếu: Với mọi số hữu tỉ dương $\varepsilon$, tồn tại số tự nhiên $N(\varepsilon)$ sao cho với mọi số tự nhiên $n, m\geq N(\varepsilon)$, chúng ta có $\abs{a_{m} - a_{n}} < \varepsilon$.
    \item (Mô tả trực giác) Dãy số hữu tỉ ${(a_{n})}_{n\in\mathbb{N}}$ được gọi là một dãy Cauchy hữu tỉ nếu và chỉ nếu: Với mọi số hữu tỉ dương $\varepsilon$ nhỏ tùy ý, từ một chỉ số nào đó trở đi, khoảng cách giữa các giá trị của dãy số nhỏ hơn $\varepsilon$.
\end{itemize}

Dãy số hữu tỉ hội tụ là một trường hợp riêng của dãy Cauchy hữu tỉ.
\begin{theorem}
    Nếu một dãy số hữu tỉ hội tụ thì đó cũng là một dãy Cauchy hữu tỉ.
\end{theorem}

\begin{proof}
    Giả sử dãy số hữu tỉ ${(a_{n})}_{n\in\mathbb{N}}$ hội tụ đến số hữu tỉ $a$. Chúng ta chọn $\varepsilon$ là một số hữu tỉ dương bất kì.

    Theo định nghĩa dãy số hữu tỉ hội tụ, tồn tại số tự nhiên $N$ sao cho với mọi số tự nhiên $n\geq N$, chúng ta có $\abs{a_{n} - a} < \dfrac{\varepsilon}{2}$. Vẫn là số tự nhiên $N$, nếu các số tự nhiên $n, m$ thỏa mãn $n, m\geq N$ thì chúng ta có
    \[
        \abs{a_{m} - a_{n}} = \abs{(a_{m} - a) + (a - a_{n})} \leq \abs{a_{m} - a} + \abs{a_{n} - a} < \frac{\varepsilon}{2} + \frac{\varepsilon}{2} = \varepsilon.
    \]

    Như vậy, với mỗi số hữu tỉ dương $\varepsilon$, tồn tại số tự nhiên $N$ sao cho với mọi số tự nhiên $m, n\geq N$, chúng ta có $\abs{a_{m} - a_{n}} < \varepsilon$. Do đó ${(a_{n})}_{n\in\mathbb{N}}$ là một dãy Cauchy hữu tỉ.
\end{proof}

Tương tự với dãy số hữu tỉ hội tụ, dãy Cauchy hữu tỉ cũng bị chặn.
\begin{theorem}\label{theorem:cauchy-sequences-are-bounded}
    Nếu ${(a_{n})}_{n\in\mathbb{N}}$ là một dãy Cauchy hữu tỉ thì ${(a_{n})}_{n\in\mathbb{N}}$ là một dãy số hữu tỉ bị chặn.
\end{theorem}

\begin{proof}
    Theo định nghĩa dãy Cauchy hữu tỉ, với số hữu tỉ dương $1$, tồn tại số tự nhiên $N$ sao cho với mọi số tự nhiên $n, m\geq N$, chúng ta có $\abs{a_{m} - a_{n}} < 1$. Do đó $\abs{a_{n} - a_{N}} < 1$ với mọi số tự nhiên $n\geq N$. Bên cạnh đó, với mọi số tự nhiên $n\geq N$, chúng ta có
    \[
        \abs{a_{n}} = \abs{(a_{n} - a_{N}) + a_{N}} \leq \abs{a_{n} - a_{N}} + \abs{a_{N}} < 1 + \abs{a_{N}}.
    \]

    Theo nguyên lý cực hạn, tập hợp $\{ \abs{a_{0}}, \abs{a_{1}}, \ldots, \abs{a_{N-1}}, 1 + \abs{a_{N}} \}$ có phần tử lớn nhất. Chúng ta kí hiệu phần tử này là $A$, $A$ còn là một số hữu tỉ dương, vì $A$ thuộc tập hợp trên (gồm các số hữu tỉ) và $A\geq 1 + \abs{a_{N}}$.

    Với hai điều trên, chúng ta suy ra $\abs{a_{n}}\leq A$ với mọi số tự nhiên $n$, điều này đồng nghĩa với việc ${(a_{n})}_{n\in\mathbb{N}}$ là một dãy số hữu tỉ bị chặn.
\end{proof}

Dãy số hữu tỉ hội tụ thì cũng là dãy Cauchy hữu tỉ nhưng điều ngược lại nói chung không đúng. Mệnh đề dưới đây đưa ra một dãy Cauchy hữu tỉ nhưng không hội tụ đến một số hữu tỉ nào.

\begin{proposition}
    Dãy số hữu tỉ ${(x_{n})}_{n\in\mathbb{N}}$ được định nghĩa bằng đệ quy như sau
    \[
        x_{n} = \begin{cases}
            1                                 & \text{khi $n = 0$}   \\
            \dfrac{2x_{n-1} + 2}{x_{n-1} + 2} & \text{khi $n\geq 1$}
        \end{cases}
    \]

    Chứng minh rằng dãy số hữu tỉ ${(x_{n})}_{n\in\mathbb{N}}$ là một dãy Cauchy và không hội tụ đến bất kì số hữu tỉ nào.
\end{proposition}

\begin{proof}
    Chúng ta chứng minh mệnh đề này bằng phản chứng, lần lượt qua các bước sau.
    \begin{enumerate}[label={\textbf{Bước \arabic*.}},itemindent=1cm]
        \item $x_{n}\geq 1$ với mọi số tự nhiên $n$.

              Khi $n = 0$, chúng ta có $x_{0} = 1$, do đó $x_{0}\geq 1$.

              Giả sử với số tự nhiên $n = k\geq 0$, chúng ta có $x_{k}\geq 1$. Khi đó theo giả thiết quy nạp
              \[
                  x_{k+1} = \frac{2x_{k-1} + 2}{x_{k-1} + 2} = 1 + \frac{x_{k-1}}{x_{k-1} + 2} \geq 1
              \]

              Theo nguyên lý quy nạp toán học, $x_{n}\geq 1$ với mọi số tự nhiên $n$.
        \item Với mọi số tự nhiên $n$, $\abs{{x_{n}}^{2} - 2}\leq \dfrac{1}{n+1}$.

              Khi $n = 0$, chúng ta có $x_{0} = 1$, do đó $\abs{{x_{0}}^{2} - 1} = 1\leq \dfrac{1}{0+1}$.

              Giả sử với số tự nhiên $n = k\geq 0$, chúng ta có $\abs{{x_{k}}^{2} - 2}\leq \dfrac{1}{k+1}$.
              \begin{align*}
                  \abs{{x_{k+1}}^{2} - 2} & = \abs{\frac{4{(x_{k} + 1)}^{2}}{{(x_{k} + 2)}^{2}} - 2} = \abs{\frac{2{x_{k}^{2}} - 4}{{(x_{k} + 2)}^{2}}} = \abs{\frac{2}{{(x_{k} + 2)}^{2}}}\abs{{x_{k}}^{2} - 2}                                   \\
                                          & \leq \frac{1}{2}\abs{{x_{k}}^{2} - 2}                                                                                                                                & \text{(vì $x_{k} > 0$)}         \\
                                          & \leq \frac{1}{2}\cdot\frac{1}{k+1}                                                                                                                                   & \text{(theo giả thiết quy nạp)} \\
                                          & = \frac{1}{2k+2}\leq \frac{1}{k+2}.
              \end{align*}

              Theo nguyên lý quy nạp toán học, $\abs{{x_{n}}^{2} - 2}\leq \dfrac{1}{n+1}$ với mọi số tự nhiên $n$.
        \item Chứng minh rằng ${(x_{n})}_{n\in\mathbb{N}}$ là một dãy Cauchy hữu tỉ.

              Chúng ta chọn $\varepsilon$ là một số hữu tỉ dương, $N = \floor{\frac{1}{\varepsilon}}$, nếu số tự nhiên $n\geq N$ thì với mọi số tự nhiên $p$, chúng ta có
              \begin{align*}
                  \abs{x_{n+p} - x_{n}} & = \frac{\abs{{x_{n+p}}^{2} - {x_{n}}^{2}}}{\abs{x_{n+p} + x_{n}}} = \frac{\abs{({x_{n+p}}^{2} - 2) + (2 - {x_{n}}^{2})}}{\abs{x_{n+p} + x_{n}}}                                                    \\
                                        & \leq \frac{\abs{{x_{n+p}}^{2} - 2} + \abs{{x_{n}}^{2} - 2}}{\abs{x_{n+p} + x_{n}}}                                                                                                                 \\
                                        & \leq \frac{1}{2}\left(\frac{1}{n+1} + \frac{1}{n+p+1}\right)                                                                                    & \text{(theo \textbf{Bước 1} và \textbf{Bước 2})} \\
                                        & \leq \frac{1}{n+1} < \frac{1}{\frac{1}{\varepsilon}} = \varepsilon.
              \end{align*}

              Do đó với mỗi số hữu tỉ dương $\varepsilon$, tồn tại số tự nhiên $N$ sao cho với mọi số tự nhiên $n\geq N$ và với mọi số tự nhiên $p$, chúng ta có $\abs{x_{n+p} - x_{n}} < \varepsilon$. Theo định nghĩa dãy Cauchy hữu tỉ, ${(x_{n})}_{n\in\mathbb{N}}$ là một dãy Cauchy hữu tỉ.
        \item Dãy số hữu tỉ ${(y_{n})}_{n\in\mathbb{N}}$ (được định nghĩa bởi $y_{n} = {x_{n}}^{2}$ với mọi số tự nhiên $n$) hội tụ đến $2$.

              Chúng ta chọn $\varepsilon$ là một số hữu tỉ dương, $N = \floor{\frac{1}{\varepsilon}}$, nếu số tự nhiên $n\geq N$ thì
              \[
                  \abs{{x_{n}}^{2} - 2}\leq \frac{1}{n+1} < \frac{1}{\frac{1}{\varepsilon}} < \varepsilon.
              \]

              Như vậy, với mỗi số hữu tỉ dương $\varepsilon$, tồn tại số tự nhiên $N$ sao cho với mọi số tự nhiên $n\geq N$, chúng ta có $\abs{{x_{n}}^{2} - 2} < \varepsilon$. Theo định nghĩa dãy số hội tụ, dãy số hữu tỉ ${(y_{n})}_{n\in\mathbb{N}}$ hội tụ đến $2$.
        \item Chứng minh rằng dãy số hữu tỉ ${(x_{n})}_{n\in\mathbb{N}}$ không hội tụ đến bất kì số hữu tỉ nào.

              Giả sử phản chứng rằng dãy số hữu tỉ ${(x_{n})}_{n\in\mathbb{N}}$ hội tụ    đến số hữu tỉ $x$.

              Theo phần (iii) của Mệnh đề~\ref{proposition:limits-of-sum-and-product}, chúng ta suy ra dãy số ${(y_{n})}_{n\in\mathbb{N}}$ hội tụ đến số hữu tỉ $x^{2}$. Theo Định lý~
              \ref{theorem:uniqueness-of-limit-points-of-convergence-rational-sequences}, chúng ta suy ra $x^{2} = 2$.

              Theo Mệnh đề~\ref{proposition:irrational-cut}, không có số hữu tỉ nào có bình phương bằng $2$. Do đó $x^{2} = 2$ (với $x$ là số hữu tỉ) là một kết quả vô lý. Như vậy giả sử phản chứng là sai, kéo theo dãy số hữu tỉ ${(x_{n})}_{n\in\mathbb{N}}$ không hội tụ đến số hữu tỉ nào.
    \end{enumerate}
\end{proof}

Khi sử dụng dãy Cauchy hữu tỉ để xây dựng tập hợp số thực, chúng ta gặp một vấn đề: \textit{có nhiều dãy Cauchy hữu tỉ hội tụ đến cùng một số hữu tỉ}. Điều này sẽ được giải quyết bằng một quan hệ tương đương như sau.

\begin{theorem}\label{theorem:equivalent-rational-sequences}
    Hai dãy số hữu tỉ ${(a_{n})}_{n\in\mathbb{N}}$ và ${(b_{n})}_{n\in\mathbb{N}}$ được gọi là có quan hệ $\sim$, và được kí hiệu là ${(a_{n})}_{n\in\mathbb{N}} \sim {(b_{n})}_{n\in\mathbb{N}}$ khi và chỉ khi dãy số hữu tỉ ${(a_{n} - b_{n})}_{n\in\mathbb{N}}$ hội tụ đến $0$, nói cách khác
    \[
        \forall\varepsilon > 0\Biggl( \exists N \Bigl( \forall n\geq N ( \abs{a_{n} - b_{n}} < \varepsilon ) \Bigr) \Biggr).
    \]
    \begin{enumerate}[label={(\roman*)},itemsep=0pt]
        \item Quan hệ $\sim$ trên tập hợp các dãy số hữu tỉ là một quan hệ tương đương. Nói riêng, quan hệ $\sim$ trên tập hợp các dãy Cauchy hữu tỉ là một quan hệ tương đương.
        \item Nếu hai dãy số hữu tỉ ${(a_{n})}_{n\in\mathbb{N}}$ và ${(b_{n})}_{n\in\mathbb{N}}$ tương đương theo quan hệ $\sim$ thì ${(a_{n})}_{n\in\mathbb{N}}$ hội tụ đến số hữu tỉ $q$ khi và chỉ khi ${(b_{n})}_{n\in\mathbb{N}}$ hội tụ đến số hữu tỉ $q$.
        \item Nếu hai dãy số hữu tỉ ${(a_{n})}_{n\in\mathbb{N}}$ và ${(b_{n})}_{n\in\mathbb{N}}$ tương đương theo quan hệ $\sim$ thì ${(a_{n})}_{n\in\mathbb{N}}$ là dãy Cauchy hữu tỉ khi và chỉ khi ${(b_{n})}_{n\in\mathbb{N}}$ là dãy Cauchy hữu tỉ.
    \end{enumerate}
\end{theorem}

\begin{proof}
    \begin{enumerate}[label={(\roman*)},itemsep=0pt]
        \item Dãy số hữu tỉ ${(a_{n} - a_{n})}_{n\in\mathbb{N}}$ là một dãy hữu tỉ hằng số, giá trị của dãy này tại mọi số tự nhiên $n$ bằng $0$, kéo theo dãy số hữu tỉ ${(a_{n} - a_{n})}_{n\in\mathbb{N}}$ hội tụ đến $0$. Do đó quan hệ $\sim$ trên tập hợp các dãy số hữu tỉ có tính chất phản xạ.

              ${(a_{n})}_{n\in\mathbb{N}}\sim {(b_{n})}_{n\in\mathbb{N}}$ khi và chỉ khi
              \[
                  \forall\varepsilon > 0\Biggl( \exists N \Bigl( \forall n\geq N ( \abs{a_{n} - b_{n}} < \varepsilon ) \Bigr) \Biggr).
              \]

              ${(b_{n})}_{n\in\mathbb{N}}\sim {(a_{n})}_{n\in\mathbb{N}}$ khi và chỉ khi
              \[
                  \forall\varepsilon > 0\Biggl( \exists N \Bigl( \forall n\geq N ( \abs{b_{n} - a_{n}} < \varepsilon ) \Bigr) \Biggr).
              \]

              Mặt khác, vì $\abs{a_{n} - b_{n}} = \abs{b_{n} - a_{n}}$ với mọi số tự nhiên $n$ nên hai điều kiện dưới đây tương đương
              \[
                  \forall\varepsilon > 0\Biggl( \exists N \Bigl( \forall n\geq N ( \abs{a_{n} - b_{n}} < \varepsilon ) \Bigr) \Biggr) \quad\longleftrightarrow\quad \forall\varepsilon > 0\Biggl( \exists N \Bigl( \forall n\geq N ( \abs{b_{n} - a_{n}} < \varepsilon ) \Bigr) \Biggr).
              \]

              Do đó ${(a_{n})}_{n\in\mathbb{N}} \sim {(b_{n})}_{n\in\mathbb{N}}$ khi và chỉ khi ${(b_{n})}_{n\in\mathbb{N}} \sim {(a_{n})}_{n\in\mathbb{N}}$, điều này có nghĩa là quan hệ $\sim$ trên tập hợp các dãy số hữu tỉ có tính chất đối xứng.

              Nếu các dãy số hữu tỉ ${(a_{n})}_{n\in\mathbb{N}}, {(b_{n})}_{n\in\mathbb{N}}, {(c_{n})}_{n\in\mathbb{N}}$ thỏa mãn ${(a_{n})}_{n\in\mathbb{N}}\sim {(b_{n})}_{n\in\mathbb{N}}$ và ${(b_{n})}_{n\in\mathbb{N}}\sim {(c_{n})}_{n\in\mathbb{N}}$ thì theo định nghĩa quan hệ $\sim$ trên tập hợp dãy số hữu tỉ, với mỗi số hữu tỉ dương $\varepsilon$
              \begin{itemize}
                  \item Tồn tại số tự nhiên $N_{ab}$ sao cho với mọi số tự nhiên $n\geq N_{ab}$, chúng ta có $\abs{a_{n} - b_{n}} < \dfrac{\varepsilon}{2}$.
                  \item Tồn tại số tự nhiên $N_{bc}$ sao cho với mọi số tự nhiên $n\geq N_{bc}$, chúng ta có $\abs{b_{n} - c_{n}} < \dfrac{\varepsilon}{2}$.
              \end{itemize}

              Chúng ta định nghĩa số tự nhiên $N$ là số tự nhiên lớn nhất trong hai số $N_{ab}, N_{bc}$. Nếu số tự nhiên $n\geq N$ thì $\abs{a_{n} - b_{n}} < \dfrac{\varepsilon}{2}$, $\abs{b_{n} - c_{n}} < \dfrac{\varepsilon}{2}$, và
              \[
                  \abs{a_{n} - c_{n}} = \abs{(a_{n} - b_{n}) + (b_{n} - c_{n})} \leq \abs{a_{n} - b_{n}} + \abs{b_{n} - c_{n}} < \dfrac{\varepsilon}{2} + \dfrac{\varepsilon}{2} = \varepsilon.
              \]

              Từ điều trên, chúng ta suy ra rằng với mỗi số hữu tỉ dương $\varepsilon$, tồn tại số tự nhiên $N$ sao cho với mọi số tự nhiên $n\geq N$, chúng ta có $ \abs{a_{n} - c_{n}} < \varepsilon$. Do đó ${(a_{n})}_{n\in\mathbb{N}} \sim {(c_{n})}_{n\in\mathbb{N}}$, điều này có nghĩa là quan hệ $\sim$ trên tập hợp các dãy số hữu tỉ có tính chất bắc cầu.

              Vậy quan hệ $\sim$ trên tập hợp các dãy số hữu tỉ là một quan hệ tương đương. Vì tập hợp các dãy Cauchy hữu tỉ là tập hợp con của tập hợp các dãy số hữu tỉ nên quan hệ $\sim$ trên tập hợp các dãy Cauchy hữu tỉ cũng là một quan hệ tương đương.
        \item ($\Rightarrow$) ${(a_{n})}_{n\in\mathbb{N}}$ hội tụ đến số hữu tỉ $q$.

              Chúng ta chọn số hữu tỉ dương $\varepsilon$ bất kì. Theo định nghĩa dãy số hữu tỉ hội tụ, tồn tại số tự nhiên $N_{a}$ sao cho với mọi số tự nhiên $n\geq N_{a}$, chúng ta có $\abs{a_{n} - q} < \dfrac{\varepsilon}{2}$. Theo định nghĩa quan hệ $\sim$ trên tập hợp dãy số hữu tỉ, ${(a_{n})}_{n\in\mathbb{N}}\sim {(b_{n})}_{n\in\mathbb{N}}$ kéo theo tồn tại số tự nhiên $N_{ab}$ sao cho với mọi số tự nhiên $n\geq N_{ab}$, chúng ta có $\abs{a_{n} - b_{n}} < \varepsilon$.

              Chúng ta chọn số tự nhiên $N = \max\{ N_{a}, N_{ab} \}$. Khi đó, với mọi số tự nhiên $n\geq N$, chúng ta có
              \[
                  \abs{b_{n} - q} = \abs{(b_{n} - a_{n}) + (a_{n} - q)} \leq \abs{b_{n} - a_{n}} + \abs{a_{n} - q} < \frac{\varepsilon}{2} + \frac{\varepsilon}{2} = \varepsilon
              \]

              Theo định nghĩa dãy số hữu tỉ hội tụ, dãy số hữu tỉ ${(b_{n})}_{n\in\mathbb{N}}$ hội tụ đến số hữu tỉ $q$.

              ($\Rightarrow$) ${(b_{n})}_{n\in\mathbb{N}}$ hội tụ đến số hữu tỉ $q$.

              Hoàn toàn tương tự, chúng ta chọn số hữu tỉ dương $\varepsilon$ bất kì. Theo định nghĩa dãy số hữu tỉ hội tụ, tồn tại số tự nhiên $N_{a}$ sao cho với mọi số tự nhiên $n\geq N_{b}$, chúng ta có $\abs{b_{n} - q} < \dfrac{\varepsilon}{2}$. Theo định nghĩa quan hệ $\sim$ trên tập hợp dãy số hữu tỉ, ${(a_{n})}_{n\in\mathbb{N}}\sim {(b_{n})}_{n\in\mathbb{N}}$ thì tồn tại số tự nhiên $N_{ab}$ sao cho với mọi số tự nhiên $n\geq N_{ab}$, chúng ta có $\abs{a_{n} - b_{n}} < \varepsilon$.

              Chúng ta chọn số tự nhiên $N = \max\{ N_{b}, N_{ab} \}$. Khi đó, với mọi số tự nhiên $n\geq N$, chúng ta có
              \[
                  \abs{a_{n} - q} = \abs{(a_{n} - b_{n}) + (b_{n} - q)} \leq \abs{a_{n} - b_{n}} + \abs{b_{n} - q} < \frac{\varepsilon}{2} + \frac{\varepsilon}{2} = \varepsilon
              \]

              Theo định nghĩa dãy số hữu tỉ hội tụ, dãy số hữu tỉ ${(a_{n})}_{n\in\mathbb{N}}$ hội tụ đến số hữu tỉ $q$.
        \item ($\Rightarrow$) ${(a_{n})}_{n\in\mathbb{N}}$ là dãy Cauchy hữu tỉ.

              Chúng ta chọn số hữu tỉ dương $\varepsilon$ bất kì. Theo định nghĩa dãy Cauchy hữu tỉ, tồn tại số tự nhiên $N_{a}$ sao cho với mọi số tự nhiên $n, m\geq N_{a}$, chúng ta có $\abs{a_{n} - a_{m}} < \dfrac{\varepsilon}{2}$. Theo định nghĩa quan hệ $\sim$ trên tập hợp dãy số hữu tỉ, ${(a_{n})}_{n\in\mathbb{N}}\sim {(b_{n})}_{n\in\mathbb{N}}$ kéo theo tồn tại số tự nhiên $N_{ab}$ sao cho với mọi số tự nhiên $n \geq N_{ab}$, chúng ta có $\abs{a_{n} - b_{n}} < \dfrac{\varepsilon}{4}$.

              Chúng ta chọn số tự nhiên $N = \max\{ N_{a}, N_{ab} \}$. Khi đó, với mọi số tự nhiên $n, m\geq N$, chúng ta có
              \begin{align*}
                  \abs{b_{m} - b_{n}} & = \abs{(b_{m} - a_{m}) + (a_{m} - a_{n}) + (a_{n} - b_{n})}                            \\
                                      & \leq \abs{b_{m} - a_{m}} + \abs{a_{m} - a_{n}} + \abs{a_{n} - b_{n}}                   \\
                                      & < \frac{\varepsilon}{4} + \frac{\varepsilon}{2} + \frac{\varepsilon}{4} = \varepsilon.
              \end{align*}

              Từ điều trên, chúng ta suy ra rằng với mỗi số hữu tỉ dương $\varepsilon$, tồn tại số tự nhiên $N$ sao cho với mọi số tự nhiên $n, m\geq N$, chúng ta có $\abs{b_{m} - b_{n}} < \varepsilon$. Do đó ${(b_{n})}_{n\in\mathbb{N}}$ là một dãy Cauchy hữu tỉ.

              ($\Rightarrow$) ${(b_{n})}_{n\in\mathbb{N}}$ là dãy Cauchy hữu tỉ.

              Chúng ta chọn số hữu tỉ dương $\varepsilon$ bất kì. Theo định nghĩa dãy Cauchy hữu tỉ, tồn tại số tự nhiên $N_{b}$ sao cho với mọi số tự nhiên $n, m\geq N_{b}$, chúng ta có $\abs{b_{n} - b_{m}} < \dfrac{\varepsilon}{2}$. Theo định nghĩa quan hệ $\sim$ trên tập hợp dãy số hữu tỉ, ${(a_{n})}_{n\in\mathbb{N}}\sim {(b_{n})}_{n\in\mathbb{N}}$ kéo theo tồn tại số tự nhiên $N_{ab}$ sao cho với mọi số tự nhiên $n \geq N_{ab}$, chúng ta có $\abs{a_{n} - b_{n}} < \dfrac{\varepsilon}{4}$.

              Chúng ta chọn số tự nhiên $N = \max\{ N_{b}, N_{ab} \}$. Khi đó, với mọi số tự nhiên $n, m\geq N$, chúng ta có
              \begin{align*}
                  \abs{a_{m} - a_{n}} & = \abs{(a_{m} - b_{m}) + (b_{m} - b_{n}) + (b_{n} - a_{n})}                            \\
                                      & \leq \abs{a_{m} - b_{m}} + \abs{b_{m} - b_{n}} + \abs{b_{n} - a_{n}}                   \\
                                      & < \frac{\varepsilon}{4} + \frac{\varepsilon}{2} + \frac{\varepsilon}{4} = \varepsilon.
              \end{align*}

              Từ điều trên, chúng ta suy ra rằng với mỗi số hữu tỉ dương $\varepsilon$, tồn tại số tự nhiên $N$ sao cho với mọi số tự nhiên $n, m\geq N$, chúng ta có $\abs{a_{m} - a_{n}} < \varepsilon$. Do đó ${(a_{n})}_{n\in\mathbb{N}}$ là một dãy Cauchy hữu tỉ.
    \end{enumerate}
\end{proof}

Chúng ta kí hiệu tập thương gồm các lớp tương đương của quan hệ $\sim$ trên tập hợp các dãy Cauchy hữu tỉ là $\mathscr{C}_{\mathbb{Q}}/_{\sim}$. Trong nội dung tiếp theo của mục này, chúng ta chỉ làm việc với dãy Cauchy hữu tỉ và các lớp tương đương của các dãy Cauchy hữu tỉ.

\subsection{* Các phép toán với dãy Cauchy hữu tỉ}

\begin{theorem}[Phép cộng và phép nhân dãy Cauchy hữu tỉ]\label{theorem:addition-and-multiplication-of-cauchy-sequences}
    ${(a_{n})}_{n\in\mathbb{N}}$ và ${(b_{n})}_{n\in\mathbb{N}}$ là các dãy Cauchy hữu tỉ thì
    \begin{enumerate}[label={(\roman*)}]
        \item ${(a_{n} + b_{n})}_{n\in\mathbb{N}}$ là một dãy Cauchy hữu tỉ\index{Phép cộng dãy Cauchy hữu tỉ}. Chúng ta cũng kí hiệu ${(a_{n} + b_{n})}_{n\in\mathbb{N}} = {(a_{n})}_{n\in\mathbb{N}} + {(b_{n})}_{n\in\mathbb{N}}$.
        \item ${(a_{n}b_{n})}_{n\in\mathbb{N}}$ là một dãy Cauchy hữu tỉ\index{Phép nhân dãy Cauchy hữu tỉ}. Chúng ta cũng kí hiệu ${(a_{n}b_{n})}_{n\in\mathbb{N}} = {(a_{n})}_{n\in\mathbb{N}}\cdot {(b_{n})}_{n\in\mathbb{N}}$.
    \end{enumerate}
\end{theorem}

\begin{proof}
    \begin{enumerate}[label={(\roman*)}]
        \item Chúng ta chọn $\varepsilon$ là một số hữu tỉ dương.

              Theo định nghĩa dãy Cauchy hữu tỉ, với số hữu tỉ dương $\dfrac{\varepsilon}{2}$
              \begin{itemize}
                  \item tồn tại số tự nhiên $N_{a}$ sao cho với mọi số tự nhiên $n, m\geq N_{a}$, chúng ta có $\abs{a_{m} - a_{n}} < \dfrac{\varepsilon}{2}$
                  \item tồn tại số tự nhiên $N_{b}$ sao cho với mọi số tự nhiên $n, m\geq N_{b}$, chúng ta có $\abs{b_{m} - b_{n}} < \dfrac{\varepsilon}{2}$.
              \end{itemize}

              Chúng ta định nghĩa $N = \max\{ N_{a}, N_{b} \}$. Với mọi số tự nhiên $n, m\geq N$, chúng ta có $\abs{a_{m} - a_{n}} < \dfrac{\varepsilon}{2}$ và $\abs{b_{m} - b_{n}} < \dfrac{\varepsilon}{2}$, và
              \[
                  \abs{(a_{m} + b_{m}) - (a_{n} + b_{n})} = \abs{(a_{m} - a_{n}) + (b_{m} - b_{n})} \leq \abs{a_{m} - a_{n}} + \abs{b_{m} - b_{n}} < \frac{\varepsilon}{2} + \frac{\varepsilon}{2} = \varepsilon.
              \]

              Từ những điều trên, chúng ta suy ra rằng với mỗi số hữu tỉ dương $\varepsilon$, tồn tại số tự nhiên $N$ sao cho với mọi số tự nhiên $n, m\geq N$, chúng ta có $\abs{(a_{m} + b_{m}) - (a_{n} + b_{n})} < \varepsilon$. Như vậy ${(a_{n} + b_{n})}_{n\in\mathbb{N}}$ là một dãy Cauchy hữu tỉ.
        \item ${(a_{n})}_{n\in\mathbb{N}}$ và ${(b_{n})}_{n\in\mathbb{N}}$ là các dãy Cauchy hữu tỉ. Theo Định lý~\ref{theorem:cauchy-sequences-are-bounded}, tồn tại hai số hữu tỉ dương $A, B$ sao cho $\abs{a_{n}}\leq A$ và $\abs{b_{n}}\leq B$ với mọi số tự nhiên $n$.

              Chúng ta chọn số hữu tỉ dương $\varepsilon$ bất kì. Theo định nghĩa dãy Cauchy hữu tỉ
              \begin{itemize}
                  \item với số hữu tỉ dương $\dfrac{\varepsilon}{2B}$, tồn tại số tự nhiên $N_{a}$ sao cho với mọi số tự nhiên $n, m\geq N_{a}$, chúng ta có $\abs{a_{m} - a_{n}} < \dfrac{\varepsilon}{2B}$
                  \item với số hữu tỉ dương $\dfrac{\varepsilon}{2A}$, tồn tại số tự nhiên $N_{b}$ sao cho với mọi số tự nhiên $n, m\geq N_{b}$, chúng ta có $\abs{b_{m} - b_{n}} < \dfrac{\varepsilon}{2A}$.
              \end{itemize}

              Chúng ta định nghĩa $N = \max\{ N_{a}, N_{b}\}$. Nếu các số tự nhiên $n, m\geq N$ thì $\abs{a_{m} - a_{n}} < \dfrac{\varepsilon}{2B}$ và $\abs{b_{m} - b_{n}} < \dfrac{\varepsilon}{2A}$. Khi đó chúng ta có
              \begin{align*}
                  \abs{a_{m}b_{m} - a_{n}b_{n}} & = \abs{a_{m}(b_{m} - b_{n}) + b_{n}(a_{m} - a_{n})}               \\
                                                & \leq \abs{a_{m}(b_{m} - b_{n})} + \abs{b_{n}(a_{m} - a_{n})}      \\
                                                & = \abs{a_{m}}\abs{b_{m} - b_{n}} + \abs{b_{n}}\abs{a_{m} - a_{n}} \\
                                                & < A\cdot \frac{\varepsilon}{2A} + B\cdot\frac{\varepsilon}{2B}    \\
                                                & = \frac{\varepsilon}{2} + \frac{\varepsilon}{2} = \varepsilon.
              \end{align*}

              Từ những điều trên, chúng ta suy ra rằng với mỗi số hữu tỉ dương $\varepsilon$, tồn tại số tự nhiên $N$ sao cho với mọi số tự nhiên $n, m\geq N$, chúng ta có $\abs{a_{m}b_{m} - a_{n}b_{n}} < \varepsilon$. Như vậy ${(a_{n}b_{n})}_{n\in\mathbb{N}}$ là một dãy Cauchy hữu tỉ.
    \end{enumerate}
\end{proof}

\begin{corollary}\label{corollary:linear-combination-of-cauchy-sequences}
    ${(a_{n})}_{n\in\mathbb{N}}$ và ${(b_{n})}_{n\in\mathbb{N}}$ là các dãy Cauchy hữu tỉ thì ${(xa_{n} + yb_{n})}_{n\in\mathbb{N}}$ cũng là một dãy Cauchy hữu tỉ, trong đó $x, y$ là hai hằng số hữu tỉ.
\end{corollary}

\begin{proof}
    Theo phần (ii) của Định lý~\ref{theorem:addition-and-multiplication-of-cauchy-sequences}, ${(xa_{n})}_{n\in\mathbb{N}}$ và ${(yb_{n})}_{n\in\mathbb{N}}$ là các dãy Cauchy hữu tỉ.

    Theo phần (i) của Định lý~\ref{theorem:addition-and-multiplication-of-cauchy-sequences}, ${(xa_{n} + yb_{n})}_{n\in\mathbb{N}}$ là một dãy Cauchy hữu tỉ.
\end{proof}

Để định nghĩa phép cộng và phép nhân hai \textit{lớp tương đương của các dãy Cauchy hữu tỉ} (nói cách khác là hai phần tử của $\mathscr{C}_{\mathbb{Q}}/_{\sim}$) theo cách tương tự như với \textit{dãy Cauchy hữu tỉ}, chúng ta cần một định nghĩa không phụ thuộc vào việc chọn phần tử đại diện của lớp tương đương.

\begin{theorem}\label{theorem:basis-of-equivalence-class-of-rational-cauchy-sequences-addition}
    Nếu các dãy Cauchy hữu tỉ ${(a_{n})}_{n\in\mathbb{N}}$, ${(b_{n})}_{n\in\mathbb{N}}$, ${(c_{n})}_{n\in\mathbb{N}}$, ${(d_{n})}_{n\in\mathbb{N}}$ thỏa mãn ${(a_{n})}_{n\in\mathbb{N}}\sim {(c_{n})}_{n\in\mathbb{N}}$ và ${(b_{n})}_{n\in\mathbb{N}}\sim {(d_{n})}_{n\in\mathbb{N}}$ thì
    \begin{enumerate}[label={(\roman*)}]
        \item ${(a_{n} + b_{n})}_{n\in\mathbb{N}} \sim {(c_{n} + d_{n})}_{n\in\mathbb{N}}$.
        \item ${(a_{n}b_{n})}_{n\in\mathbb{N}} \sim {(c_{n}d_{n})}_{n\in\mathbb{N}}$.
    \end{enumerate}
\end{theorem}

\begin{proof}
    \begin{enumerate}[label={(\roman*)}]
        \item Chúng ta chọn số hữu tỉ dương $\varepsilon$ bất kì.

              Theo định nghĩa quan hệ $\sim$ trên tập hợp các dãy Cauchy hữu tỉ, với số hữu tỉ dương $\dfrac{\varepsilon}{2}$
              \begin{itemize}[topsep=0pt]
                  \item tồn tại số tự nhiên $N_{ac}$ sao cho với mọi số tự nhiên $n\geq N_{ac}$, chúng ta có $\abs{a_{n} - c_{n}} < \dfrac{\varepsilon}{2}$,
                  \item tồn tại số tự nhiên $N_{bd}$ sao cho với mọi số tự nhiên $n\geq N_{bd}$, chúng ta có $\abs{b_{n} - d_{n}} < \dfrac{\varepsilon}{2}$.
              \end{itemize}

              Chúng ta định nghĩa $N = \max\{ N_{ac}, N_{bd} \}$. Với mọi số tự nhiên $n\geq N$, chúng ta có $\abs{a_{n} - c_{n}} < \dfrac{\varepsilon}{2}$, $\abs{b_{n} - d_{n}} < \dfrac{\varepsilon}{2}$, và
              \[
                  \abs{(a_{n} + b_{n}) - (c_{n} + d_{n})} = \abs{(a_{n} - c_{n}) + (b_{n} - d_{n})} \leq \abs{a_{n} - c_{n}} + \abs{b_{n} - d_{n}} < \frac{\varepsilon}{2} + \frac{\varepsilon}{2} = \varepsilon.
              \]

              Từ những điều trên, chúng ta suy ra rằng với mỗi số hữu tỉ dương $\varepsilon$, tồn tại số tự nhiên $N$ sao cho với mọi số tự nhiên $n\geq N$, chúng ta có $\abs{(a_{n} + b_{n}) - (c_{n} + d_{n})} < \varepsilon$. Như vậy ${(a_{n} + b_{n})}_{n\in\mathbb{N}} \sim {(c_{n} + d_{n})}_{n\in\mathbb{N}}$.
        \item Chúng ta chọn số hữu tỉ dương $\varepsilon$ bất kì.

              Theo Định lý~\ref{theorem:cauchy-sequences-are-bounded}, vì ${(a_{n})}_{n\in\mathbb{N}}$, ${(d_{n})}_{n\in\mathbb{N}}$ là các dãy Cauchy hữu tỉ nên tồn tại hai số hữu tỉ dương $A, D$ sao cho $\abs{a_{n}} < A$ và $\abs{d_{n}} < D$ với mọi số tự nhiên $n$.

              Theo định nghĩa quan hệ $\sim$ trên tập hợp dãy Cauchy hữu tỉ
              \begin{itemize}
                  \item với số hữu tỉ dương $\dfrac{\varepsilon}{2D}$, tồn tại số tự nhiên $N_{ac}$ sao cho với mọi số tự nhiên $n\geq N_{ac}$, chúng ta có $\abs{a_{n} - c_{n}} < \dfrac{\varepsilon}{2D}$,
                  \item với số hữu tỉ dương $\dfrac{\varepsilon}{2A}$, tồn tại số tự nhiên $N_{bd}$ sao cho với mọi số tự nhiên $n\geq N_{bd}$, chúng ta có $\abs{b_{n} - d_{n}} < \dfrac{\varepsilon}{2A}$.
              \end{itemize}

              Chúng ta định nghĩa $N = \max\{ N_{ac}, N_{bd} \}$. Với mọi số tự nhiên $n\geq N$, chúng ta có $\abs{a_{n} - c_{n}} < \dfrac{\varepsilon}{2D}$, $\abs{b_{n} - d_{n}} < \dfrac{\varepsilon}{2A}$ và
              \begin{align*}
                  \abs{a_{n}b_{n} - c_{n}d_{n}} & = \abs{a_{n}(b_{n} - d_{n}) + d_{n}(a_{n} - c_{n})}               \\
                                                & \leq \abs{a_{n}(b_{n} - d_{n})} + \abs{d_{n}(a_{n} - c_{n})}      \\
                                                & = \abs{a_{n}}\abs{b_{n} - d_{n}} + \abs{d_{n}}\abs{a_{n} - c_{n}} \\
                                                & < A\cdot\frac{\varepsilon}{2A} + D\cdot\frac{\varepsilon}{2D}     \\
                                                & = \frac{\varepsilon}{2} + \frac{\varepsilon}{2} = \varepsilon.
              \end{align*}

              Từ những điều trên, chúng ta suy ra rằng với mỗi số hữu tỉ dương $\varepsilon$, tồn tại số tự nhiên $N$ sao cho với mọi số tự nhiên $n\geq N$, chúng ta có $\abs{a_{n}b_{n} - c_{n}d_{n}} < \varepsilon$. Như vậy ${(a_{n}b_{n})}_{n\in\mathbb{N}} \sim {(c_{n}d_{n})}_{n\in\mathbb{N}}$.
    \end{enumerate}
\end{proof}

\begin{definition}
    Chúng ta kí hiệu lớp tương đương gồm các \textit{dãy số hữu tỉ} tương đương với dãy số hữu tỉ ${(a_{n})}_{n\in\mathbb{N}}$ là $\clsseq{a_{n}}{n}$.

    \begin{enumerate}[label={(\roman*)}]
        \item Phép cộng hai phần tử\index{Phép cộng hai lớp tương đương dãy Cauchy hữu tỉ} của $\mathscr{C}_{\mathbb{Q}}/_{\sim}$ là một phép toán hai ngôi trên tập hợp $\mathscr{C}_{\mathbb{Q}}/_{\sim}$, được kí hiệu là $+$ và được xác định như sau
              \[
                  \clsseq{a_{n}}{n} + \clsseq{b_{n}}{n} = \clsseq{a_{n} + b_{n}}{n}
              \]

              trong đó ${(a_{n})}_{n\in\mathbb{N}}, {(b_{n})}_{n\in\mathbb{N}}$ là các dãy Cauchy hữu tỉ.
        \item Phép nhân hai phần tử\index{Phép nhân hai lớp tương đương dãy Cauchy hữu tỉ} của $\mathscr{C}_{\mathbb{Q}}/_{\sim}$ là một phép toán hai ngôi trên tập hợp $\mathscr{C}_{\mathbb{Q}}/_{\sim}$, được kí hiệu là $\cdot$ và được xác định như sau
              \[
                  \clsseq{a_{n}}{n} \cdot \clsseq{b_{n}}{n} = \clsseq{a_{n}b_{n}}{n}
              \]

              trong đó ${(a_{n})}_{n\in\mathbb{N}}, {(b_{n})}_{n\in\mathbb{N}}$ là các dãy Cauchy hữu tỉ.
    \end{enumerate}
\end{definition}

Một cách không hình thức, chúng ta có thể diễn đạt định nghĩa trên thành: tổng (tích) của hai lớp tương đương là lớp tương đương của tổng (tích) hai dãy Cauchy. Với định nghĩa trên, phép cộng và phép nhân của hai phần tử trong $\mathscr{C}_{\mathbb{Q}}/_{\sim}$ không phụ thuộc vào việc chọn phần tử đại diện của lớp tương đương. Định nghĩa này cũng cho phép chúng ta chứng minh các tính chất của phép cộng, phép nhân trên $\mathscr{C}_{\mathbb{Q}}/_{\sim}$ theo cách dễ dàng.

\begin{theorem}
    \begin{enumerate}[label={(\roman*)}]
        \item Phép cộng trên $\mathscr{C}_{\mathbb{Q}}/_{\sim}$ có tính chất kết hợp.
        \item Phép cộng trên $\mathscr{C}_{\mathbb{Q}}/_{\sim}$ có phần tử đồng nhất.
        \item Mỗi phần tử của $\mathscr{C}_{\mathbb{Q}}/_{\sim}$ có phần tử đối.
        \item Phép cộng trên $\mathscr{C}_{\mathbb{Q}}/_{\sim}$ có tính chất giao hoán.
    \end{enumerate}
\end{theorem}

\begin{proof}
    \begin{enumerate}[label={(\roman*)}]
        \item Với mỗi phần tử $\clsseq{a_{n}}{n}, \clsseq{b_{n}}{n}, \clsseq{c_{n}}{n}$ của $\mathscr{C}_{\mathbb{Q}}/_{\sim}$, theo định nghĩa phép cộng trên $\mathscr{C}_{\mathbb{Q}}/_{\sim}$ và tính chất kết hợp của phép cộng số hữu tỉ, chúng ta có
              \begin{align*}
                  \left(\clsseq{a_{n}}{n} + \clsseq{b_{n}}{n}\right) + \clsseq{c_{n}}{n} & = \clsseq{a_{n} + b_{n}}{n} + \clsseq{c_{n}}{n}                          \\
                                                                                         & = \clsseq{(a_{n} + b_{n}) + c_{n}}{n}                                    \\
                                                                                         & = \clsseq{a_{n} + (b_{n} + c_{n})}{n}                                    \\
                                                                                         & = \clsseq{a_{n}}{n} + \clsseq{b_{n} + c_{n}}{n}                          \\
                                                                                         & = \clsseq{a_{n}}{n} + \left(\clsseq{b_{n}}{n} + \clsseq{c_{n}}{n}\right)
              \end{align*}

              Vậy phép cộng trên $\mathscr{C}_{\mathbb{Q}}/_{\sim}$ có tính chất kết hợp.
        \item $\clsseq{0}{n}$ là một phần tử của $\mathscr{C}_{\mathbb{Q}}/_{\sim}$. Với mọi phần tử $\clsseq{a_{n}}{n}$ của $\mathscr{C}_{\mathbb{Q}}/_{\sim}$, theo định nghĩa phép cộng trên $\mathscr{C}_{\mathbb{Q}}/_{\sim}$, chúng ta có
              \begin{align*}
                  \clsseq{a_{n}}{n} + \clsseq{0}{n} & = \clsseq{a_{n} + 0}{n} = \clsseq{a_{n}}{n}, \\
                  \clsseq{0}{n} + \clsseq{a_{n}}{n} & = \clsseq{0 + a_{n}}{n} = \clsseq{a_{n}}{n}.
              \end{align*}

              Vậy $\clsseq{0}{n}$ là một phần tử đồng nhất của phép cộng trên $\mathscr{C}_{\mathbb{Q}}/_{\sim}$.
        \item Với mỗi phần tử $\clsseq{a_{n}}{n}$ của $\mathscr{C}_{\mathbb{Q}}/_{\sim}$, chúng ta có
              \begin{align*}
                  \clsseq{a_{n}}{n} + \clsseq{-a_{n}}{n} & = \clsseq{a_{n} + (-a_{n})}{n} = \clsseq{0}{n}, \\
                  \clsseq{-a_{n}}{n} + \clsseq{a_{n}}{n} & = \clsseq{(-a_{n}) + a_{n}}{n} = \clsseq{0}{n}.
              \end{align*}

              Vậy mỗi phần tử của $\mathscr{C}_{\mathbb{Q}}/_{\sim}$ có phần tử đối.
        \item Với mỗi phần tử $\clsseq{a_{n}}{n}, \clsseq{b_{n}}{n}$ của $\mathscr{C}_{\mathbb{Q}}/_{\sim}$, theo định nghĩa phép cộng trên $\mathscr{C}_{\mathbb{Q}}/_{\sim}$ và tính chất giao hoán của phép cộng số hữu tỉ, chúng ta có
              \begin{align*}
                  \clsseq{a_{n}}{n} + \clsseq{b_{n}}{n} & = \clsseq{a_{n} + b_{n}}{n}              \\
                                                        & = \clsseq{b_{n} + a_{n}}{n}              \\
                                                        & = \clsseq{b_{n}}{n} + \clsseq{a_{n}}{n}.
              \end{align*}

              Vậy phép cộng trên $\mathscr{C}_{\mathbb{Q}}/_{\sim}$ có tính chất giao hoán.
    \end{enumerate}
\end{proof}

\begin{theorem}\label{theorem:rational-cauchy-sequences-abelian-group}
    \begin{enumerate}[label={(\roman*)}]
        \item Phép nhân trên $\mathscr{C}_{\mathbb{Q}}/_{\sim}$ có tính chất kết hợp.
        \item Phép nhân trên $\mathscr{C}_{\mathbb{Q}}/_{\sim}$ có tính chất phân phối với phép cộng trên $\mathscr{C}_{\mathbb{Q}}/_{\sim}$.
        \item Phép nhân trên $\mathscr{C}_{\mathbb{Q}}/_{\sim}$ có phần tử đồng nhất, và phần tử này khác với phần tử đồng nhất của phép cộng trên $\mathscr{C}_{\mathbb{Q}}/_{\sim}$.
        \item Phép nhân trên $\mathscr{C}_{\mathbb{Q}}/_{\sim}$ có tính chất giao hoán.
    \end{enumerate}
\end{theorem}

\begin{proof}
    \begin{enumerate}[label=(\roman*)]
        \item Với mỗi phần tử $\clsseq{a_{n}}{n}, \clsseq{b_{n}}{n}, \clsseq{c_{n}}{n}$ của $\mathscr{C}_{\mathbb{Q}}/_{\sim}$, theo định nghĩa phép nhân trên $\mathscr{C}_{\mathbb{Q}}/_{\sim}$ và tính chất kết hợp của phép nhân số hữu tỉ, chúng ta có
              \begin{align*}
                  \left(\clsseq{a_{n}}{n} \cdot \clsseq{b_{n}}{n}\right) \cdot \clsseq{c_{n}}{n} & = \clsseq{a_{n}b_{n}}{n} \cdot \clsseq{c_{n}}{n}                                 \\
                                                                                                 & = \clsseq{(a_{n}b_{n})c_{n}}{n}                                                  \\
                                                                                                 & = \clsseq{a_{n}(b_{n}c_{n})}{n}                                                  \\
                                                                                                 & = \clsseq{a_{n}}{n} \cdot \clsseq{b_{n}c_{n}}{n}                                 \\
                                                                                                 & = \clsseq{a_{n}}{n} \cdot \left(\clsseq{b_{n}}{n} \cdot \clsseq{c_{n}}{n}\right)
              \end{align*}

              Vậy phép nhân trên $\mathscr{C}_{\mathbb{Q}}/_{\sim}$ có tính chất kết hợp.
        \item Với mỗi phần tử $\clsseq{a_{n}}{n}, \clsseq{b_{n}}{n}, \clsseq{c_{n}}{n}$ của $\mathscr{C}_{\mathbb{Q}}/_{\sim}$, theo định nghĩa phép nhân và phép cộng trên $\mathscr{C}_{\mathbb{Q}}/_{\sim}$ và tính chất phân phối của phép nhân với phép cộng số hữu tỉ, chúng ta có
              \begin{align*}
                  \left(\clsseq{a_{n}}{n} + \clsseq{b_{n}}{n}\right) \cdot \clsseq{c_{n}}{n} & = \clsseq{a_{n} + b_{n}}{n} \cdot \clsseq{c_{n}}{n}                                      \\
                                                                                             & = \clsseq{(a_{n}+ b_{n})c_{n}}{n}                                                        \\
                                                                                             & = \clsseq{a_{n}c_{n} + b_{n}c_{n}}{n}                                                    \\
                                                                                             & = \clsseq{a_{n}c_{n}}{n} + \clsseq{b_{n}c_{n}}{n}                                        \\
                                                                                             & = \clsseq{a_{n}}{n} \cdot \clsseq{c_{n}}{n} + \clsseq{b_{n}}{n} \cdot \clsseq{c_{n}}{n}.
              \end{align*}

              Hoàn toàn tương tự, chúng ta cũng chứng minh được rằng
              \[
                  \clsseq{c_{n}}{n}\cdot \left(\clsseq{a_{n}}{n} + \clsseq{b_{n}}{n}\right) = \clsseq{c_{n}}{n} \cdot \clsseq{a_{n}}{n} + \clsseq{c_{n}}{n} \cdot \clsseq{b_{n}}{n}.
              \]

              Vậy phép nhân trên $\mathscr{C}_{\mathbb{Q}}/_{\sim}$ có tính chất phân phối với phép cộng trên $\mathscr{C}_{\mathbb{Q}}/_{\sim}$.
        \item $\clsseq{1}{n}$ là một phần tử của $\mathscr{C}_{\mathbb{Q}}/_{\sim}$. Với mọi phần tử $\clsseq{a_{n}}{n}$ của $\mathscr{C}_{\mathbb{Q}}/_{\sim}$, theo định nghĩa phép nhân trên $\mathscr{C}_{\mathbb{Q}}/_{\sim}$, chúng ta có
              \begin{align*}
                  \clsseq{a_{n}}{n}\cdot \clsseq{1}{n} & = \clsseq{a_{n}\cdot 1}{n} = \clsseq{a_{n}}{n}, \\
                  \clsseq{1}{n}\cdot \clsseq{a_{n}}{n} & = \clsseq{1\cdot a_{n}}{n} = \clsseq{a_{n}}{n}.
              \end{align*}

              Vậy $\clsseq{1}{n}$ là một phần tử đồng nhất của phép nhân trên $\mathscr{C}_{\mathbb{Q}}/_{\sim}$.

              Giả sử phản chứng rằng $\clsseq{1}{n} = \clsseq{0}{n}$. Theo định nghĩa quan hệ $\sim$ trên $\mathscr{C}_{\mathbb{Q}}$, chúng ta có ${(1)}_{n\in\mathbb{N}} \sim {(0)}_{n\in\mathbb{N}}$. Theo Định lý~\ref{theorem:equivalent-rational-sequences}, hai dãy số hữu tỉ ${(1)}_{n\in\mathbb{N}}$ và ${(0)}_{n\in\mathbb{N}}$ hội tụ đến cùng một số hữu tỉ. Điều này là vô lý vì ${(1)}_{n\in\mathbb{N}}$ hội tụ đến $1$, còn ${(0)}_{n\in\mathbb{N}}$ hội tụ đến $0$ và $0\ne 1$.

              Do đó $\clsseq{1}{n}\ne \clsseq{0}{n}$.
        \item Với mỗi phần tử $\clsseq{a_{n}}{n}, \clsseq{b_{n}}{n}$ của $\mathscr{C}_{\mathbb{Q}}/_{\sim}$, theo định nghĩa phép nhân trên $\mathscr{C}_{\mathbb{Q}}/_{\sim}$ và tính chất giao hoán của phép nhân số hữu tỉ, chúng ta có
              \begin{align*}
                  \clsseq{a_{n}}{n} \cdot \clsseq{b_{n}}{n} & = \clsseq{a_{n}b_{n}}{n}                     \\
                                                            & = \clsseq{b_{n}a_{n}}{n}                     \\
                                                            & = \clsseq{b_{n}}{n} \cdot \clsseq{a_{n}}{n}.
              \end{align*}

              Vậy phép nhân trên $\mathscr{C}_{\mathbb{Q}}/_{\sim}$ có tính chất giao hoán.
    \end{enumerate}
\end{proof}

Trên đây chúng ta đã chứng minh được rằng tập hợp $\mathscr{C}_{\mathbb{Q}}/_{\sim}$ cùng hai phép toán cộng và nhân thỏa mãn 8 tiên đề đầu tiên trong các tiên đề về trường. Để chứng minh rằng tiên đề thứ 9 cũng được thỏa mãn, chúng ta sử dụng định lý sau.
\begin{theorem}\label{theorem:nonzero-cauchy-sequences}
    Nếu dãy Cauchy hữu tỉ ${(a_{n})}_{n\in\mathbb{N}}$ không hội tụ đến $0$ thì tồn tại số hữu tỉ dương $\varepsilon$ sao cho tồn tại số tự nhiên $N$ sao cho với mọi số tự nhiên $n\geq N$, chúng ta có $\abs{a_{n}}\geq \varepsilon$ và $a_{n}\ne 0$.
\end{theorem}

\begin{proof}
    Theo định nghĩa dãy số hữu tỉ hội tụ, dãy Cauchy hữu tỉ ${(a_{n})}_{n\in\mathbb{N}}$ không hội tụ đến $0$ khi và chỉ khi tồn tại số hữu tỉ dương $\varepsilon_{0}$ sao cho với mọi số tự nhiên $N_{a}$, tồn tại số tự nhiên $n\geq N_{a}$, chúng ta có $\abs{a_{n}} \geq \varepsilon_{0}$. ($\star$)

    Theo định nghĩa dãy Cauchy hữu tỉ, với số hữu tỉ dương $\dfrac{\varepsilon_{0}}{2}$, tồn tại số tự nhiên $N$ sao cho với mọi số tự nhiên $n, m\geq N$, chúng ta có $\abs{a_{m} - a_{n}} < \dfrac{\varepsilon_{0}}{2}$. ($\star\star$)

    Vì ($\star$) nên với số tự nhiên $N$, tồn tại số tự nhiên $N'\geq N$ sao cho $\abs{a_{N'}} \geq \varepsilon_{0}$. Cùng với ($\star\star$), chúng ta suy ra rằng với mọi số tự nhiên $n\geq N$, chúng ta có
    \[
        \abs{a_{n}} = \abs{a_{N'} - (a_{N'} - a_{n})} \geq \abs{a_{N'}} - \abs{a_{N'} - a_{n}} \geq \varepsilon_{0} - \frac{\varepsilon_{0}}{2} = \frac{\varepsilon_{0}}{2} > 0.
    \]

    Do đó, với mọi số tự nhiên $n\geq N$, chúng ta có $\abs{a_{n}}\ne 0$. Vì giá trị tuyệt đối của một số hữu tỉ bằng $0$ khi và chỉ khi số hữu tỉ đó bằng $0$ nên với mọi số tự nhiên $n\geq N$, chúng ta có $a_{n}\ne 0$.

    Vậy, với số hữu tỉ dương $\varepsilon = \dfrac{\varepsilon_{0}}{2}$, tồn tại tự nhiên $N$ sao cho với mọi số tự nhiên $n\geq N$, chúng ta có $\abs{a_{n}}\geq \varepsilon$ và $a_{n}\ne 0$.
\end{proof}

\begin{theorem}
    Với mỗi phần tử $\alpha\ne \clsseq{0}{n}$ trong $\mathscr{C}_{\mathbb{Q}}/_{\sim}$, tồn tại phần tử $\beta$ trong $\mathscr{C}_{\mathbb{Q}}/_{\sim}$ sao cho $\alpha\cdot\beta = \beta\cdot\alpha = \clsseq{1}{n}$.
\end{theorem}

\begin{proof}
    Theo định nghĩa quan hệ $\sim$ trên tập hợp $\mathscr{C}_{\mathbb{Q}}$, mọi phần tử của lớp tương đương $\alpha$ đều không tương đương với dãy số hữu tỉ ${(0)}_{n\in\mathbb{N}}$, kéo theo mọi phần tử của lớp tương đương $\alpha$ đều không hội tụ đến $0$.

    Chúng ta chọn ${(a_{n})}_{n\in\mathbb{N}}$ là một phần tử của lớp tương đương $\alpha$. Vì ${(a_{n})}_{n\in\mathbb{N}}$ không hội tụ đến $0$ nên theo Định lý~\ref{theorem:nonzero-cauchy-sequences}, tồn tại số tự nhiên $N$ sao cho với mọi số tự nhiên $n\geq N$, chúng ta có $a_{n}\ne 0$. Chúng ta định nghĩa dãy số hữu tỉ ${(b_{n})}_{n\in\mathbb{N}}$ như sau.
    \[
        b_{n} = \begin{cases}
            0                & \text{nếu $n < N$}   \\
            \dfrac{1}{a_{n}} & \text{nếu $n\geq N$}
        \end{cases}
    \]

    Chúng ta chứng minh ${(b_{n})}_{n\in\mathbb{N}}$ là một dãy Cauchy hữu tỉ.

    Chúng ta chọn số hữu tỉ dương $\varepsilon$ bất kì. Theo Định lý~\ref{theorem:nonzero-cauchy-sequences}, tồn tại số hữu tỉ dương $\varepsilon_{0}$ sao cho tồn tại số tự nhiên $N'$ sao cho với mọi số tự nhiên $n\geq N'$, chúng ta có $\abs{a_{n}}\geq \varepsilon_{0}$. Theo định nghĩa dãy Cauchy hữu tỉ, với số hữu tỉ dương $\varepsilon\cdot{\varepsilon_{0}}^{2}$, tồn tại số tự nhiên $N_{a}$ sao cho với mọi số tự nhiên $n, m\geq N_{a}$, chúng ta có $\abs{a_{m} - a_{n}} < \varepsilon\cdot{\varepsilon_{0}}^{2}$.

    Chúng ta định nghĩa $N'' = \max\{ N_{a}, N' \}$. Với mọi số tự nhiên $n, m\geq N''$, chúng ta có $\abs{a_{m} - a_{n}} < \varepsilon\cdot{\varepsilon_{0}}^{2}$, $\abs{a_{n}}\geq\varepsilon_{0}$, và
    \[
        \abs{b_{m} - b_{n}} = \abs{\frac{1}{a_{m}} - \frac{1}{a_{n}}} = \abs{\frac{a_{m} - a_{n}}{a_{m}a_{n}}} = \frac{\abs{a_{m} - a_{n}}}{\abs{a_{m}}\abs{a_{n}}} < \frac{\varepsilon\cdot{\varepsilon_{0}}^{2}}{{\varepsilon_{0}}^{2}} = \varepsilon.
    \]

    Do đó ${(b_{n})}_{n\in\mathbb{N}}$ là một dãy Cauchy hữu tỉ.

    Mặt khác, ${(a_{n}b_{n})}_{n\in\mathbb{N}}$ là một dãy số hữu tỉ dừng, vì với mọi số tự nhiên $n\geq N$, chúng ta có $a_{n}b_{n} = 1$. Do đó, ${(a_{n}b_{n})}_{n\in\mathbb{N}}$ hội tụ đến $1$. Theo định nghĩa và tính chất giao hoán của phép nhân trong $\mathscr{C}_{\mathbb{Q}}/_{\sim}$, chúng ta suy ra
    \[
        \clsseq{a_{n}}{n}\cdot\clsseq{b_{n}}{n} = \clsseq{b_{n}}{n}\cdot\clsseq{a_{n}}{n} = \clsseq{b_{n}a_{n}}{n} = \clsseq{1}{n}.
    \]

    Vậy, với mỗi phần tử $\alpha\ne \clsseq{0}{n}$ trong $\mathscr{C}_{\mathbb{Q}}/_{\sim}$, tồn tại phần tử $\beta$ trong $\mathscr{C}_{\mathbb{Q}}/_{\sim}$ sao cho $\alpha\cdot\beta = \beta\cdot\alpha = \clsseq{1}{n}$.
\end{proof}

Đến lúc này, chúng ta đã chứng minh được rằng tập hợp $\mathscr{C}_{\mathbb{Q}}/_{\sim}$ cùng hai phép toán cộng và nhân thỏa mãn các tiên đề về trường.

\subsection{* Quan hệ tiền thứ tự giữa các dãy Cauchy hữu tỉ}

Trong nội dung này, chúng ta định nghĩa một \textit{quan hệ tiền thứ tự} giữa các dãy Cauchy hữu tỉ để từ đó định nghĩa một \textit{quan hệ thứ tự} giữa các phần tử của $\mathscr{C}_{\mathbb{Q}}/_{\sim}$. Để xem thêm về \textit{quan hệ tiền thứ tự} và \textit{quan hệ thứ tự}, bạn đọc có thể đọc tại Chương~\ref{chapter:relations-and-mappings}, Mục~\ref{subsection:order-and-preorder-relations}.

\begin{definition}
    ${(a_{n})}_{n\in\mathbb{N}}$ và ${(b_{n})}_{n\in\mathbb{N}}$ là hai dãy Cauchy hữu tỉ.
    \begin{enumerate}[label={(\roman*)}]
        \item Chúng ta nói ${(a_{n})}_{n\in\mathbb{N}}$ và ${(b_{n})}_{n\in\mathbb{N}}$ có quan hệ $\lesssim$ nếu và chỉ nếu\index{Quan hệ tiền thứ tự giữa các dãy Cauchy hữu tỉ} ${(a_{n})}_{n\in\mathbb{N}}$ \textbf{tương đương} với ${(b_{n})}_{n\in\mathbb{N}}$ hoặc tồn tại số tự nhiên $N$ sao cho với mọi số tự nhiên $n\geq N$, có $a_{n}\leq b_{n}$.
        \item Chúng ta nói ${(a_{n})}_{n\in\mathbb{N}}$ và ${(b_{n})}_{n\in\mathbb{N}}$ có quan hệ $<$ nếu và chỉ nếu ${(a_{n})}_{n\in\mathbb{N}}$ \textbf{không tương đương} với ${(b_{n})}_{n\in\mathbb{N}}$ và ${(a_{n})}_{n\in\mathbb{N}}\leq {(b_{n})}_{n\in\mathbb{N}}$.
    \end{enumerate}
\end{definition}

Theo định nghĩa trên, quan hệ $<$ trên tập hợp $\mathscr{C}_{\mathbb{Q}}$ là trường hợp riêng của quan hệ $\lesssim$ trên tập hợp $\mathscr{C}_{\mathbb{Q}}$.

Để chứng minh các tính chất của quan hệ $<, \lesssim$ trên tập hợp $\mathscr{C}_{\mathbb{Q}}$, chúng ta mở rộng Định lý~\ref{theorem:nonzero-cauchy-sequences} thành định lý sau. Chứng minh của định lý sau là chứng minh của Định lý~\ref{theorem:nonzero-cauchy-sequences} sau khi được bổ sung.
\begin{theorem}\label{theorem:nonzero-cauchy-sequences-and-preorder}
    Nếu dãy Cauchy hữu tỉ ${(a_{n})}_{n\in\mathbb{N}}$ không hội tụ đến $0$ thì có đúng một trong hai khả năng sau
    \begin{itemize}
        \item Tồn tại số hữu tỉ dương $q$ sao cho tồn tại số tự nhiên $N$ sao cho với mọi số tự nhiên $n\geq N$, có $a_{n}\geq q$.
        \item Tồn tại số hữu tỉ dương $q$ sao cho tồn tại số tự nhiên $N$ sao cho với mọi số tự nhiên $n\geq N$, có $a_{n}\leq -q$.
    \end{itemize}
\end{theorem}

\begin{proof}
    Theo định nghĩa dãy số hữu tỉ hội tụ, dãy Cauchy hữu tỉ ${(a_{n})}_{n\in\mathbb{N}}$ không hội tụ đến $0$ khi và chỉ khi tồn tại số hữu tỉ dương $\varepsilon_{0}$ sao cho với mọi số tự nhiên $N_{a}$, tồn tại số tự nhiên $n\geq N_{a}$, chúng ta có $\abs{a_{n}} \geq \varepsilon_{0}$. ($\star$)

    Theo định nghĩa dãy Cauchy hữu tỉ, với số hữu tỉ dương $\dfrac{\varepsilon_{0}}{2}$, tồn tại số tự nhiên $N$ sao cho với mọi số tự nhiên $n, m\geq N$, chúng ta có $\abs{a_{m} - a_{n}} < \dfrac{\varepsilon_{0}}{2}$. ($\star\star$)

    Vì ($\star$) nên với số tự nhiên $N$, tồn tại số tự nhiên $N'\geq N$ sao cho $\abs{a_{N'}} \geq \varepsilon_{0}$. Cùng với ($\star\star$), chúng ta suy ra rằng với mọi số tự nhiên $n\geq N$, chúng ta có
    \[
        \abs{a_{n}} = \abs{a_{N'} - (a_{N'} - a_{n})} \geq \abs{a_{N'}} - \abs{a_{N'} - a_{n}} \geq \varepsilon_{0} - \frac{\varepsilon_{0}}{2} = \frac{\varepsilon_{0}}{2} > 0.
    \]

    Do đó, với mọi số tự nhiên $n\geq N$, chúng ta có $\abs{a_{n}}\ne 0$. Vì giá trị tuyệt đối của một số hữu tỉ bằng $0$ khi và chỉ khi số hữu tỉ đó bằng $0$ nên với mọi số tự nhiên $n\geq N$, chúng ta có $a_{n}\ne 0$.

    Theo định nghĩa dãy Cauchy hữu tỉ, với số hữu tỉ dương $\dfrac{\varepsilon_{0}}{4}$, tồn tại số tự nhiên $N''$ sao cho với mọi số tự nhiên $n, m\geq N''$, chúng ta có $\abs{a_{m} - a_{n}} < \dfrac{\varepsilon_{0}}{4}$. Chúng ta chọn $N_{0} = \max\{ N, N'' \}$. Khi đó, với mọi số tự nhiên $n, m\geq N_{0}$, có $\abs{a_{m} - a_{n}} < \dfrac{\varepsilon_{0}}{4}$ và $\abs{a_{n}}\geq\dfrac{\varepsilon_{0}}{2}$. Vì $\abs{a_{N_{0}}}\geq\dfrac{\varepsilon_{0}}{2}$ và $a_{N_{0}}\ne 0$ nên chỉ có thể xảy ra một trong hai khả năng sau:
    \begin{enumerate}[label={\textbf{Khả năng \arabic*.}},itemindent=2cm]
        \item $a_{N_{0}}\geq \dfrac{\varepsilon_{0}}{2}$.

              Với mọi $n\geq N_{0}$, chúng ta có
              \begin{align*}
                  a_{n} & = (a_{n} - a_{N_{0}}) + a_{N_{0}} \geq -\abs{a_{n} - a_{N_{0}}} + \frac{\varepsilon_{0}}{2} \geq -\frac{\varepsilon_{0}}{4} + \frac{\varepsilon_{0}}{2} = \frac{\varepsilon_{0}}{4}.
              \end{align*}

              Do đó, với số hữu tỉ dương $q = \dfrac{\varepsilon_{0}}{4}$, tồn tại số tự nhiên $N_{0}$ sao cho với mọi số tự nhiên $n\geq N_{0}$, có $a_{n}\geq q$.
        \item $a_{N_{0}}\leq -\dfrac{\varepsilon_{0}}{2}$.

              Với mọi $n\geq N_{0}$, chúng ta có
              \begin{align*}
                  a_{n} & = (a_{n} - a_{N_{0}}) + a_{N_{0}} \leq \abs{a_{n} - a_{N_{0}}} + \left(-\frac{\varepsilon_{0}}{2}\right) \leq \frac{\varepsilon_{0}}{4} + \left(-\frac{\varepsilon_{0}}{2}\right) = -\frac{\varepsilon_{0}}{4}.
              \end{align*}

              Do đó, với số hữu tỉ dương $q = \dfrac{\varepsilon_{0}}{4}$, tồn tại số tự nhiên $N_{0}$ sao cho với mọi số tự nhiên $n\geq N_{0}$, có $a_{n}\leq -q$.
    \end{enumerate}
\end{proof}

Bằng định lý trên, chúng ta chứng minh được điều kiện cần và đủ để hai dãy Cauchy hữu tỉ có quan hệ $<$.

\begin{theorem}\label{theorem:strictly-precedence-cauchy-sequence}
    Hai dãy Cauchy hữu tỉ ${(a_{n})}_{n\in\mathbb{N}}$ và ${(b_{n})}_{n\in\mathbb{N}}$ thỏa mãn ${(a_{n})}_{n\in\mathbb{N}} < {(b_{n})}_{n\in\mathbb{N}}$ khi và chỉ khi tồn tại số hữu tỉ dương $q$ sao cho tồn tại số tự nhiên $N$ sao cho với mọi số tự nhiên $n\geq N$, có $a_{n} - b_{n}\leq -q$. Bằng kí hiệu hình thức, điều kiện trên được viết như sau:
    \[
        \exists q > 0 \Biggl( \exists N \bigl( \forall n\geq N (a_{n} - b_{n}\leq -q) \bigr) \Biggr).
    \]
\end{theorem}

\begin{proof}
    ($\Rightarrow$) ${(a_{n})}_{n\in\mathbb{N}} < {(b_{n})}_{n\in\mathbb{N}}$.

    Theo định nghĩa quan hệ $<$ trên tập hợp $\mathscr{C}_{\mathbb{Q}}$, hai dãy Cauchy hữu tỉ ${(a_{n})}_{n\in\mathbb{N}}$ và ${(b_{n})}_{n\in\mathbb{N}}$ không tương đương. Theo định nghĩa quan hệ tương đương giữa các dãy số hữu tỉ, dãy số hữu tỉ ${(a_{n} - b_{n})}_{n\in\mathbb{N}}$ không hội tụ đến $0$. Theo Định lý~\ref{theorem:nonzero-cauchy-sequences-and-preorder}, chỉ có một trong hai khả năng sau
    \begin{enumerate}[label={\textbf{Khả năng \arabic*.}},itemindent=2cm]
        \item Tồn tại số hữu tỉ dương $q$ sao cho tồn tại số tự nhiên $N$ sao cho với mọi số tự nhiên $n\geq N$, có $a_{n} - b_{n}\geq q$.
        \item Tồn tại số hữu tỉ dương $q$ sao cho tồn tại số tự nhiên $N$ sao cho với mọi số tự nhiên $n\geq N$, có $a_{n} - b_{n}\leq -q$.
    \end{enumerate}

    Giả sử phản chứng rằng \textbf{Khả năng 1} xảy ra: Tồn tại số hữu tỉ dương $q$ sao cho tồn tại số tự nhiên $N$ sao cho với mọi số tự nhiên $n\geq N$, có $a_{n} - b_{n}\geq q$.

    Mặt khác, theo định nghĩa quan hệ $<$ trên tập hợp $\mathscr{C}_{\mathbb{Q}}$, vì ${(a_{n})}_{n\in\mathbb{N}} < {(b_{n})}_{n\in\mathbb{N}}$ nên tồn tại số tự nhiên $N_{ab}$ sao cho với mọi số tự nhiên $n\geq N_{ab}$, có $a_{n}\leq b_{n}$.

    Chúng ta chọn $m$ là một số tự nhiên lớn hơn $N$ và $N_{ab}$, khi đó $a_{m} > b_{m}$ (vì $a_{m} - b_{m}\geq q$) và $a_{m}\leq b_{m}$. Đây là một điều vô lý vì chỉ có đúng một trong hai mệnh đề $a_{m} > b_{m}$ và $a_{m}\leq b_{m}$ là đúng. Do đó giả sử phản chứng là sai.

    Như vậy chỉ có \textbf{Khả năng 2}: Tồn tại số hữu tỉ dương $q$ sao cho tồn tại số tự nhiên $N$ sao cho với mọi số tự nhiên $n\geq N$, có $a_{n} - b_{n}\leq -q$.

    \bigskip

    ($\Leftarrow$) Tồn tại số hữu tỉ dương $q$ sao cho tồn tại số tự nhiên $N$ sao cho với mọi số tự nhiên $n\geq N$, có $a_{n} - b_{n} < -q$.

    Từ điều trên, chúng ta suy ra rằng với mọi số tự nhiên $n\geq N$, có $a_{n}\leq b_{n}$ (vì $a_{n} - b_{n} < -q < 0$). Theo định nghĩa quan hệ $\lesssim$ trên tập hợp $\mathscr{C}_{\mathbb{Q}}$, chúng ta suy ra ${(a_{n})}_{n\in\mathbb{N}}\lesssim {(b_{n})}_{n\in\mathbb{N}}$.

    Mặt khác, với số hữu tỉ dương $q$, với mọi số tự nhiên $N'$, tồn tại số tự nhiên $n\geq N'$ sao cho $\abs{a_{n} - b_{n}}\geq q$ (một số tự nhiên $n$ như vậy chính là $\max\{ N, N' \}$). Bằng kí hiệu hình thức, phát biểu vừa rồi được viết là
    \[
        \exists \varepsilon > 0\Biggl( \forall N' \bigl( \exists n\geq N'( \abs{a_{n} - b_{n}} \geq \varepsilon ) \bigr) \Biggr)
    \]

    Điều trên có nghĩa là dãy số hữu tỉ ${(a_{n} - b_{n})}_{n\in\mathbb{N}}$ không hội tụ đến $0$, kéo theo hai dãy số hữu tỉ ${(a_{n})}_{n\in\mathbb{N}}$ và ${(b_{n})}_{n\in\mathbb{N}}$ không tương đương.

    Theo định nghĩa quan hệ $<$ trên tập hợp $\mathscr{C}_{\mathbb{Q}}$, chúng ta suy ra ${(a_{n})}_{n\in\mathbb{N}} < {(b_{n})}_{n\in\mathbb{N}}$.

    \bigskip

    Vậy hai dãy Cauchy hữu tỉ ${(a_{n})}_{n\in\mathbb{N}}$ và ${(b_{n})}_{n\in\mathbb{N}}$ thỏa mãn ${(a_{n})}_{n\in\mathbb{N}} < {(b_{n})}_{n\in\mathbb{N}}$ khi và chỉ khi tồn tại số hữu tỉ dương $q$ sao cho tồn tại số tự nhiên $N$ sao cho với mọi số tự nhiên $n\geq N$, có $a_{n} - b_{n}\leq -q$.
\end{proof}

\begin{theorem}
    Quan hệ $\lesssim$ giữa các dãy Cauchy hữu tỉ là một quan hệ tiền thứ tự toàn phần.
\end{theorem}

\begin{proof}
    Với mỗi dãy Cauchy hữu tỉ ${(a_{n})}_{n\in\mathbb{N}}$, chúng ta có ${(a_{n})}_{n\in\mathbb{N}} \sim {(a_{n})}_{n\in\mathbb{N}}$. Theo định nghĩa quan hệ $\lesssim$ trên tập hợp $\mathscr{C}_{\mathbb{Q}}$, ${(a_{n})}_{n\in\mathbb{N}} \lesssim {(a_{n})}_{n\in\mathbb{N}}$. Do đó quan hệ $\lesssim$ có tính chất phản xạ.

    Nếu các dãy Cauchy hữu tỉ ${(a_{n})}_{n\in\mathbb{N}}, {(b_{n})}_{n\in\mathbb{N}}, {(c_{n})}_{n\in\mathbb{N}}$ thỏa mãn ${(a_{n})}_{n\in\mathbb{N}}\lesssim {(b_{n})}_{n\in\mathbb{N}}$ và ${(b_{n})}_{n\in\mathbb{N}}\lesssim {(c_{n})}_{n\in\mathbb{N}}$ thì chúng ta xét đủ các trường hợp sau:
    \begin{enumerate}[label={\textbf{Trường hợp \arabic*.}},itemindent=2cm]
        \item ${(a_{n})}_{n\in\mathbb{N}}\sim {(b_{n})}_{n\in\mathbb{N}}$ và ${(b_{n})}_{n\in\mathbb{N}}\sim {(c_{n})}_{n\in\mathbb{N}}$.

              Vì quan hệ $\sim$ giữa các dãy Cauchy hữu tỉ là một quan hệ tương đương nên theo tính chất bắc cầu, chúng ta suy ra ${(a_{n})}_{n\in\mathbb{N}}\sim {(c_{n})}_{n\in\mathbb{N}}$. Theo định nghĩa quan hệ $\lesssim$ giữa các dãy Cauchy hữu tỉ, ${(a_{n})}_{n\in\mathbb{N}}\lesssim {(c_{n})}_{n\in\mathbb{N}}$.
        \item ${(a_{n})}_{n\in\mathbb{N}}\sim {(b_{n})}_{n\in\mathbb{N}}$ và ${(b_{n})}_{n\in\mathbb{N}} < {(c_{n})}_{n\in\mathbb{N}}$.

              Theo Định lý~\ref{theorem:strictly-precedence-cauchy-sequence}, tồn tại số hữu tỉ dương $q$ sao cho tồn tại số tự nhiên $N_{bc}$ sao cho với mọi số tự nhiên $n\geq N_{bc}$, có $b_{n} - c_{n}\leq -q$.

              Theo định nghĩa quan hệ tương đương giữa các dãy số hữu tỉ, vẫn là với số hữu tỉ dương $q$, tồn tại số tự nhiên $N_{ab}$ sao cho với mọi số tự nhiên $n\geq N_{ab}$, có $\abs{a_{n} - b_{n}} < q$.

              Chúng ta định nghĩa $N = \max\{ N_{bc}, N_{ab} \}$. Khi đó, với mọi số tự nhiên $n\geq N$, chúng ta có $b_{n} - c_{n}\leq -q$, $\abs{a_{n} - b_{n}} < q$, và
              \[
                  a_{n} - c_{n} = (a_{n} - b_{n}) + (b_{n} - c_{n})\leq \abs{a_{n} - b_{n}} + (b_{n} - c_{n})\leq q + (-q) = 0
              \]

              hay nói cách khác, $a_{n}\leq c_{n}$ với mọi số tự nhiên $n\geq N$.

              Theo định nghĩa quan hệ $\lesssim$ giữa các dãy Cauchy hữu tỉ, ${(a_{n})}_{n\in\mathbb{N}}\lesssim {(c_{n})}_{n\in\mathbb{N}}$.
        \item ${(a_{n})}_{n\in\mathbb{N}} < {(b_{n})}_{n\in\mathbb{N}}$ và ${(b_{n})}_{n\in\mathbb{N}}\sim {(c_{n})}_{n\in\mathbb{N}}$.

              Chúng ta thực hiện hoàn toàn tương tự \textbf{Trường hợp 2}.

              Theo Định lý~\ref{theorem:strictly-precedence-cauchy-sequence}, tồn tại số hữu tỉ dương $q$ sao cho tồn tại số tự nhiên $N_{ab}$ sao cho với mọi số tự nhiên $n\geq N_{ab}$, có $a_{n} - b_{n}\leq -q$.

              Theo định nghĩa quan hệ tương đương giữa các dãy số hữu tỉ, vẫn là với số hữu tỉ dương $q$, tồn tại số tự nhiên $N_{bc}$ sao cho với mọi số tự nhiên $n\geq N_{bc}$, có $\abs{b_{n} - c_{n}} < q$.

              Chúng ta định nghĩa $N = \max\{ N_{ab}, N_{bc} \}$. Khi đó, với mọi số tự nhiên $n\geq N$, chúng ta có $a_{n} - b_{n}\leq -q$, $\abs{b_{n} - c_{n}} < q$, và
              \[
                  a_{n} - c_{n} = (a_{n} - b_{n}) + (b_{n} - c_{n})\leq (a_{n} - b_{n}) + \abs{b_{n} - c_{n}}\leq (-q) + q = 0
              \]

              hay nói cách khác, $a_{n}\leq c_{n}$ với mọi số tự nhiên $n\geq N$.

              Theo định nghĩa quan hệ $\lesssim$ giữa các dãy Cauchy hữu tỉ, ${(a_{n})}_{n\in\mathbb{N}}\lesssim {(c_{n})}_{n\in\mathbb{N}}$.
        \item ${(a_{n})}_{n\in\mathbb{N}} < {(b_{n})}_{n\in\mathbb{N}}$ và ${(b_{n})}_{n\in\mathbb{N}} < {(c_{n})}_{n\in\mathbb{N}}$.

              Theo Định lý~\ref{theorem:strictly-precedence-cauchy-sequence}
              \begin{itemize}
                  \item tồn tại số hữu tỉ dương $q_{ab}$ sao cho tồn tại số tự nhiên $N_{ab}$ sao cho với mọi số tự nhiên $n\geq N_{ab}$, có $a_{n} - b_{n}\leq -q_{ab}$,
                  \item tồn tại số hữu tỉ dương $q_{bc}$ sao cho tồn tại số tự nhiên $N_{bc}$ sao cho với mọi số tự nhiên $n\geq N_{bc}$, có $b_{n} - c_{n}\leq -q_{bc}$.
              \end{itemize}

              Chúng ta định nghĩa $q = q_{ab} + q_{bc}$, $N = \max\{ N_{ab}, N_{bc} \}$. Với mọi số tự nhiên $n\geq N$, chúng ta có
              \[
                  a_{n} - c_{n} = (a_{n} - b_{n}) + (b_{n} - c_{n}) \leq (-q_{ab}) + (-q_{bc}) = -q .
              \]

              Theo Định lý~\ref{theorem:strictly-precedence-cauchy-sequence}, chúng ta suy ra ${(a_{n})}_{n\in\mathbb{N}} < {(c_{n})}_{n\in\mathbb{N}}$, kéo theo ${(a_{n})}_{n\in\mathbb{N}}\lesssim {(c_{n})}_{n\in\mathbb{N}}$.
    \end{enumerate}

    Các trường hợp khẳng định rằng quan hệ $\lesssim$ giữa các dãy Cauchy hữu tỉ có tính chất bắc cầu. Đến lúc này, chúng ta đã chứng minh quan hệ $\lesssim$ giữa các dãy Cauchy hữu tỉ là một quan hệ tiền thứ tự.

    Xét hai dãy Cauchy hữu tỉ bất kì ${(a_{n})}_{n\in\mathbb{N}}$ và ${(b_{n})}_{n\in\mathbb{N}}$. Theo Hệ quả~\ref{corollary:linear-combination-of-cauchy-sequences}, ${(a_{n} - b_{n})}_{n\in\mathbb{N}}$ là một dãy Cauchy hữu tỉ. Chúng ta xét các trường hợp sau.
    \begin{enumerate}[label={\textbf{Trường hợp \arabic*.}},itemindent=2cm]
        \item ${(a_{n})}_{n\in\mathbb{N}}$ và ${(b_{n})}_{n\in\mathbb{N}}$ tương đương.

              Theo định nghĩa quan hệ $\lesssim$ giữa các dãy Cauchy hữu tỉ, chúng ta suy ra ${(a_{n})}_{n\in\mathbb{N}}\lesssim {(b_{n})}_{n\in\mathbb{N}}$.
        \item ${(a_{n})}_{n\in\mathbb{N}}$ và ${(b_{n})}_{n\in\mathbb{N}}$ không tương đương.

              Khi đó, dãy Cauchy hữu tỉ ${(a_{n} - b_{n})}_{n\in\mathbb{N}}$ không hội tụ đến $0$. Theo Định lý~\ref{theorem:nonzero-cauchy-sequences-and-preorder}, chỉ có một trong hai khả năng sau:
              \begin{itemize}
                  \item Tồn tại số hữu tỉ dương $q$ sao cho tồn tại số tự nhiên $N$ sao cho với mọi số tự nhiên $n\geq N$, có $a_{n} - b_{n}\geq q$.

                        Theo Định lý~\ref{theorem:strictly-precedence-cauchy-sequence}, khả năng này tương đương với ${(b_{n})}_{n\in\mathbb{N}} < {(a_{n})}_{n\in\mathbb{N}}$, kéo theo ${(b_{n})}_{n\in\mathbb{N}}\lesssim {(a_{n})}_{n\in\mathbb{N}}$.
                  \item Tồn tại số hữu tỉ dương $q$ sao cho tồn tại số tự nhiên $N$ sao cho với mọi số tự nhiên $n\geq N$, có $a_{n} - b_{n}\leq -q$.

                        Theo Định lý~\ref{theorem:strictly-precedence-cauchy-sequence}, khả năng này tương đương với ${(a_{n})}_{n\in\mathbb{N}} < {(b_{n})}_{n\in\mathbb{N}}$, kéo theo ${(a_{n})}_{n\in\mathbb{N}}\lesssim {(b_{n})}_{n\in\mathbb{N}}$.
              \end{itemize}
    \end{enumerate}

    Cả hai trường hợp đều kéo theo ${(b_{n})}_{n\in\mathbb{N}}\lesssim {(a_{n})}_{n\in\mathbb{N}}$ hoặc ${(a_{n})}_{n\in\mathbb{N}}\lesssim {(b_{n})}_{n\in\mathbb{N}}$.

    Như vậy, quan hệ $\lesssim$ trên tập hợp các dãy số Cauchy hữu tỉ là một quan hệ tiền thứ tự toàn phần.
\end{proof}

Tuy quan hệ $\lesssim$ trên tập hợp $\mathscr{C}_{\mathbb{Q}}$ không phải một quan hệ thứ tự vì thiếu tính chất phản đối xứng, nhưng quan hệ này có một tính chất gần với tính chất phản đối xứng, được nêu trong định lý sau.

\begin{theorem}\label{theorem:cauchy-sequences-pre-antisymmetric}
    Nếu hai dãy Cauchy hữu tỉ ${(a_{n})}_{n\in\mathbb{N}}$ và ${(b_{n})}_{n\in\mathbb{N}}$ thỏa mãn ${(a_{n})}_{n\in\mathbb{N}}\lesssim {(b_{n})}_{n\in\mathbb{N}}$ và ${(b_{n})}_{n\in\mathbb{N}}\lesssim {(a_{n})}_{n\in\mathbb{N}}$ thì ${(a_{n})}_{n\in\mathbb{N}}\sim {(b_{n})}_{n\in\mathbb{N}}$.
\end{theorem}

\begin{proof}
    Giả sử phản chứng rằng hai dãy Cauchy hữu tỉ ${(a_{n})}_{n\in\mathbb{N}}$ và ${(b_{n})}_{n\in\mathbb{N}}$ không tương đương.

    Cùng với giả thiết ${(a_{n})}_{n\in\mathbb{N}}\lesssim {(b_{n})}_{n\in\mathbb{N}}$ và ${(b_{n})}_{n\in\mathbb{N}}\lesssim {(a_{n})}_{n\in\mathbb{N}}$, chúng ta suy ra  ${(a_{n})}_{n\in\mathbb{N}} < {(b_{n})}_{n\in\mathbb{N}}$ và ${(b_{n})}_{n\in\mathbb{N}} < {(a_{n})}_{n\in\mathbb{N}}$. Theo Định lý~\ref{theorem:strictly-precedence-cauchy-sequence}, chúng ta suy ra
    \begin{itemize}
        \item Tồn tại số hữu tỉ dương $q_{1}$ sao cho tồn tại số tự nhiên $N_{1}$ sao cho với mọi số tự nhiên $n\geq N_{1}$, có $a_{n} - b_{n}\leq -q_{1}$.
        \item Tồn tại số hữu tỉ dương $q_{2}$ sao cho tồn tại số tự nhiên $N_{2}$ sao cho với mọi số tự nhiên $n\geq N_{2}$, có $b_{n} - a_{n}\leq -q_{2}$.
    \end{itemize}

    Chúng ta định nghĩa $N = \max\{ N_{1}, N_{2} \}$. Khi đó, $a_{N} - b_{N}\leq -q_{1}$ và $b_{N} - a_{N}\leq -q_{2}$, kéo theo $a_{N} < b_{N}$ và $b_{N} < a_{N}$. Đây là một điều vô lý vì $a_{N} < b_{N}$ và $b_{N} < a_{N}$ không thể xảy ra đồng thời. Do đó giả sử phản chứng là sai.

    Vậy nếu hai dãy Cauchy hữu tỉ ${(a_{n})}_{n\in\mathbb{N}}$ và ${(b_{n})}_{n\in\mathbb{N}}$ thỏa mãn ${(a_{n})}_{n\in\mathbb{N}}\lesssim {(b_{n})}_{n\in\mathbb{N}}$ và ${(b_{n})}_{n\in\mathbb{N}}\lesssim {(a_{n})}_{n\in\mathbb{N}}$ thì ${(a_{n})}_{n\in\mathbb{N}}\sim {(b_{n})}_{n\in\mathbb{N}}$.
\end{proof}

Với các định lý trên, chúng ta đã có đủ cơ sở và công cụ để định nghĩa và chứng minh các tính chất của quan hệ thứ tự trên tập hợp $\mathscr{C}_{\mathbb{Q}}/_{\sim}$.

\begin{theorem}\label{theorem:equivalent-cauchy-sequences-and-preorder}
    Nếu các dãy Cauchy hữu tỉ ${(a_{n})}_{n\in\mathbb{N}}$, ${(b_{n})}_{n\in\mathbb{N}}$, ${(c_{n})}_{n\in\mathbb{N}}$, ${(d_{n})}_{n\in\mathbb{N}}$ thỏa mãn ${(a_{n})}_{n\in\mathbb{N}}\sim {(b_{n})}_{n\in\mathbb{N}}$ và ${(c_{n})}_{n\in\mathbb{N}}\sim {(d_{n})}_{n\in\mathbb{N}}$ thì ${(a_{n})}_{n\in\mathbb{N}}\lesssim {(c_{n})}_{n\in\mathbb{N}}$ khi và chỉ khi ${(b_{n})}_{n\in\mathbb{N}}\lesssim {(d_{n})}_{n\in\mathbb{N}}$.
\end{theorem}

\begin{proof}
    Do tính đối xứng trong phát biểu, chúng ta chỉ cần chứng minh ${(a_{n})}_{n\in\mathbb{N}}\lesssim {(c_{n})}_{n\in\mathbb{N}}$ kéo theo ${(b_{n})}_{n\in\mathbb{N}}\lesssim {(d_{n})}_{n\in\mathbb{N}}$.

    \begin{enumerate}[label={\textbf{Trường hợp \arabic*.}},itemindent=2cm]
        \item ${(a_{n})}_{n\in\mathbb{N}}$ và ${(c_{n})}_{n\in\mathbb{N}}$ tương đương.

              Vì quan hệ $\sim$ giữa các dãy Cauchy hữu tỉ là một quan hệ tương đương nên theo tính chất bắc cầu, chúng ta suy ra ${(b_{n})}_{n\in\mathbb{N}}\sim {(d_{n})}_{n\in\mathbb{N}}$. Theo định nghĩa quan hệ $\lesssim$ giữa các dãy Cauchy hữu tỉ, ${(b_{n})}_{n\in\mathbb{N}}\lesssim {(d_{n})}_{n\in\mathbb{N}}$.

        \item ${(a_{n})}_{n\in\mathbb{N}}$ và ${(c_{n})}_{n\in\mathbb{N}}$ không tương đương.

              Cùng với ${(a_{n})}_{n\in\mathbb{N}}\lesssim {(c_{n})}_{n\in\mathbb{N}}$, chúng ta suy ra ${(a_{n})}_{n\in\mathbb{N}} < {(c_{n})}_{n\in\mathbb{N}}$.  Theo Định lý~\ref{theorem:strictly-precedence-cauchy-sequence}, tồn tại số hữu tỉ dương $q$ sao cho tồn tại số tự nhiên $N_{ac}$ sao cho với mọi số tự nhiên $n\geq N_{ac}$, có $a_{n} - c_{n}\leq -q$.

              Vì ${(a_{n})}_{n\in\mathbb{N}}\sim {(b_{n})}_{n\in\mathbb{N}}$ và ${(c_{n})}_{n\in\mathbb{N}}\sim {(d_{n})}_{n\in\mathbb{N}}$ nên theo định nghĩa quan hệ $\sim$ giữa các dãy số hữu tỉ, chúng ta suy ra
              \begin{itemize}
                  \item Với số hữu tỉ dương $\dfrac{q}{3}$, tồn tại số tự nhiên $N_{ab}$ sao cho với mọi số tự nhiên $n\geq N_{ab}$, có $\abs{a_{n} - b_{n}} < \dfrac{q}{3}$.
                  \item Với số hữu tỉ dương $\dfrac{q}{3}$, tồn tại số tự nhiên $N_{cd}$ sao cho với mọi số tự nhiên $n\geq N_{cd}$, có $\abs{c_{n} - d_{n}} < \dfrac{q}{3}$.
              \end{itemize}

              Chúng ta định nghĩa $N = \max\{ N_{ac}, N_{ab}, N_{cd} \}$. Với mọi số tự nhiên $n\geq N$, chúng ta có
              \[
                  b_{n} - d_{n} = (b_{n} - a_{n}) + (a_{n} - c_{n}) + (c_{n} - d_{n}) \leq \abs{b_{n} - a_{n}} + (a_{n} - c_{n}) + \abs{c_{n} - d_{n}} \leq \frac{q}{3} + (-q) + \frac{q}{3} = -\frac{q}{3}
              \]

              Theo Định lý~\ref{theorem:strictly-precedence-cauchy-sequence}, chúng ta suy ra ${(b_{n})}_{n\in\mathbb{N}} < {(d_{n})}_{n\in\mathbb{N}}$, kéo theo ${(b_{n})}_{n\in\mathbb{N}}\lesssim {(d_{n})}_{n\in\mathbb{N}}$.
    \end{enumerate}

    Như vậy, nếu ${(a_{n})}_{n\in\mathbb{N}}$, ${(b_{n})}_{n\in\mathbb{N}}$, ${(c_{n})}_{n\in\mathbb{N}}$, ${(d_{n})}_{n\in\mathbb{N}}$ là các dãy Cauchy hữu tỉ thỏa mãn ${(a_{n})}_{n\in\mathbb{N}}\sim {(b_{n})}_{n\in\mathbb{N}}$ và ${(c_{n})}_{n\in\mathbb{N}}\sim {(d_{n})}_{n\in\mathbb{N}}$ thì ${(a_{n})}_{n\in\mathbb{N}}\lesssim {(c_{n})}_{n\in\mathbb{N}}$ kéo theo ${(b_{n})}_{n\in\mathbb{N}}\lesssim {(d_{n})}_{n\in\mathbb{N}}$. Tương tự, theo chiều ngược lại, chúng ta cũng có ${(b_{n})}_{n\in\mathbb{N}}\lesssim {(d_{n})}_{n\in\mathbb{N}}$ kéo theo ${(a_{n})}_{n\in\mathbb{N}}\lesssim {(c_{n})}_{n\in\mathbb{N}}$.
\end{proof}

Định lý trên được sử dụng làm cơ sở cho định nghĩa sau.
\begin{definition}
    Hai phần tử $\alpha, \beta$ của tập thương $\mathscr{C}_{\mathbb{Q}}/_{\sim}$ được gọi là có quan hệ $\leq$ nếu và chỉ nếu với mỗi dãy Cauchy hữu tỉ ${(a_{n})}_{n\in\mathbb{N}}$ trong $\alpha$ và dãy Cauchy hữu tỉ ${(b_{n})}_{n\in\mathbb{N}}$ trong $\beta$, chúng ta có ${(a_{n})}_{n\in\mathbb{N}}\lesssim {(b_{n})}_{n\in\mathbb{N}}$.
\end{definition}

\begin{theorem}\label{theorem:preorder-to-order}
    Nếu hai dãy Cauchy hữu tỉ ${(a_{n})}_{n\in\mathbb{N}}$ và ${(b_{n})}_{n\in\mathbb{N}}$ thỏa mãn ${(a_{n})}_{n\in\mathbb{N}}\lesssim {(b_{n})}_{n\in\mathbb{N}}$ thì $\clsseq{a_{n}}{n}\leq \clsseq{b_{n}}{n}$.
\end{theorem}

\begin{proof}
    Giả sử ${(x_{n})}_{n\in\mathbb{N}}$ là một dãy Cauchy hữu tỉ tương đương với ${(a_{n})}_{n\in\mathbb{N}}$, và ${(y_{n})}_{n\in\mathbb{N}}$ là một dãy Cauchy hữu tỉ tương đương với ${(b_{n})}_{n\in\mathbb{N}}$.

    Theo Định lý~\ref{theorem:equivalent-cauchy-sequences-and-preorder}, ${(a_{n})}_{n\in\mathbb{N}}\lesssim {(b_{n})}_{n\in\mathbb{N}}$ kéo theo ${(x_{n})}_{n\in\mathbb{N}}\lesssim {(y_{n})}_{n\in\mathbb{N}}$.

    Vì chúng ta đang xét các dãy Cauchy hữu tỉ ${(x_{n})}_{n\in\mathbb{N}}$ và ${(y_{n})}_{n\in\mathbb{N}}$ bất kì từ hai lớp tương đương $\clsseq{a_{n}}{n}$ và $\clsseq{b_{n}}{n}$ nên theo định nghĩa quan hệ $\leq$ trên tập hợp $\mathscr{C}_{\mathbb{Q}}/_{\sim}$, chúng ta suy ra $\clsseq{a_{n}}{n}\leq \clsseq{b_{n}}{n}$.
\end{proof}

\begin{theorem}
    Quan hệ $\leq$ trên tập hợp $\mathscr{C}_{\mathbb{Q}}/_{\sim}$ là một quan hệ thứ tự toàn phần.
\end{theorem}

\begin{proof}
    Với mỗi phần tử $\alpha$ của $\mathscr{C}_{\mathbb{Q}}/_{\sim}$, chúng ta có $\alpha\leq\alpha$ vì hai phần tử bất kì của lớp tương đương $\alpha$ đều tương đương với nhau (và vì vậy có quan hệ $\lesssim$). Do đó quan hệ $\leq$ trên tập hợp $\mathscr{C}_{\mathbb{Q}}/_{\sim}$ có tính chất phản xạ.

    Nếu các phần tử $\alpha, \beta$ của $\mathscr{C}_{\mathbb{Q}}/_{\sim}$ thỏa mãn $\alpha\leq \beta$ và $\beta\leq \alpha$ thì với mỗi phần tử ${(a_{n})}_{n\in\mathbb{N}}$ trong lớp tương đương $\alpha$ và phần tử ${(b_{n})}_{n\in\mathbb{N}}$ trong lớp tương đương $\beta$, chúng ta có ${(a_{n})}_{n\in\mathbb{N}}\lesssim {(b_{n})}_{n\in\mathbb{N}}$ và ${(b_{n})}_{n\in\mathbb{N}}\lesssim {(a_{n})}_{n\in\mathbb{N}}$. Theo Định lý~\ref{theorem:cauchy-sequences-pre-antisymmetric}, ${(a_{n})}_{n\in\mathbb{N}}\sim {(b_{n})}_{n\in\mathbb{N}}$, kéo theo $\alpha = \beta$. Do đó quan hệ $\leq$ trên tập hợp $\mathscr{C}_{\mathbb{Q}}/_{\sim}$ có tính chất phản đối xứng.

    Nếu các phần tử $\alpha, \beta, \gamma$ của $\mathscr{C}_{\mathbb{Q}}/_{\sim}$ thỏa mãn $\alpha\leq \beta$ và $\beta\leq \gamma$ thì với mỗi phần tử ${(a_{n})}_{n\in\mathbb{N}}$ trong lớp tương đương $\alpha$, phần tử ${(b_{n})}_{n\in\mathbb{N}}$ trong lớp tương đương $\beta$, phần tử ${(c_{n})}_{n\in\mathbb{N}}$ trong lớp tương đương $\gamma$, chúng ta có ${(a_{n})}_{n\in\mathbb{N}}\lesssim {(b_{n})}_{n\in\mathbb{N}}$ và ${(b_{n})}_{n\in\mathbb{N}}\lesssim {(c_{n})}_{n\in\mathbb{N}}$. Vì quan hệ $\lesssim$ trong tập hợp $\mathscr{C}_{\mathbb{Q}}/_{\sim}$ có tính chất bắc cầu nên ${(a_{n})}_{n\in\mathbb{N}}\lesssim {(c_{n})}_{n\in\mathbb{N}}$. Theo Định lý~\ref{theorem:preorder-to-order}, chúng ta suy ra $\alpha\leq \gamma$. Do đó quan hệ $\leq$ trên tập hợp $\mathscr{C}_{\mathbb{Q}}/_{\sim}$ có tính chất bắc cầu.

    Như vậy quan hệ $\leq$ trên tập hợp $\mathscr{C}_{\mathbb{Q}}/_{\sim}$ là một quan hệ thứ tự.

    Xét hai phần tử bất kì $\alpha$ và $\beta$ trong $\mathscr{C}_{\mathbb{Q}}/_{\sim}$. Với mỗi phần tử ${(a_{n})}_{n\in\mathbb{N}}$ trong lớp tương đương $\alpha$, phần tử ${(b_{n})}_{n\in\mathbb{N}}$ trong lớp tương đương $\beta$, chúng ta có ${(a_{n})}_{n\in\mathbb{N}}\lesssim {(b_{n})}_{n\in\mathbb{N}}$ hoặc ${(b_{n})}_{n\in\mathbb{N}}\lesssim {(a_{n})}_{n\in\mathbb{N}}$ vì quan hệ $\lesssim$ trên tập hợp $\mathscr{C}_{\mathbb{Q}}/_{\sim}$ là một quan hệ thứ tự toàn phần. Theo Định lý~\ref{theorem:preorder-to-order}, chúng ta suy ra $\alpha\leq \beta$ hoặc $\beta\leq \alpha$.

    Vậy quan hệ $\leq$ trên tập hợp $\mathscr{C}_{\mathbb{Q}}/_{\sim}$ là một quan hệ thứ tự toàn phần.
\end{proof}

\begin{theorem}\label{theorem:addition-multiplication-and-preorder-of-cauchy-sequences}
    \begin{enumerate}[label={(\roman*)}]
        \item Nếu các dãy Cauchy hữu tỉ ${(a_{n})}_{n\in\mathbb{N}}, {(b_{n})}_{n\in\mathbb{N}}$ thỏa mãn ${(a_{n})}_{n\in\mathbb{N}}\lesssim {(b_{n})}_{n\in\mathbb{N}}$ thì với mọi dãy Cauchy hữu tỉ ${(c_{n})}_{n\in\mathbb{N}}$, chúng ta có ${(a_{n})}_{n\in\mathbb{N}} + {(c_{n})}_{n\in\mathbb{N}}\lesssim {(b_{n})}_{n\in\mathbb{N}} + {(c_{n})}_{n\in\mathbb{N}}$.
        \item  Nếu các dãy Cauchy hữu tỉ ${(a_{n})}_{n\in\mathbb{N}}, {(b_{n})}_{n\in\mathbb{N}}$ thỏa mãn ${(0)}_{n\in\mathbb{N}}\lesssim {(a_{n})}_{n\in\mathbb{N}}$ và ${(0)}_{n\in\mathbb{N}}\lesssim {(b_{n})}_{n\in\mathbb{N}}$ thì chúng ta có ${(0)}_{n\in\mathbb{N}}\lesssim {(a_{n})}_{n\in\mathbb{N}}\cdot {(b_{n})}_{n\in\mathbb{N}}$.
    \end{enumerate}
\end{theorem}

\begin{proof}
    \begin{enumerate}[label={(\roman*)}]
        \item Chúng ta xét đủ hai trường hợp sau.
              \begin{enumerate}[label={\textbf{Trường hợp \arabic*.}},itemindent=1cm]
                  \item ${(a_{n})}_{n\in\mathbb{N}}$ và ${(b_{n})}_{n\in\mathbb{N}}$ tương đương.

                        Cùng với ${(c_{n})}_{n\in\mathbb{N}}\sim {(c_{n})}_{n\in\mathbb{N}}$, theo Định lý~\ref{theorem:basis-of-equivalence-class-of-rational-cauchy-sequences-addition}, chúng ta suy ra ${(a_{n})}_{n\in\mathbb{N}} + {(c_{n})}_{n\in\mathbb{N}}\sim {(b_{n})}_{n\in\mathbb{N}} + {(c_{n})}_{n\in\mathbb{N}}$. Điều này kéo theo ${(a_{n})}_{n\in\mathbb{N}} + {(c_{n})}_{n\in\mathbb{N}}\lesssim {(b_{n})}_{n\in\mathbb{N}} + {(c_{n})}_{n\in\mathbb{N}}$.
                  \item ${(a_{n})}_{n\in\mathbb{N}}$ và ${(b_{n})}_{n\in\mathbb{N}}$ không tương đương.

                        Vì ${(a_{n})}_{n\in\mathbb{N}}\lesssim {(b_{n})}_{n\in\mathbb{N}}$ và ${(a_{n})}_{n\in\mathbb{N}}$, ${(b_{n})}_{n\in\mathbb{N}}$ không tương đương nên ${(a_{n})}_{n\in\mathbb{N}} < {(b_{n})}_{n\in\mathbb{N}}$. Theo Định lý~\ref{theorem:strictly-precedence-cauchy-sequence}, tồn tại số hữu tỉ dương $q$ sao cho tồn tại số tự nhiên $N$ sao cho với mọi số tự nhiên $n\geq N$, có $a_{n} - b_{n}\leq -q$. Từ điều này, chúng ta suy ra rằng: với số hữu tỉ dương $q$, với số tự nhiên $N$ thì với mọi số tự nhiên $n\geq N$, chúng ta có $(a_{n} + c_{n}) - (b_{n} + c_{n}) = a_{n} - b_{n}\leq -q$. Một lần nữa, cũng theo Định lý~\ref{theorem:strictly-precedence-cauchy-sequence}, chúng ta suy ra ${(a_{n} + c_{n})}_{n\in\mathbb{N}} < {(b_{n} + c_{n})}_{n\in\mathbb{N}}$.

                        Mặt khác, theo định nghĩa phép cộng dãy Cauchy hữu tỉ trong Định lý~\ref{theorem:addition-and-multiplication-of-cauchy-sequences}, chúng ta có
                        \[
                            {(a_{n} + c_{n})}_{n\in\mathbb{N}} = {(a_{n})}_{n\in\mathbb{N}} + {(c_{n})}_{n\in\mathbb{N}},\qquad  {(b_{n} + c_{n})}_{n\in\mathbb{N}} = {(b_{n})}_{n\in\mathbb{N}} + {(c_{n})}_{n\in\mathbb{N}}.
                        \]

                        Do đó ${(a_{n})}_{n\in\mathbb{N}} + {(c_{n})}_{n\in\mathbb{N}} < {(b_{n})}_{n\in\mathbb{N}} + {(c_{n})}_{n\in\mathbb{N}}$, kéo theo ${(a_{n})}_{n\in\mathbb{N}} + {(c_{n})}_{n\in\mathbb{N}}\lesssim {(b_{n})}_{n\in\mathbb{N}} + {(c_{n})}_{n\in\mathbb{N}}$.
              \end{enumerate}
        \item Chúng ta xét đủ hai trường hợp sau.
              \begin{enumerate}[label={\textbf{Trường hợp \arabic*.}},itemindent=1cm]
                  \item ${(0)}_{n\in\mathbb{N}}\sim {(a_{n})}_{n\in\mathbb{N}}$ hoặc ${(0)}_{n\in\mathbb{N}} \sim {(b_{n})}_{n\in\mathbb{N}}$ (ít nhất một trong hai mệnh đề này đúng).

                        Không mất tính tổng quát, giả sử ${(0)}_{n\in\mathbb{N}}\sim {(a_{n})}_{n\in\mathbb{N}}$. Chúng ta chọn số hữu tỉ dương $\varepsilon$ bất kì.

                        Vì ${(b_{n})}_{n\in\mathbb{N}}$ là một dãy Cauchy hữu tỉ nên theo Định lý~\ref{theorem:cauchy-sequences-are-bounded}, tồn tại một số hữu tỉ dương $B$ sao cho $\abs{b_{n}}\leq B$ với mọi số tự nhiên $n$. Mặt khác, ${(0)}_{n\in\mathbb{N}}\sim {(a_{n})}_{n\in\mathbb{N}}$, do đó, theo định nghĩa quan hệ $\sim$ giữa các dãy số hữu tỉ, với số hữu tỉ dương $\dfrac{\varepsilon}{B}$, tồn tại số tự nhiên $N$ sao cho với mọi số tự nhiên $n\geq N$, có $\abs{a_{n}} < \dfrac{\varepsilon}{B}$. Do đó, với mọi số tự nhiên $n\geq N$, chúng ta có
                        \[
                            \abs{a_{n}b_{n} - 0} = \abs{a_{n}b_{n}} = \abs{a_{n}}\cdot\abs{b_{n}} < \dfrac{\varepsilon}{B}\cdot B = \varepsilon.
                        \]

                        Theo định nghĩa quan hệ $\sim$ giữa các dãy số hữu tỉ, chúng ta suy ra ${(a_{n}b_{n})}_{n\in\mathbb{N}}\sim {(0)}_{n\in\mathbb{N}}$. Bên cạnh đó, theo định nghĩa phép nhân dãy Cauchy hữu tỉ trong Định lý~\ref{theorem:addition-and-multiplication-of-cauchy-sequences}, ${(a_{n}b_{n})}_{n\in\mathbb{N}} = {(a_{n})}_{n\in\mathbb{N}}\cdot {(b_{n})}_{n\in\mathbb{N}}$. Do vậy ${(0)}_{n\in\mathbb{N}}\sim {(a_{n})}_{n\in\mathbb{N}}\cdot {(b_{n})}_{n\in\mathbb{N}}$, kéo theo ${(0)}_{n\in\mathbb{N}}\lesssim {(a_{n})}_{n\in\mathbb{N}}\cdot {(b_{n})}_{n\in\mathbb{N}}$.
                  \item ${(0)}_{n\in\mathbb{N}} < {(a_{n})}_{n\in\mathbb{N}}$ và ${(0)}_{n\in\mathbb{N}} < {(b_{n})}_{n\in\mathbb{N}}$.

                        Theo Định lý~\ref{theorem:strictly-precedence-cauchy-sequence}, chúng ta suy ra
                        \begin{itemize}
                            \item tồn tại số hữu tỉ dương $a$ sao cho tồn tại số tự nhiên $N_{a}$ sao cho với mọi số tự nhiên $n\geq N_{a}$, có $-a_{n}\leq -a$,
                            \item tồn tại số hữu tỉ dương $a$ sao cho tồn tại số tự nhiên $N_{b}$ sao cho với mọi số tự nhiên $n\geq N_{b}$, có $-b_{n}\leq -b$.
                        \end{itemize}

                        Chúng ta định nghĩa $q = ab$ và $N = \max\{ N_{a}, N_{b} \}$. Với mọi số tự nhiên $n\geq N$, chúng ta có $a_{n}\geq 0, b_{n}\geq 0$ và $a_{n}b_{n}\geq a_{n}b \geq ab$, kéo theo $-a_{n}b_{n}\leq -ab = -q$. Theo Định lý~\ref{theorem:strictly-precedence-cauchy-sequence}, chúng ta suy ra ${(0)}_{n\in\mathbb{N}} < {(a_{n}b_{n})}_{n\in\mathbb{N}}$. Theo định nghĩa phép nhân dãy Cauchy hữu tỉ trong Định lý~\ref{theorem:addition-and-multiplication-of-cauchy-sequences}, ${(a_{n}b_{n})}_{n\in\mathbb{N}} = {(a_{n})}_{n\in\mathbb{N}}\cdot {(b_{n})}_{n\in\mathbb{N}}$. Do vậy ${(0)}_{n\in\mathbb{N}} < {(a_{n})}_{n\in\mathbb{N}}\cdot {(b_{n})}_{n\in\mathbb{N}}$, kéo theo ${(0)}_{n\in\mathbb{N}} \lesssim {(a_{n})}_{n\in\mathbb{N}}\cdot {(b_{n})}_{n\in\mathbb{N}}$
              \end{enumerate}

              Như vậy nếu các dãy Cauchy hữu tỉ ${(a_{n})}_{n\in\mathbb{N}} và {(b_{n})}_{n\in\mathbb{N}}$ thỏa mãn ${(0)}_{n\in\mathbb{N}}\lesssim {(a_{n})}_{n\in\mathbb{N}}$ và ${(0)}_{n\in\mathbb{N}}\lesssim {(b_{n})}_{n\in\mathbb{N}}$ thì ${(0)}_{n\in\mathbb{N}}\lesssim {(a_{n})}_{n\in\mathbb{N}}\cdot {(b_{n})}_{n\in\mathbb{N}}$.
    \end{enumerate}
\end{proof}

\begin{proposition}\label{proposition:uniqueness-of-additive-inverse-cauchy-sequence}
    Trong tập hợp $\mathscr{C}_{\mathbb{Q}}/_{\sim}$, với mỗi phần tử $\alpha$ tồn tại duy nhất phần tử $\beta$ sao cho $\alpha + \beta = \beta + \alpha = \clsseq{0}{n}$ (chúng ta kí hiệu phần tử $\beta$ là $-\alpha$).
\end{proposition}

\begin{proof}
    Định lý~\ref{theorem:rational-cauchy-sequences-abelian-group} khẳng định sự tồn tại của một phần tử như vậy với mỗi phần tử $\alpha$ trong tập hợp $\mathscr{C}_{\mathbb{Q}}/_{\sim}$.

    Giả với với phần tử $\alpha$ trong tập hợp $\mathscr{C}_{\mathbb{Q}}/_{\sim}$, có hai phần tử $\beta, \beta'$ thỏa mãn
    \[
        \alpha + \beta = \beta + \alpha = \alpha + \beta' = \beta' + \alpha = \clsseq{0}{n}.
    \]

    Khi đó, theo phần (i), (ii), (iii) của Định lý~\ref{theorem:rational-cauchy-sequences-abelian-group}, chúng ta có
    \[
        \beta' = \beta' + \clsseq{0}{n} = \beta' + (\alpha + \beta) = (\beta' + \alpha) + \beta = \clsseq{0}{n} + \beta = \beta.
    \]

    Như vậy, trong tập hợp $\mathscr{C}_{\mathbb{Q}}/_{\sim}$, với mỗi phần tử $\alpha$ tồn tại duy nhất phần tử $\beta$ sao cho $\alpha + \beta = \beta + \alpha = \clsseq{0}{n}$.
\end{proof}

\begin{theorem}
    \begin{enumerate}[label={(\roman*)}]
        \item Nếu các phần tử $\alpha, \beta$ của $\mathscr{C}_{\mathbb{Q}}/_{\sim}$ thỏa mãn $\alpha\leq \beta$ thì với mọi phần tử $\gamma$ của $\mathscr{C}_{\mathbb{Q}}/_{\sim}$, chúng ta có $\alpha + \gamma\leq \beta + \gamma$.
        \item Nếu các phần tử $\alpha, \beta$ của $\mathscr{C}_{\mathbb{Q}}/_{\sim}$ thỏa mãn $\clsseq{0}{n} \leq \alpha$ và $\clsseq{0}{n}\leq \beta$ thì chúng ta có $\clsseq{0}{n}\leq \alpha\cdot\beta$.
    \end{enumerate}
\end{theorem}

\begin{proof}
    \begin{enumerate}[label={(\roman*)}]
        \item ${(x_{n})}_{n\in\mathbb{N}}$ và ${(y_{n})}_{n\in\mathbb{N}}$ lần lượt là hai phần tử của hai lớp tương đương $\alpha + \gamma$ và $\beta + \gamma$.

              ${(a_{n})}_{n\in\mathbb{N}}, {(b_{n})}_{n\in\mathbb{N}}$ lần lượt là các phần tử của các lớp tương đương $\alpha, \beta$. Chúng ta sẽ chứng minh rằng ${(x_{n} - a_{n})}_{n\in\mathbb{N}}$ và ${(y_{n} - b_{n})}_{n\in\mathbb{N}}$ thuộc lớp tương đương $\gamma$ để sử dụng Định lý~\ref{theorem:addition-multiplication-and-preorder-of-cauchy-sequences}.

              Theo định nghĩa phép cộng hai dãy Cauchy hữu tỉ, định nghĩa phép cộng trên tập hợp $\mathscr{C}_{\mathbb{Q}}/_{\sim}$, Định lý~\ref{theorem:rational-cauchy-sequences-abelian-group}, và Mệnh đề~\ref{proposition:uniqueness-of-additive-inverse-cauchy-sequence} chúng ta có
              \begin{align*}
                  \clsseq{x_{n} - a_{n}}{n} & = \clsseq{x_{n} - a_{n}}{n} + (\alpha + (-\alpha))            \\
                                            & = (\clsseq{x_{n} - a_{n}}{n} + \alpha) + (-\alpha)            \\
                                            & = (\clsseq{x_{n} - a_{n}}{n} + \clsseq{a_{n}}{n}) + (-\alpha) \\
                                            & = \clsseq{(x_{n} - a_{n}) + a_{n}}{n} + (-\alpha)             \\
                                            & = \clsseq{x_{n}}{n} + (-\alpha)                               \\
                                            & = (\alpha + \gamma) + (-\alpha)                               \\
                                            & = (\gamma + \alpha) + (-\alpha)                               \\
                                            & = \gamma + (\alpha + (-\alpha))                               \\
                                            & = \gamma + \clsseq{0}{n} = \gamma.
              \end{align*}

              Do đó ${(x_{n} - a_{n})}_{n\in\mathbb{N}}$ thuộc lớp tương đương $\gamma$. Hoàn toàn tương tự, chúng ta chứng minh được ${(y_{n} - b_{n})}_{n\in\mathbb{N}}$ thuộc lớp tương đương $\gamma$. Theo Định lý~\ref{theorem:addition-multiplication-and-preorder-of-cauchy-sequences}
              \begin{align*}
                  {(x_{n})}_{n\in\mathbb{N}} & = {(x_{n} - a_{n})}_{n\in\mathbb{N}} + {(a_{n})}_{n\in\mathbb{N}}         \\
                                             & \lesssim  {(y_{n} - b_{n})}_{n\in\mathbb{N}} + {(a_{n})}_{n\in\mathbb{N}} \\
                                             & \lesssim {(y_{n} - b_{n})}_{n\in\mathbb{N}} + {(b_{n})}_{n\in\mathbb{N}}  \\
                                             & = {(y_{n})}_{n\in\mathbb{N}}.
              \end{align*}

              Như vậy, với hai phần tử ${(x_{n})}_{n\in\mathbb{N}}$ và ${(y_{n})}_{n\in\mathbb{N}}$ bất kì của hai lớp tương đương $\alpha + \gamma$ và $\beta + \gamma$, chúng ta có ${(x_{n})}_{n\in\mathbb{N}}\lesssim {(y_{n})}_{n\in\mathbb{N}}$, điều này có nghĩa là $\alpha + \gamma \leq \beta + \gamma$.
        \item  ${(a_{n})}_{n\in\mathbb{N}}, {(b_{n})}_{n\in\mathbb{N}}$ lần lượt là các phần tử của các lớp tương đương $\alpha, \beta$. Theo định nghĩa quan hệ $\leq$ trên tập hợp $\mathscr{C}_{\mathbb{Q}}/_{\sim}$, chúng ta có ${(0)}_{n\in\mathbb{N}}\lesssim {(a_{n})}_{n\in\mathbb{N}}$ và ${(0)}_{n\in\mathbb{N}}\lesssim {(b_{n})}_{n\in\mathbb{N}}$. Theo Định lý~\ref{theorem:addition-multiplication-and-preorder-of-cauchy-sequences} và định nghĩa phép nhân dãy Cauchy hữu tỉ, chúng ta suy ra ${(0)}_{n\in\mathbb{N}} \lesssim {(a_{n})}_{n\in\mathbb{N}}\cdot{(b_{n})}_{n\in\mathbb{N}} = {(a_{n}b_{n})}_{n\in\mathbb{N}}$.

              Mặt khác, theo Định lý~\ref{theorem:preorder-to-order} và định nghĩa phép nhân dãy Cauchy hữu tỉ và phép nhân trên tập hợp $\mathscr{C}_{\mathbb{Q}}/_{\sim}$, chúng ta có
              \begin{align*}
                  \clsseq{0}{n} \leq \clsseq{a_{n}b_{n}}{n} = \left[{(a_{n})}_{n\in\mathbb{N}}\cdot{(b_{n})}_{n\in\mathbb{N}}\right] = \clsseq{a_{n}}{n}\cdot\clsseq{b_{n}}{n} = \alpha\cdot\beta.
              \end{align*}

              Vậy nếu các phần tử $\alpha, \beta$ của $\mathscr{C}_{\mathbb{Q}}/_{\sim}$ thỏa mãn $\clsseq{0}{n} \leq \alpha$ và $\clsseq{0}{n}\leq \beta$ thì chúng ta có $\clsseq{0}{n}\leq \alpha\cdot\beta$.
    \end{enumerate}
\end{proof}

\subsection{* Dãy Cauchy hữu tỉ và tiên đề về cận trên}

\subsection{Liên hệ lớp tương đương các dãy Cauchy hữu tỉ với số hữu tỉ}

\section{Các mô hình số thực}

\subsection{Sự tương đương của các mô hình số thực}

\subsection{Số hữu tỉ và số vô tỉ}

\subsection{Âm vô cực và dương vô cực}

\subsection{Nhìn nhận về các tập hợp số}

\section{Tiếp tục từ hệ tiên đề về số thực}

\subsection{Lưu ý về tiên đề và mô hình}

\subsection{Các tiên đề về trường}

\subsection{Các tiên đề về thứ tự và tiên đề về cận trên}

\subsection{Tính đầy đủ-Dedekind và tính đầy đủ-Cauchy}

\subsection{Hàm lũy thừa}

\subsection{Hàm mũ và hàm logarithm}

