\chapter{Số thực và Số phức}\label{chapter:real-numbers-and-complex-numbers}

\section{Số thực là gì?}

\subsection{Hệ tiên đề về số thực}

\begin{axiom}
    Tập hợp số thực được kí hiệu là $\mathbb{R}$. Các phần tử của $\mathbb{R}$ thỏa mãn ba nhóm tiên đề sau.

    \textbf{Các tiên đề về trường.} $\mathbb{R}$ có hai phép toán hai ngôi là phép cộng (được kí hiệu là $+$) và phép nhân (được kí hiệu là $\cdot$) và các phép toán này thỏa mãn các tính chất sau:
    \begin{enumerate}[label={(\roman*)}]
        \item Phép cộng có tính chất kết hợp.
        \item Phép cộng có phần tử đồng nhất.
        \item Mỗi số thực có phần tử đối.
        \item Phép cộng có tính chất giao hoán.
        \item Phép nhân có tính chất kết hợp.
        \item Phép nhân có tính chất phân phối với phép cộng.
        \item Phép nhân có phần tử đồng nhất.
        \item Phép nhân có tính chất giao hoán.
        \item Mỗi số thực khác $0$ có phần tử nghịch đảo.
    \end{enumerate}

    \textbf{Các tiên đề về thứ tự.} $\mathbb{R}$ có quan hệ $\leq$ thỏa mãn các tính chất sau:
    \begin{enumerate}[label={(\roman*)}]
        \item $\leq$ là một quan hệ thứ tự toàn phần.
        \item Với mọi số thực $x, y$, nếu $x\leq y$ thì với mọi số thực $z$, $x + z\leq y + z$.
        \item Với mọi số thực $x, y$, nếu $0\leq x$ và $0\leq y$ thì $0\leq x\cdot y$.
    \end{enumerate}

    \textbf{Tiên đề về cận trên.} Nếu một tập hợp con khác rỗng của $\mathbb{R}$ có cận trên thì cũng có cận trên nhỏ nhất.
\end{axiom}

\subsection{Xây dựng tập hợp số thực}

\section{Dãy Cauchy}

\subsection{Dãy Cauchy hữu tỉ}

\subsection{Quan hệ tương đương giữa các dãy Cauchy hữu tỉ}

\subsection{Các phép toán với các dãy Cauchy hữu tỉ}

\subsection{Quan hệ thứ tự giữa các lớp tương đương của các dãy Cauchy hữu tỉ}

\section{Tính đầy đủ của tập hợp số thực}

\subsection{Dãy Cauchy hữu tỉ và tiên đề về cận trên}

\subsection{Dãy Cauchy thực}

\subsection{Một số hệ quả quan trọng của tính đầy đủ}

\section{$\dagger$ Các cách xây dựng tập số thực}

\subsection{Sơ lược về lát cắt Dedekind}

\subsection{Tính duy nhất của tập số thực}

\section{Số phức}

\subsection{Định nghĩa số phức}

\subsection{Dạng lượng giác của số phức}

\subsection{Định lý cơ bản của đại số}

\subsection{$\dagger$ Sơ lược về số siêu phức}
