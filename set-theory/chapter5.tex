\chapter{Số thực}\label{chapter:real-numbers}

\section{Số thực là gì?}

\subsection{Hệ tiên đề về số thực}

\begin{axiom}
	Tập hợp số thực được kí hiệu là $\mathbb{R}$. Các phần tử của $\mathbb{R}$ thỏa mãn ba nhóm tiên đề sau.

	\textbf{Các tiên đề về trường.} $\mathbb{R}$ có hai phép toán hai ngôi là phép cộng (được kí hiệu là $+$) và phép nhân (được kí hiệu là $\cdot$) và các phép toán này thỏa mãn các tính chất sau:
	\begin{enumerate}[label={(\roman*)}]
		\item Phép cộng có tính chất kết hợp. Nói cách khác, với mọi số thực $x, y, z$, chúng ta có
		      \[
			      (x + y) + z = x + (y + z).
		      \]
		\item Phép cộng có phần tử đồng nhất. Nói cách khác, tồn tại số thực $0$ sao cho với mọi số thực $x$, chúng ta có
		      \[
			      x + 0 = 0 + x = x.
		      \]
		\item Mỗi số thực có phần tử đối. Nói cách khác, với mỗi số thực $x$, tồn tại số thực $(-x)$ thỏa mãn
		      \[
			      x + (-x) = (-x) + x = 0.
		      \]
		\item Phép cộng có tính chất giao hoán. Nói cách khác, với mọi số thực $x, y$, chúng ta có
		      \[
			      x + y = y + x.
		      \]
		\item Phép nhân có tính chất kết hợp. Nói cách khác, với mọi số thực $x, y, z$, chúng ta có
		      \[
			      (x \cdot y) \cdot z = x \cdot (y \cdot z).
		      \]
		\item Phép nhân có tính chất phân phối với phép cộng. Nói cách khác, với mọi số thực $x, y, z$, chúng ta có
		      \[
			      \begin{split}
				      (x + y)\cdot z = x\cdot z + y\cdot z, \\
				      z\cdot (x + y) = z\cdot x + z\cdot y.
			      \end{split}
		      \]

		\item Phép nhân có phần tử đồng nhất. Nói cách khác, tồn tại số thực $1$ sao cho với mọi số thực $x$, chúng ta có
		      \[
			      x + 1 = 1 + x = x.
		      \]
		\item Phép nhân có tính chất giao hoán. Nói cách khác, với mọi số thực $x, y$, chúng ta có
		      \[
			      x\cdot y = y\cdot x.
		      \]
		\item Mỗi số thực khác $0$ có phần tử nghịch đảo. Nói cách khác, với mỗi số thực $x\ne 0$, tồn tại số thực $x^{-1}$ sao cho
		      \[
			      x\cdot x^{-1} = x^{-1}\cdot x = 1.
		      \]
	\end{enumerate}

	\textbf{Các tiên đề về thứ tự.} $\mathbb{R}$ có quan hệ $\leq$ thỏa mãn các tính chất sau:
	\begin{enumerate}[label={(\roman*)}]
		\item $\leq$ là một quan hệ thứ tự toàn phần.
		\item Với mọi số thực $x, y$, nếu $x\leq y$ thì với mọi số thực $z$, $x + z\leq y + z$.
		\item Với mọi số thực $x, y$, nếu $0\leq x$ và $0\leq y$ thì $0\leq x\cdot y$.
	\end{enumerate}

	\textbf{Tiên đề về cận trên.} Nếu một tập hợp con khác rỗng của $\mathbb{R}$ có cận trên thì cũng có cận trên nhỏ nhất.
\end{axiom}

\subsection{Xây dựng tập hợp số thực bằng dãy Cauchy}

\section{Dãy Cauchy}

\subsection{Dãy Cauchy hữu tỉ}

\subsection{Quan hệ tương đương giữa các dãy Cauchy hữu tỉ}

\subsection{Các phép toán với dãy Cauchy hữu tỉ}

\subsection{Quan hệ tiền thứ tự giữa các dãy Cauchy hữu tỉ}

\subsection{Quan hệ thứ tự giữa các lớp tương đương của các dãy Cauchy hữu tỉ}

\section{Tính đầy đủ của tập hợp số thực}

\subsection{Dãy Cauchy hữu tỉ và tiên đề về cận trên}

\subsection{Dãy Cauchy thực}

\section{$\dagger$ Các cách xây dựng tập hợp số thực}

\subsection{Sơ lược về lát cắt Dedekind}

\subsection{Tính duy nhất của tập hợp số thực}

\subsection{So sánh hai cách xây dựng tập hợp số thực}

\section{Một số hệ quả từ hệ tiên đề về số thực}

\subsection{Nhóm tiên đề về trường}

\subsection{Nhóm tiên đề vế thứ tự}

\subsection{Tiên đề về tính đầy đủ}

\section{Mở đầu về giải tích thực}

\subsection{Dãy số}

\subsection{Chuỗi số}

\subsection{Giới hạn của hàm số}

\subsection{Hàm số liên tục}
