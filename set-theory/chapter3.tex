\chapter{Lực lượng của tập hợp}\label{chapter:cardinality}

Cho đến thời điểm này, mặc dù chúng ta đã bắt gặp và làm việc cùng sự vô hạn (chẳng hạn như các tập hợp có vô hạn phần tử), nhưng chúng ta chưa hề định nghĩa thế nào là vô hạn. Mục tiêu của chương này không phải là trả lời vô hạn là gì, mà là giới thiệu khái niệm lực lượng của tập hợp để từ đó tìm ra một số đặc điểm của các tập hợp có vô hạn phần tử.

\section{Các định nghĩa}

\subsection{Lực lượng và so sánh lực lượng giữa các tập hợp}

\subsection{Tập hợp lũy thừa}

\subsection{Bài tập}

\section{Lực lượng đếm được và Lực lượng không đếm được}

\subsection{Lực lượng đếm được}

\subsection{Lực lượng không đếm được}

\subsection{Bài tập}

\section{$\dagger$ Một số thông tin về lực lượng của tập hợp}

\subsection{Lục lượng của tập hợp số thực}

\subsection{Định lý Cantor-Schr\"{o}der-Bernstein}

\subsection{Giả thuyết continuum}
